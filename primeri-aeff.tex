\section{Primeri v \aeff{}} \label{sec:primeri-aeff}

Najprej si poglejmo nekaj primerov, na katerih lahko vidimo konkretne težave s katerimi se lahko srečamo ko programiramo. 
Funkcija, ki pogosto pride prav, je funkcija \emph{map}. V tem konkretnem primeru naj map uporabi funkcijo $f$, ki vrne $true$, če je število praštevilo in false sicer, na trojici $(3,42,97)$ in vrne $(f(3),f(42),f(97))$.  

\begin{lstlisting}[caption={TODO},label={prog:primer-map-seq}]
run let f = [fun x -> (x, x)] in
send op f;
promise (op' y -> 
<y>)
as p in
await p until <y'> in
return y'

\end{lstlisting}