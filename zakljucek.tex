\section{Zaključek}

Pokazali smo kako združiti asinhrono izvajanje programa v navezi z učinki in prestrezniki. Namesto da čakamo na odgovor v obliki prekinitve, izvajamo nadaljevanje. Šele ko pride ustrezna prekinitev pa nanjo reagiramo. Glavne ideje so formalizirane v \lae-računu. Rešili smo tri primanjkljaje osnovne verzije. Dodali smo rekurzivne prestreznike, da nimamo potrebe po prekomerni uporabi pomožnih rekurzivnih funkcij. Kadar je smiselno dovolimo uporabo tovorov višjega rega. Pri tem bistveno dovolimo uporabo funkcij v tovoru. Dodali smo tudi ustvarjanje novih procesov, s čimer lahko program naredimo še bolj asinhron in vzporeden. Pokazali smo izreka o napredku in ohranitvi za osnovno in razširjeno verzijo \lae-računa.

\subsection{Implementacija}\label{sec:implementacija}

Medtem ko je cilj \lae-računa, da pokaže pravilnost asinhronih konceptov prikazanih v prejšnjih poglavjih, je cilj jezika \aeff{}, da te koncepte lahko tudi preizkusimo in uporabimo. Zato ima jezik \aeff{} določene stvari, ki jih \lae-račun nima.

Jezik \aeff{} je implementiran v jeziku Ocaml. Implementacija je sestavljena iz treh glavnih delov.
Parserja, ki prevede tekst v abstraktno sintaktično drevo (AST) in preveri ustreznost sintakse.  
Pregledovalnika tipov, ki preveri, da ima dani program tip.
In tolmača, ki izvaja korake, dokler ne pride do rezultata.


Da \aeff{} postane bolj uporaben so mu dodane konstantni vrednosti $\true$ in $\false$, funkcije kot so $+, -, *, / ...$, reference in globalne spremenljivke.

Izračun $\uparrow\hspace{-1ex}\op (V);M$ je sintaktičen sladkor za $\tmlet{\_}{\tmopout{op}{V}{return ()}}{M}$.
Podobno je $\tmkw{spawn}(M);N$ sintaktičen sladkor za $\tmlet{\_}{\tmspawn{M}{return ()}}{N}$. 
%Izračun $M;N$ je sintaktičen sladkor za $\tmlet{\_}{M}{N}$, ki ga lahko uporabimo kadar rezultat izračuna $M$ ni uporabljen v $N$.

Varovan prestreznik $$\tmwithrecgu{op}{x}{r}{P(x)}{M}{p}{N},$$ je sintaktičen sladkor za $$\tmwithrec{op}{x}{r}{\tmif{P(x)}{M}{r()}}{p}{N}.$$
Nanj lahko gledamo kot na prestreznik, ki se sproži, le kadar je poleg pravega imena operacije tudi pravi tovor.

Dodamo tudi seznam. Za prazen seznam uporabimo konstruktor []. Elementa mu lahko dodajamo z $::$, recimo 1::2::3::[].
Sedaj imamo problem, ker uporabljamo [] tako za sezname kot za zavite vrednosti.
V poglavju~\ref{sec:razsirjen-lae}, kjer v dveh primerih uporabimo seznam, nadomestimo konstruktor za prazen seznam z $\mathsf{nil}$ in ohranimo [] za zavite vrednosti, ki so bistvene v tem delu.
Nasprotno v dejanski implementaciji \aeff{} raje ohranimo [] za prazen seznam in spremenimo sintakso zavite vrednosti v $[|V|]$. 

Prvotna implementacija jezika \aeff{} izračuna tip programa z Hindley–Milnerjevim algoritmom. 
Kasneje je bil typechecker spremenjen, da je uporabljal dvosmerni sistem tipov (bidirectional typing), ki temelji na algoritmu iz članka~\cite{bidirectional}.
Glavna ideja dvosmernega sistema tipov je, da lahko dinamično izmenjuje med preverjanjem in računanjem tipa. Kadar so tipi pripisani izrazom, tip le preverimo, kar je enostavneje, kot ga izračunati in v primeru napake lahko izpiše boljše poročilo o napaki. Vendar pa pripisovanje tipov ni obvezno in ga v tem primeru lahko izračunamo. Posledično lahko uporabnik, poda tip le nekaterim delom programa, preostalim pa ne.
Tako prva kot druga verzija ne preverjata tipe učinkov, ampak le standardni tip.

Čeprav ima \aeff{} možnost vzporednega izvajanja in smo to do sedaj v primerih programov predpostavljali, sama implementacije zaenkrat le simulira vzporedno izvajanje.

Dodana je kratka standardna knjižnica v kateri lahko najdemo osnovne funkcije kot so $\mathsf{nth}$, $\mathsf{map}$, $\mathsf{filter}$, $\mathsf{min}$, $\mathsf{max}$, $\mathsf{length}$, $\mathsf{fold\_left}$, $\mathsf{fold\_right}$...

\subsection{Bodoče delo}

Še zmeraj so določene stvari nenarejene ali pa bi jih lahko izboljšali.

V bodoče bi se lahko tolmač napisalo v razširitvi Multicore OCaml, ki omogoča vzporedno izvajanje, in posledično bi lahko tudi jezik \aeff{} podpiral vzporedno izvajanje. 

V implementaciji manjka preverjanje tipov učinkov. Posledično nimamo koristi, ki jih tipi učinkov prinesejo. Ena izmed koristi je sledeča optimizacija. Če imamo $\tmopin{op}{V}{M}$, kjer $\Gamma \types M \of \tycomp{A}{\o, \i}$ in $\i(\op) = \bot$, potem lahko naredimo korak direktno v $M$, saj iz $\i(\op) = \bot$ vemo, da $M$ ne vsebuje prestreznika za to prekinitev. 

Naslednja optimizacija bi bila da, namesto da se signali počasi širijo navzven do nivoja procesov korak za korakom, bi se lahko razširili v enem samem velikem koraku.

Sedaj ko imamo dinamično ustvarjanje procesov, lahko to potencialno ustvari zelo veliko novih procesov. Tej novo ustvarjeni procesi se običajno izvedejo relativno hitro v primerjavi z ročno ustvarjenimi. Prav tako se običajno bistvene informacije prenesejo v druge procese z učinki in nas njihov rezultat ne zanima. Posledično imamo potencialno veliko procesov, ki bodo prejemali prekinitve, kar lahko drastično upočasni izvajanje. Potencialna rešitev bi bila možnost odstranitve posameznega procesa.

\section*{Slovar strokovnih izrazov}



\geslo{await}{blokada}
\geslo{bidirectional type system}{dvosmerni sistem tipov}
\geslo{boxed type}{zavit tip}
\geslo{ground type}{osnovni tip}
\geslo{handler}{prestreznik}
\geslo{interrupt}{prekinitev}
\geslo{mobile type}{prenosljiv tip}
\geslo{promise}{obljuba}
\geslo{signal}{signal}
\geslo{spawn}{dinamični proces}
\geslo{unbox}{odvijanje}


%guarded promise = varovana obljuba
%abstraction = abstrakcija
%spawn =  ustvaritev novega procesa
%interpreter = tolmač
%type sistem = sistem tipov
%computation = izračun
%calculus = račun
%unit = enota
%Syntactic sugar = sintaktičen sladkor
%black box in types = kocka 
%recursion = rekurzija
%bidirectional = dvosmerni?
%small step semantics = op. sem. malih korakov
%pattern matching = ujemanje vzorca
%substitution = substitucija
%substitute = substituirati
%source code = izvorna koda
%machine code = strojna koda
%shared memory = skupni spomin
%context = kontekst
%true/false = resnica/neresnica
%runtime error = napaka ob izvajanju
%typing rules = pravila za tipe
%interupt propagation = širitev prekinitve
%signal hoisting = dvig signala
%broadcasting = oddajanje signala
%evaluation context rule = Vrednotenje v kontekstu
%turing complete = turingovo poln