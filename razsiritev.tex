\section{Razširjen \lae{}}\label{sec:razsirjen-lae}

Poglejmo si primer~\ref{prog:razsiritev-0}, ki bi ga lahko napisali v \aeff{}, ki pripada osnovni verziji \lae{}\nobreakdash-računa. Primer je nekoliko bolj konkretna verzija primera~\ref{prog:primer-1} iz poglavja~\ref{sec:primeri-aeff}. Dve pomožni vrednosti se poračunata v drugem procesu. Medtem se glavni proces nemoteno nadaljuje. 
Primer si lahko predstavljamo kot program, ki v pomožnem procesu poračuna časovno zahteven del, med tem ko glavni proces nemoteno nadaljuje delo.

\begin{lstlisting}[caption={Računanje zahtevne funkcije v ozadju.},label={prog:razsiritev-0},float,floatplacement=h]
operation call : int * int * int
operation result : int * int

let callWith =
	fun callCounter fNo x ->
		let callNo = !callCounter in
		send call (callNo, fNo, x);
		callCounter := callNo + 1;
		promise (result (callNo', y) when callNo = callNo' ->
			return <y>
		) as p in
		let valueThunk () = awaitValue p in
		return valueThunk

let remote fList =
	let rec loop () =
		promise (call (callNo, fNo, x) ->
			let f = nth fList fNo in
			let y = f x in
			send result (callNo, y);
			loop ()
		) as p in
		return p
	in
	loop ()

run	let callCounter = ref 0 in
	let y1 = callWith callCounter 0 2 in
	let y2 = callWith callCounter 1 3 in
	let z = 42 + 7 in
	return y1 () + y2 () + z

run remote (fun x |-> 2 * x)::(fun x |-> 7 * x)::[]
\end{lstlisting}

Ta primer, ima še zmeraj nekaj problemov, ki jih bomo v nadaljevanju odpravili.


\subsection{Rekurzivni prestreznik}


Kadar pričakujemo več prekinitev z isto operacijo, je lahko iz različnih razlogov za programerja priročno, če ima na voljo prestreznik, ki se po potrebi ponovno namesti. Lahko, da je namen prestreznika, da na vsako prekinitev odgovori z signalom. Lahko nam prihajajoča prekinitev prinese pravo operacijo, vendar ne pravega pripadajočega tovora, zato obljube še nočemo izpolniti, in čakamo na naslednjo prekinitev.

To funkcionalnost smo do sedaj dosegli tako, da smo obljubo definirali v pomožni rekurzivni funkciji in po potrebi znotraj obljube ponovno klicali to funkcijo, kot lahko vidimo v programu~\ref{prog:razsiritev-0}.


Predvsem z namenom bolj pregledne kode nadomestimo obstoječi prestreznik z rekurzivnim prestreznikom. Le ta ima poleg imena operacije $op$, tovora $x$, obljube $M$ in kontinuacije $N$ tudi spremenljivko $r$ s pomočjo katere lahko ponovno namestimo isti prestreznik. Pripadajočo sintakso vidimo na sliki~\ref{fig:izrazi-prestreznik}.


\begin{figure}[H]
	\centering
	\small
	\begin{align*}
		M, N
		\bnfis& ...                            & &\text{obstoječi izračuni} \\
		\bnfor& \tmwithrec{op}{x}{r}{M}{p}{N}  & &\text{rekurzivni prestreznik}
	\end{align*}
	
	\caption{Izračuni z rekurzivnim prestreznikom}
	\label{fig:izrazi-prestreznik}
\end{figure}

Operacijska semantika rekurzivnega prestreznika je zelo podobna prejšnji verziji. S to izjemo da, ko ustrezna prekinitev pride do prestreznika, le ta v obljubi $M$ naredi dve substituciji. Tako kot prej spremenljivko $x$ substituira z vrednostjo $V$, dodatno pa spremenljivko $r$ substituira z lambda funkcijo $R$, ki sprejme enoto in vrne svežo kopijo prestreznika. Znotraj $M$ lahko uporabimo funkcijo $R$ in tako ponovno namestimo prestreznik.   

\begin{figure}[H]
	\centering
	\small
	\begin{align*}
		(\tmwithrec{op}{y}{r}{M}{p}{N})[V/x] &= \tmwith{op}{y}{r}{M[V/x]}{p}{N[V/x]}
	\end{align*}
	\caption{Substitucija za rekurzivni prestreznik.}
	\label{fig:substitucija-prestreznik}
\end{figure}

\begin{figure}[H]
	\centering
	\small
	\begin{align*}
		\tmlet{x}{(\tmwithrec{op}{y}{r}{M}{p}{N_1})}{N_2} &\reduces \tmwithrec{op}{y}{r}{M}{p}{(\tmlet{x}{N_1}{N_2})}
		\\
		\tmwithrec{op}{x}{r}{M}{p}{\tmopout{op'}{V}{N}} &\reduces \tmopout{op'}{V}{\tmwithrec{op}{x}{r}{M}{p}{N}}
		\\
		\tmopin{op}{V}{\tmwithrec{op}{x}{r}{M}{p}{N}} &\reduces \tmlet{p}{M[V/x, R/r]}{\tmopin{op}{V}{N}} \\
		R = \;& \tmfunano{y}{(\tmwithrec{op}{r}{x}{M}{p}{\tmreturn{p}})} \\
		\tmopin{op'}{V}{\tmwithrec{op}{x}{r}{M}{p}{N}} &\reduces \tmwithrec{op}{x}{r}{M}{p}{\tmopin{op'}{V}{N}} \\
		&\qquad {\color{rulenameColor}(\op \neq \op')}
	\end{align*}
	
	\caption{Operacijska semantika rekurzivnega prestreznika.}
	\label{fig:semantika-prestreznik}
\end{figure}

Tudi pravilo za tip rekurzivnega prestreznika je zelo podobno prvotnemu prestrezniku. 
Za razliko od prej imamo sedaj v $({\o'} , {\i'}) \order{O \times I} \i\, (\op)$ red namesto enakosti. Ker sedaj v kontekst dodamo tip $r$ moramo paziti da pripadajoč tip efektov ni prevelik, saj sicer morda ne bomo mogli določiti tipa za $M$.

\jR{Omeni zakaj se je enakost spremenila v order}
\begin{figure}[H]
	\centering
	\small
	\begin{mathpar}
		\coopinfer{TyComp-RecPromise}{
			({\o'} , {\i'}) \order{O \times I} \i\, (\op)\\
			\Gamma, x \of \tymobile{A}_\op, r \of \tyfun{\tyunit}{\tycomp{\typromise{B}}{\o', \i'}} \types M : \tycomp{\typromise B}{\o',\i'} \\
			\Gamma, p \of \typromise B \types N : \tycomp{C}{\o,\i}
		}{
			\Gamma \types \tmwithrec{op}{x}{r}{M}{p}{N} \of \tycomp{C}{\o, \i}
		}
	\end{mathpar}
	
	\caption{Pravilo za izračun tipa rekurzivnega prestreznika.}
	\label{fig:tipi-pravila-rekurzivni-prestreznik}
\end{figure} 

Program~\ref{prog:razsiritev-1} ima isti semantičen pomen kot program~\ref{prog:razsiritev-0}, le da je tokrat napisan z rekurzivno obljubo.

\begin{lstlisting}[caption={Primer z rekurzivnim prestreznikom},label={prog:razsiritev-1},float,floatplacement=h]
operation call : int * int * int
operation result : int * int

let callWith =
	fun callCounter fBoxed x ->
		let callNo = !callCounter in
		send call (callNo, fBoxed, x);
		callCounter := callNo + 1;
		promise (result (callNo', y) when callNo = callNo' ->
			return <y>
		) as p in
		let valueThunk () = awaitValue p in
		return valueThunk

let remote fList =
	promise (call (callNo, fNo, x) r ->
		let f = nth fList fNo in
		let y = f x in
		send result (callNo, y);
		r()
	) as _ in
	()

run	let callCounter = ref 0 in
	let y1 = callWith callCounter 0 2 in
	let y2 = callWith callCounter 1 3 in
	let z = 42 + 7 in
	return y1 () + y2 () + z

run remote (fun x |-> 2 * x)::(fun x |-> 7 * x)::[]
\end{lstlisting}


\subsection{Prenosljivi tipi}

Ko imamo nek signal, je ključno, da tovor ne vsebuje vrednosti obljub, kar smo do sedaj rešili tako, da smo omejili signale na osnovne tipe. Glavni problem tega pristopa je, da funkcijski tip ni med osnovnimi tipi in posledično ne moremo poslati funkcij. Ta problem rešimo z uvedbo \emph{zavitih vrednosti} in \emph{zavitih tipop}. Zavite vrednosti bodo lahko tudi lambda funkcije, ki ne bodo vsebovale zunanjih obljub. Posledično jih bo varno poslati. Osnovne tipe skupaj z zavitim tipom imenujemo \emph{prenosljivi tipi}. Da bomo lahko zavito vrednost tudi uporabili, dodamo izračun \emph{odvijanje}, ki iz zavite vrednosti odvije pripadajočo vrednost in jo veže na spremenljivko.

Tipi za izračune in procese ostanejo enaki.

\begin{figure}[H]
	\centering
	\small
	\begin{align*}
	\intertext{\textbf{Vrednosti}}
	V
	\bnfis& ...                            & &\text{obstoječe vrednosti} \\
	\bnfor& \tmboxed{V}  & &\text{zavita vrednost}
	\intertext{\textbf{Izračuni}}
	M, N
	\bnfis& ...                            & &\text{obstoječi izračuni} \\
	\bnfor& \tmunbox{V}{x}{M}  & &\text{odvijanje}
	\end{align*}
	
	\caption{Prenosljivi izrazi.}
	\label{fig:izrazi-prenosljivi}
\end{figure}

\begin{figure}[H]
	\centering
	\small
	\begin{align*}
	\text{Prenosljivi tipi vrednosti $\tymobile{A}$, $\tymobile{B}$}
	\bnfis & \tysym{int} \,\bnfor\! \tysym{bool} \,\bnfor\! \tyunit \,\bnfor\! \tyempty \,\bnfor\! \typrod{\tymobile{A}}{\tymobile{B}} \,\bnfor\! \tysum{\tymobile{A}}{\tymobile{B}} \,\bnfor\! \tyboxed{A}
	\\%[1ex]
	\text{Tipi vrednosti $A$, $B$}
	\bnfis & \tymobile{A} \, \bnfor\! \typrod{A}{B} \,\bnfor\! \tysum{A}{B} \,\bnfor\! \tyfun{A}{\tycomp{B}{\o,\i}} \,\bnfor\! \typromise{A}
	\end{align*}
%	\vspace{-5ex}
%	\begin{align*}
%	(op_1, \tymobile{A}_{op_1}),\, (op_2, \tymobile{A}_{op_2}),\, ... ,\, (op_n, \tymobile{A}_{op_k})
%	\end{align*}
	
	\caption{Prenosljivi tipi.}
	\label{fig:tipi-prenosljivi}
\end{figure}

Sedaj vsaki operaciji namesto osnovnega tipa pripada prenosljiv tip.

\begin{figure}[H]
	\centering
	\small
	\begin{align*}
	(op_1, \tymobile{A}_{op_1}),\, (op_2, \tymobile{A}_{op_2}),\, ... ,\, (op_n, \tymobile{A}_{op_k})
	\end{align*}
	\vspace{-5ex}
	\caption{Operacije z pripadajočimi prenosljivimi tipi.}
	\label{fig:prenosljive-operacije}
\end{figure}

Izračun odvijanje veže vrednost znotraj zavite vrednosti $\tmboxed{V}$ na spremenljivko $x$, ki jo lahko uporabimo v kontinuaciji $M$.

\begin{figure}[H]
	\centering
	\small
	\jR{[W][V/x] is not loking good}
	\begin{align*}
		\tmboxed{W}[V/x] &= \tmboxed{W[V/x]} \\
		(\tmunbox{W}{x}{M})[V/x] &= \tmunbox{W[V/x]}{x}{M[V/x]}
	\end{align*}
	\caption{Substitucija za rekurzivni prestreznik.}
	\label{fig:substitucija-box}
\end{figure}

\begin{figure}[h]
	\centering
	\small
	\begin{align*}
	\tmunbox{[V]}{x}{M} &\reduces M[V/x]
	\end{align*}
	
	\caption{Operacijska semantika odvijanja.}
	\label{fig:semantika-odvijanje}
\end{figure}

Da zagotovimo, da ne bi kakšni vrednosti, ki vsebuje zunanjo obljubo, dodelili zavit tip, nekoliko spremenimo pravilo za izračun tipa spremenljivke in dodamo pravili za zavito vrednost in izračun odvijanje.
Okolje je imelo že od začetka določen vrstni red, vendar šele sedaj to postane zares pomembno. Ko določamo tip zavite vrednosti, v okolje dodamo $\bb$. S tem za v bodoče označimo katere spremenljivke so zunanje ali notranje glede na zavito vrednost. 
Če je spremenljivka notranja, torej od kar smo jo dodali v okolje, nismo dodali nobenega $\bb$, je pravilo standardno. Če je spremenljivka zunanja, torej od kar je bila dodana v okolje, je bil v okolje dodan tudi $\bb$, mora biti pripadajoči tip v okolju prenosljiv.

\begin{figure}[H]
	\centering
	\small
	\begin{mathpar}
		\coopinfer{TyVal-Var'}{
			\text{A prenosljiv ali } \bb \notin \Gamma'
		}{
			\Gamma, x \of A, \Gamma' \types x \of A
		}
		\quad
		\coopinfer{TyVal-Boxed}{
			\Gamma, \bb \types V \of A
		}{
			\Gamma \types \tmboxed{V} \of \tyboxed{A}
		}
		\quad
		\coopinfer{TyComp-Unbox}{
			\Gamma \types V \of \tyboxed{A} \\ \Gamma, x \of A \types M \of \tycomp{B}{\o, \i}
		}{
			\Gamma \types \tmunbox{V}{x}{M} \of \tycomp{B}{\o, \i}
		}
	\end{mathpar}
	
	\caption{Dodatna pravila za določanje tipov.}
	\label{fig:tipi-pravila-prenosljivi}
\end{figure} 

Sedaj lahko program~\ref{prog:razsiritev-1} spremenimo v program~\ref{prog:razsiritev-2}. Namesto, da ima \emph{remote} statičen seznam funkcij, ki jih lahko poračuna, preprosto dobi funkcijo, kot del tovora.  

\begin{lstlisting}[caption={Računanje zahtevne funkcije v ozadju.},label={prog:razsiritev-2},float,floatplacement=h]
operation call : int * [int -> int] * int
operation result : int * int

let callWith =
	fun callCounter fBoxed x ->
		let callNo = !callCounter in
		send call (callNo, fBoxed, x);
		callCounter := callNo + 1;
		promise (result (callNo', y) when callNo = callNo' ->
			return <y>
		) as p in
		let valueThunk () = awaitValue p in
		return valueThunk

let remote () =
	promise (call (callNo, fBoxed, x) r ->
		unbox fBoxed as [f] in
		let y = f x in
		send result (callNo, y);
		r ()
	) as _ in
	()

run	let callCounter = ref 0 in
let y1 = callWith callCounter [fun x |-> 2 * x] 2 in
let y2 = callWith callCounter [fun x |-> 7 * x] 3 in
let z = 42 + 7 in
return y1 () + y2 () + z

run remote ()
	
\end{lstlisting}

\subsection{Dinamični procesi}

Funkcija $f$ v oddaljenem klicu se potencialno izvaja zelo počasi. Zato si želimo, da bi vsak oddaljen klic lahko opravili v svojem procesu, da lahko izvajamo vzporedno namesto zaporedno. 
Do sedaj so bili procesi statični v smislu, da smo vse procese morali ustvariti na začetku.
Posledično bi morali vnaprej vedeti, koliko klicev bo oddaljeni klic hkrati dobil, da bi imeli temu primerno pripravljenih procesov.

Da rešimo ta problem, dodamo dinamično ustvarjanje novih procesov s pomočjo izračuna $\tmspawn{M}{N}$. 

\begin{figure}[H]	
	\centering
	\small
	\begin{align*}
	\shortintertext{\textbf{Izračuni}}
	M, N
	\bnfis & ... \,\bnfor\! \tmspawn{M}{N}
	\end{align*}
	
	\caption{Izračun $\tmkw{Spawn}$.}
	\label{fig:izračun-spawn}
\end{figure}


\begin{figure}[H]
	\centering
	\small
	\begin{align*}
		(\tmspawn{M}{N})[V/x] &= \tmspawn{M[V/x]}{N[V/x]}
	\end{align*}
	\caption{Substitucija za dinamične procese.}
	\label{fig:substitucija-spawn}
\end{figure}

Izračun $\tmkw{spawn}$ se bo pomikal navzven. Ko bo postal zgornji izračun v procesu, se bo ta proces razcepil v dva procesa. 
Prvi proces bo izvajal izračun $M$, drugi pa bo izvajal nadaljevanje $N$.

\begin{figure}[H]
	\centering
	\small
	\begin{align*}
	\tmlet{x}{(\tmspawn{M_1}{M_2})}{N} & \reduces \tmspawn{M_1}{\tmlet{x}{M_2}{N}}
	\\
	\tmwithrec{op}{x}{r}{M}{p}{\tmspawn{N_1}{N_2}} & \reduces \tmspawn{N_1}{\tmwithrec{op}{x}{r}{M}{p}{N_2}}
	\\
	\tmopin{op}{V}{\tmspawn{M}{N}} & \reduces \tmspawn{M}{\tmopin{op}{V}{N}}
	\\
	\tmrun{(\tmspawn{M}{N})} & \reduces \tmpar{\tmrun{M}}{\tmrun{N}}
	\\
	\tmrun{(\tmspawn{M}{N})} & \reduces \tmpar{\tmrun{N}}{\tmrun{M}}
	\end{align*}
	
	\caption{Operacijska semantika dinamičnih procesov.}
	\label{fig:semantika-spawn}
\end{figure}

Ker se izračun $M$ v $\tmspawn{M}{N}$ lahko razširi mimo prestreznikov moramo paziti, da ima $M$ prenosljiv tip in je posledično varen.   

\begin{figure}[H]
	\centering
	\small
	\begin{mathpar}
		\coopinfer{TyComp-Spawn}{
			\Gamma, \blacksquare \types M : \tycomp{B}{\o', \i'} \\ \Gamma \types N : \tycomp{A}{\o, \i}
		}{
			\Gamma \types \tmspawn{M}{N} : \tycomp{A}{\o, \i}
		}
	\end{mathpar}
	\vspace{-5ex}
	\caption{Dodatno pravilo za določitev tipa.}
	\label{fig:tipi-pravila-spawn}
\end{figure}


Sedaj ko imamo dinamične procese, lahko obe evalvaciji funkcije $f$ opravimo hkrati v svojem procesu, kar potencialno pohitri izvajanje programa. Primer~\ref{prog:razsiritev-2} lahko sedaj spremenimo v~\ref{prog:razsiritev-3}. 

\begin{lstlisting}[caption={Primer z dinamičnimi procesi},label={prog:razsiritev-3},float,floatplacement=H]
operation call : int * [int -> int] * int
operation result : int * int

let callWith =
	fun callCounter fBoxed x ->
		let callNo = !callCounter in
		send call (callNo, fBoxed, x);
		callCounter := callNo + 1;
		promise (result (callNo', y) when callNo = callNo' ->
			return <y>
		) as p in
		let valueThunk () = awaitValue p in
		return valueThunk

let remote () =
	promise (call (callNo, fBoxed, x) r ->
		unbox fBoxed as [f] in
		spawn(
			let y = f x in
			send result (callNo, y)
		);
		r ()
	) as _ in
	()

run	let callCounter = ref 0 in
let y1 = callWith callCounter [fun x |-> 2 * x] 2 in
let y2 = callWith callCounter [fun x |-> 7 * x] 3 in
let z = 42 + 7 in
return y1 () + y2 () + z

run remote ()
\end{lstlisting}


\subsection{Varnost}

Sedaj dokažemo izrek o varnosti za \lae{} razširjen z rekurzivnimi prestrezniki, prenosljivimi tipi, in dinamičnimi procesi.

\begin{figure}[H]
	\centering
	\begin{mathpar}
		\coopinfer{R-Spawn}{
			\CompResult {\Psi} {N}
		}{
			\CompResult {\Psi} {\tmspawn{M}{N}}
		}
	\end{mathpar}
	

	\caption{Dodaten rezultat za izračune.}
	\label{fig:results-rules-2}
\end{figure}



\begin{trditev}[o napredku]\label{trd:gamma-napredek-2}
	Naj za izračun $M$ velja $\Gamma \types M \of \tycomp{A}{\o, \i}$, kjer je $\Gamma = x_1 \of \typromise{A_1}, x_2 \of \typromise{A_2},..., x_i \of \typromise{A_i}$. Potem ali (i) obstaja izračun $M'$, tak da $M \reduces M'$, ali pa (ii) velja $\CompResult{\Gamma}{M}$.
\end{trditev}

\begin{proof}
	Dokazujemo z strukturno indukcijo na drevo izpeljave za $\Gamma \types M \of \tycomp{A}{\o, \i}$.
	Ločimo primere glede na zadnje uporabljeno pravilo.
	Primeri, ki so enaki kot v dokazu~\ref{trd:gamma-napredek} so izpuščeni. V primerih, ki so drugačni, omenimo le spremenjen ali dodan del.
	
	\begin{itemize}
		\item Če je zadnje pravilo \rulename{TyComp-Let}, potem je $M$ enak $\tmlet{x}{N_1}{N_2}$.
		Če je $N_1$ rezultat, dobimo dodatno možnost.
		\begin{itemize}
			\item Če je zadnje pravilo \rulename{R-Spawn}, potem je $N_1$ enak $\tmspawn{N_3}{N_4}$ in lahko $M$ naredi korak v $\tmspawn{N_3}{(\tmlet{x}{N_4}{N_2})}$.
		\end{itemize}
		
		\item Če je zadnje pravilo \rulename{TyComp-Interrupt}, potem je $M$ enak $\tmopin{op}{V}{N}$.
		Če je $N$ rezultat, dobimo dodatno možnost.
		\begin{itemize}
			\item Če je zadnje pravilo \rulename{R-Spawn}, potem je $N$ enak $\tmspawn{N_1}{N_2}$ in lahko $M$ naredi korak v $\tmspawn{N_1}{(\tmopin{op}{V}{N_2})}$.
		\end{itemize}
		
		
		\item Če je zadnje pravilo \rulename{TyComp-RecPromise}, potem je $M$ enak $\tmwith{op}{x}{N_1}{p}{N_2}$.
		Če je $N_2$ rezultat, dobimo dodatno možnost.
		\begin{itemize}
			\item Če je zadnje pravilo \rulename{R-Spawn}, potem je $N_2$ oblike $\tmspawn{N_3}{N_4}$ in lahko $M$ naredi korak v $\tmspawn{N_3}{\tmwith{op}{x}{N_1}{p}{N_4}}$.
		\end{itemize}		
	\end{itemize}	
\end{proof}


\begin{posledica}[o napredku]\label{pos:prazen-napredek-2}
	Naj za izračun $M$ velja $\emptyset \types M \of \tycomp{A}{\o, \i}$. Potem ali (i) obstaja izračun $M'$, tak da $M \reduces M'$, ali pa (ii) velja $\RunResult{\emptyset}{M}$.
\end{posledica}

\begin{proof}
	Trditev \ref{trd:gamma-napredek-2} velja za vsak kontekst $\Gamma$, torej tudi za $\Gamma = \emptyset$.
\end{proof}


\begin{izrek}[o napredku]
	Naj za proces $P$ velja $\emptyset \types P \of C$. Potem ali (i) obstaja proces $P'$, tak da $P \reduces P'$, ali pa (ii) velja $\ProcResult{P}$.
\end{izrek}

\begin{proof}
	Dokazujemo z strukturno indukcijo na drevo izpeljave za $\emptyset \types P \of C$.
	Ločimo primere glede na zadnje uporabljeno pravilo.
	Primeri, ki so enaki kot v dokazu~\ref{izr:napredek} so izpuščeni. V primerih, ki so drugačni, omenimo le spremenjen ali dodan del.
	
	\begin{itemize}
		\item Če je zadnje pravilo \rulename{TyProc-Run}, potem je $P$ enak $\tmrun{M}$. Ker ima $M$ tip, po posledici~\ref{pos:prazen-napredek-2} velja da, ali lahko $M$ naredi korak v $M'$ ali pa je $M$ delni rezultat.
		Če je $M$ rezultat, dobimo dodatno možnost.
		\begin{itemize}
			\item Če je zadnje pravilo \rulename{R-Spawn}, potem je $M$ enak $\tmspawn{N_1}{N_2}$ in lahko $P$ naredi korak v $\tmpar{N_1}{N_2}$ ali v $\tmpar{N_2}{N_1}$.
		\end{itemize}		
	\end{itemize}
\end{proof}

Lemi~\ref{lem:weakening-values-2} in~\ref{lem:weakening-comp-2} se navezujeta ena na drugo.

\begin{lema}\label{lem:weakening-values-2}
	Naj za vrednost $V$ velja $\Gamma_1, \Gamma_2 \types V \of A$. Za vsak kontekst $\Gamma_3$, ki vsebuje le sveže spremenljivke glede na $\Gamma_1$ in $\Gamma_2$, velja $\Gamma_1, \Gamma_3, \Gamma_2 \types V \of A$
\end{lema}

\begin{proof}
	Dokazujemo z strukturno indukcijo na drevo izpeljave za $\Gamma_1, \Gamma_2 \types V \of A$.
	Ločimo primere glede na zadnje uporabljeno pravilo.
	Primeri, ki so enaki kot v dokazu~\ref{lem:weakening-values} so izpuščeni.
	
	\begin{itemize}
		\item Če je zadnje pravilo \rulename{TyVal-Var'}, potem je $V = x$ in velja $x \in \Gamma_1$ ali $x \in \Gamma_2$.
		Ker $\Gamma_3$ ne vsebuje $\bb$, velja $\Gamma_1, \Gamma_3, \Gamma_2 \types V \of A$ po pravilu \rulename{TyVal-Var'}
		
		\item Če je zadnje pravilo \rulename{TyVal-Boxed}, je $V = \tmboxed{V'}$. Po indukcijski predpostavki ima $V'$ isti tip v razširjenem kontekstu.
		Po pravilu \rulename{TyVal-Boxed} velja $\Gamma_1, \Gamma_3, \Gamma_2 \types V \of A$.
		
	\end{itemize}
\end{proof}

\begin{lema}\label{lem:weakening-comp-2}
	Naj za izračun $M$ velja $\Gamma_1, \Gamma_2 \types M \of \tycomp{A}{\o, \i}$. Za vsak kontekst $\Gamma_3$, ki vsebuje le sveže spremenljivke glede na $\Gamma_1$ in $\Gamma_2$, velja $\Gamma_1, \Gamma_3, \Gamma_2 \types M \of \tycomp{A}{\o, \i}$
\end{lema}

\begin{proof}
	Dokazujemo z strukturno indukcijo na drevo izpeljave za $\Gamma_1, \Gamma_2 \types M \of \tycomp{A}{\o, \i}$.
	Ločimo primere glede na zadnje uporabljeno pravilo.
	Primeri, ki so enaki kot v dokazu~\ref{lem:weakening-comp} so izpuščeni.
	
	\begin{itemize}
		\item Če je zadnje pravilo \rulename{TyComp-RecPromise}, potem je $M$ enak $\tmwithrec{op}{x}{r}{N_1}{p}{N_2}$.
		Po indukcijski predpostavki imata $N_1$ in $N_2$ isti tip v razširjenem kontekstu.
		Po pravilu \rulename{TyComp-RecPromise} velja $\Gamma_1, \Gamma_3, \Gamma_2 \types M \of \tycomp{A}{\o, \i}$.
		
		\item Če je zadnje pravilo \rulename{TyComp-Unbox}, potem je $M$ enak $\tmunbox{V}{x}{N}$.
		Po lemi~\ref{lem:weakening-values} ima $V$ isti tip v razširjenem kontekstu.
		Po indukcijski predpostavki ima $N$ isti tip v razširjenem kontekstu.
		Po pravilu \rulename{TyComp-Unbox} velja $\Gamma_1, \Gamma_3, \Gamma_2 \types M \of \tycomp{A}{\o, \i}$.
		
		\item Če je zadnje pravilo \rulename{TyComp-Spawn}, potem je $M$ enak $\tmspawn{N_1}{N_2}$.
		Po indukcijski predpostavki imata $N_1$ in $N_2$ isti tip v razširjenem kontekstu.
		Po pravilu \rulename{TyComp-Spawn} velja $\Gamma_1, \Gamma_3, \Gamma_2 \types M \of \tycomp{A}{\o, \i}$.
		
	\end{itemize}potem je bilo zadnje pravilo za določitev tipa $P$ \rulename{TyProc-Run}. Tip $C$ je oblike $\tyrun{A}{\o, \i}$.
Izračun $M$ ima tip $\tycomp{B}{\o', \i'}$ in $N$ tip $\tycomp{A}{\o, \i}$.
Po pravilu \rulename{TyProc-Par} ima $P'$ tip $\typar{\tycomp{A}{\o, \i}}{\tycomp{B}{\o', \i'}}$, kjer je $C$ naredil korak \rulename{TyRedu-Spawn-R}.
\end{proof}

Sledeči lemi o šibitvi konteksta za vrednosti in izračune se nanašata ena na drugo.

\begin{lema}\label{lem:weakening-values-bb}
	Naj za vrednost $V$ velja $\Gamma_1, \Gamma_2 \types V \of A$, kjer $\Gamma_2$ vsebuje $\bb$.
	Za vsak kontekst $\Gamma_3$, ki vsebuje le $\bb$ in sveže spremenljivke glede na $\Gamma_1$ in $\Gamma_2$, velja $\Gamma_1, \Gamma_3, \Gamma_2, \types V \of A$.
\end{lema}

\begin{proof}
	Dokazujemo z strukturno indukcijo na drevo izpeljave za $\Gamma_1, \Gamma_2 \types V \of A$.
	Ločimo primere glede na zadnje uporabljeno pravilo.
	
	\begin{itemize}
		\item[\sitem] Če je zadnje pravilo \rulename{TyVal-Var'}, potem je $V = x$ in velja $x \in \Gamma_1$ ali $x \in \Gamma_2$.
		\begin{itemize}
			\item Če je $x \in \Gamma_1$ potem je $A$ prenosljiv tip. Če je $x \in \Gamma_2$ in je za njim $\bb$ je $A$ spet prenosljiv tip.
			Ker je $A$ prenosljiv tip velja $\Gamma_1, \Gamma_3, \Gamma_2 \types V \of A$ po pravilu \rulename{TyVal-Var'}.
			\item Če je $x \in \Gamma_2$ in za njim ni $\bb$ potem velja $\Gamma_1, \Gamma_3, \Gamma_2 \types V \of A$ po pravilu \rulename{TyVal-Var'}.
		\end{itemize}
		
		\item Če je zadnje pravilo \rulename{TyVal-Cons}, potem je $V = n$. Po pravilu \rulename{TyVal-Cons} velja $\Gamma_1, \Gamma_3, \Gamma_2, \types V \of A$.
%		\item Če je zadnje pravilo \rulename{Ty-Cons-T}, potem je $V = true$. Po pravilu \rulename{Ty-Cons-T} velja $\Gamma_1, \Gamma_3, \Gamma_2, \bb \types V \of A$.
%		\item Če je zadnje pravilo \rulename{Ty-Cons-F}, potem je $V = false$. Po pravilu \rulename{Ty-Cons-F} velja $\Gamma_1, \Gamma_3, \Gamma_2, \bb \types V \of A$.
		
		\item Če je zadnje pravilo \rulename{TyVal-Unit}, potem je $V = ()$. Po pravilu \rulename{TyVal-Unit} velja $\Gamma_1, \Gamma_3, \Gamma_2, \types V \of A$.
		
		\item Če je zadnje pravilo \rulename{TyVal-Pair}, potem je $V =\tmpair{V_1}{V_2}$.
		Po indukcijski predpostavki imata $V_1$ in $V_2$ isti tip v razširjenem kontekstu.
		Po pravilu \rulename{TyVal-Pair} velja $\Gamma_1, \Gamma_3, \Gamma_2 \types V \of A$.
		
		\item Če je zadnje pravilo \rulename{TyVal-Inl}, potem je $V = \tminl{V_1}$.
		Po indukcijski predpostavki ima $V_1$ isti tip v razširjenem kontekstu.
		Po pravilu \rulename{TyVal-Inl} velja $\Gamma_1, \Gamma_3, \Gamma_2 \types V \of A$.
		
		\item Če je zadnje pravilo \rulename{TyVal-Inr}, potem je $V = \tminl{V_2}$.
		Po indukcijski predpostavki ima $V_2$ isti tip v razširjenem kontekstu.
		Po pravilu \rulename{TyVal-Inr} velja $\Gamma_1, \Gamma_3, \Gamma_2 \types V \of A$.
		
		\item Če je zadnje pravilo \rulename{TyVal-Fun}, potem je $V = \tmfun{x}{M}$.
		Po lemi~\ref{lem:weakening-comp-bb} ima $M$ isti tip v razširjenem kontekstu.
		Po pravilu \rulename{TyVal-Fun} velja $\Gamma_1, \Gamma_3, \Gamma_2 \types V \of A$.
		
		\item Če je zadnje pravilo \rulename{TyVal-Fun-Rec}, potem je $V = \tmfun{x}{M}$.
		Po lemi~\ref{lem:weakening-comp-bb} ima $M$ isti tip v razširjenem kontekstu.
		Po pravilu \rulename{TyVal-Fun-Rec} velja $\Gamma_1, \Gamma_3, \Gamma_2 \types V \of A$.
		
		\item Če je zadnje pravilo \rulename{TyVal-Promise}, potem je $V =\tmpromise{V'}$.
		Po indukcijski predpostavki ima $V'$ isti tip v razširjenem kontekstu.
		Po pravilu \rulename{TyVal-Promise} velja $\Gamma_1, \Gamma_3, \Gamma_2 \types V \of A$.
		
		\item Če je zadnje pravilo \rulename{TyVal-Boxed}, potem je $V = \tmboxed{V'}$.
		Po indukcijski predpostavki ima $V'$ isti tip v razširjenem kontekstu.
		Po pravilu \rulename{TyVal-Boxed} velja $\Gamma_1, \Gamma_3, \Gamma_2 \types V \of A$.
	\end{itemize}
\end{proof}

\begin{lema}\label{lem:weakening-comp-bb}
	Naj za izračun $M$ velja $\Gamma_1, \Gamma_2 \types M \of \tycomp{A}{\o, \i}$, kjer $\Gamma_2$ vsebuje $\bb$.
	Za vsak kontekst $\Gamma_3$, ki vsebuje le $\bb$ in sveže spremenljivke glede na $\Gamma_1$ in $\Gamma_2$, velja $\Gamma_1, \Gamma_3, \Gamma_2 \types M \of \tycomp{A}{\o, \i}$.
\end{lema}

\begin{proof}
	Dokazujemo z strukturno indukcijo na drevo izpeljave za $\Gamma_1, \Gamma_2 \types M \of \tycomp{A}{\o, \i}$.
	Ločimo primere glede na zadnje uporabljeno pravilo.
	
	\begin{itemize}
		\item Če je zadnje pravilo \rulename{TyComp-Return}, potem je $M$ enak $\tmreturn{V}$.
		Po lemi~\ref{lem:weakening-values-bb} ima $V$ isti tip v razširjenem kontekstu.
		Po pravilu \rulename{TyComp-Return} velja $\Gamma_1, \Gamma_3, \Gamma_2 \types M \of \tycomp{A}{\o, \i}$.
		
		\item Če je zadnje pravilo \rulename{TyComp-Let}, potem je $M$ enak $\tmlet{x}{N_1}{N_2}$.
		Po indukcijski predpostavki imata $N_1$ in $N_2$ isti tip v razširjenem kontekstu.
		Po pravilu \rulename{TyComp-Let} velja $\Gamma_1, \Gamma_3, \Gamma_2 \types M \of \tycomp{A}{\o, \i}$.
		
		\item Če je zadnje pravilo \rulename{TyComp-MatchEmpty}, potem je $M$ enak $\tmmatch{V}{}$.
		Po lemi~\ref{lem:weakening-values-bb} ima $V$ isti tip v razširjenem kontekstu.
		Po pravilu \rulename{TyComp-MatchEmpty} velja $\Gamma_1, \Gamma_3, \Gamma_2 \types M \of \tycomp{A}{\o, \i}$.
		
		\item Če je zadnje pravilo \rulename{TyComp-MatchPair}, potem je $M$ enak $\tmmatch{V}{(x_1,x_2) \mapsto N}$. 
		Po lemi~\ref{lem:weakening-values-bb} ima $V$ isti tip v razširjenem kontekstu.
		Po indukcijski predpostavki ima $N$ isti tip v razširjenem kontekstu.
		Po pravilu \rulename{TyComp-MatchPair} velja $\Gamma_1, \Gamma_3, \Gamma_2 \types M \of \tycomp{A}{\o, \i}$.
		
		\item Če je zadnje pravilo \rulename{TyComp-MatchSum}, potem je $M$ enak $\tmmatch{V}{\tminl{x_1} \mapsto N_1, \tminr{x_2} \mapsto N_2}$.
		Po lemi~\ref{lem:weakening-values-bb} ima $V$ isti tip v razširjenem kontekstu.potem je bilo zadnje pravilo za določitev tipa $P$ \rulename{TyProc-Run}. Tip $C$ je oblike $\tyrun{A}{\o, \i}$.
		Izračun $M$ ima tip $\tycomp{B}{\o', \i'}$ in $N$ tip $\tycomp{A}{\o, \i}$.
		Po pravilu \rulename{TyProc-Par} ima $P'$ tip $\typar{\tycomp{A}{\o, \i}}{\tycomp{B}{\o', \i'}}$, kjer je $C$ naredil korak \rulename{TyRedu-Spawn-R}.
		Po indukcijski predpostavki imata $N_1$ in $N_2$ isti tip v razširjenem kontekstu.
		Po pravilu \rulename{TyComp-MatchSum} velja $\Gamma_1, \Gamma_3, \Gamma_2 \types M \of \tycomp{A}{\o, \i}$.
		
		\item Če je zadnje pravilo \rulename{TyComp-Apply}, potem je $M$ enak $V_1 V_2$.
		Po lemi~\ref{lem:weakening-values-bb} imata $V_1$ in $V_2$ isti tip v razširjenem kontekstu.
		Po pravilu \rulename{TyComp-Apply} velja $\Gamma_1, \Gamma_3, \Gamma_2 \types M \of \tycomp{A}{\o, \i}$.
		
		\item Če je zadnje pravilo \rulename{TyComp-Signal}, potem je $M$ enak $\tmopout{op}{V}{N}$.
		Po lemi~\ref{lem:weakening-values-bb} ima $V$ isti tip v razširjenem kontekstu.
		Po indukcijski predpostavki ima $N$ isti tip v razširjenem kontekstu.
		Po pravilu \rulename{TyComp-Signal} velja $\Gamma_1, \Gamma_3, \Gamma_2 \types M \of \tycomp{A}{\o, \i}$.
		
		\item Če je zadnje pravilo \rulename{TyComp-Interrupt}, potem je $M$ enak $\tmopin{op}{V}{N}$.
		Po lemi~\ref{lem:weakening-values-bb} ima $V$ isti tip v razširjenem kontekstu.
		Po indukcijski predpostavki ima $N$ isti tip v razširjenem kontekstu.
		Po pravilu \rulename{TyComp-Interrupt} velja $\Gamma_1, \Gamma_3, \Gamma_2 \types M \of \tycomp{A}{\o, \i}$.
		
		\item Če je zadnje pravilo \rulename{TyComp-RecPromise}, potem je $M$ enak $\tmwithrec{op}{x}{r}{N_1}{p}{N_2}$.
		Po indukcijski predpostavki imata $N_1$ in $N_2$ isti tip v razširjenem kontekstu.
		Po pravilu \rulename{TyComp-RecPromise} velja $\Gamma_1, \Gamma_3, \Gamma_2 \types M \of \tycomp{A}{\o, \i}$.
		
		\item Če je zadnje pravilo \rulename{TyComp-Await}, potem je $M$ enak $\tmawait{V}{y}{N}$.
		Po lemi~\ref{lem:weakening-values-bb} ima $V$ isti tip v razširjenem kontekstu.
		Po indukcijski predpostavki ima $N$ isti tip v razširjenem kontekstu.
		Po pravilu \rulename{TyComp-Await} velja $\Gamma_1, \Gamma_3, \Gamma_2 \types M \of \tycomp{A}{\o, \i}$.
		
		\item Če je zadnje pravilo \rulename{TyComp-Subsume}, ima po indukcijski predpostavki $M$ isti tip v razširjenem kontekstu.
		
	\end{itemize}
\end{proof}


\begin{lema}\label{lem:weakening-values-prenosljiv}
	Naj za vrednost $V$ velja $\Gamma_1, \Gamma_2 \types V \of A$. Če je $A$ prenosljiv tip, velja $\Gamma_1, \Gamma_3, \Gamma_2 \types V \of A$ za vsak kontekst $\Gamma_3$, ki vsebuje le $\bb$ in sveže spremenljivke glede na $\Gamma_1$ in $\Gamma_2$.
\end{lema}

\begin{proof}
	Dokazujemo z strukturno indukcijo na drevo izpeljave za $\Gamma_1, \Gamma_2 \types V \of A$.
	Ločimo primere glede na zadnje uporabljeno pravilo.
	
	\begin{itemize}
		\item[\sitem] Če je zadnje pravilo \rulename{TyVal-Var'}, potem je $V = x$ in velja $x \in \Gamma_1$ ali $x \in \Gamma_2$.
		Ker je $A$ prenosljiv tip velja $\Gamma_1, \Gamma_3, \Gamma_2 \types V \of A$ po pravilu \rulename{TyVal-Var'}
		
		\item Če je zadnje pravilo \rulename{TyVal-Cons}, potem je $V = n$. Po pravilu \rulename{TyVal-Cons} velja $\Gamma_1, \Gamma_3, \Gamma_2 \types V \of A$.
%		\item Če je zadnje pravilo \rulename{Ty-Cons-T}, potem je $V = true$. Po pravilu \rulename{Ty-Cons-T} velja $\Gamma_1, \Gamma_3, \Gamma_2 \types V \of A$.
%		\item Če je zadnje pravilo \rulename{Ty-Cons-F}, potem je $V = false$. Po pravilu \rulename{Ty-Cons-F} velja $\Gamma_1, \Gamma_3, \Gamma_2 \types V \of A$.
		
		\item Če je zadnje pravilo \rulename{TyVal-Unit}, potem je $V = ()$. Po pravilu \rulename{TyVal-Unit} velja $\Gamma_1, \Gamma_3, \Gamma_2 \types V \of A$.
		
		\item Če je zadnje pravilo \rulename{TyVal-Pair}, potem je $V =\tmpair{V_1}{V_2}$. Po indukcijski predpostavki imata $V_1$ in $V_2$ isti tip v razširjenem kontekstu.
		Po pravilu \rulename{TyVal-Pair} velja $\Gamma_1, \Gamma_3, \Gamma_2 \types V \of A$.
		
		\item Če je zadnje pravilo \rulename{TyVal-Inl}, potem je $V = \tminl{V_1}$. Po indukcijski predpostavki ima $V_1$ isti tip v razširjenem kontekstu.
		Po pravilu \rulename{TyVal-Inl} velja $\Gamma_1, \Gamma_3, \Gamma_2 \types V \of A$.
		
		\item Če je zadnje pravilo \rulename{TyVal-Inr}, potem je $V = \tminl{V_2}$. Po indukcijski predpostavki ima $V_2$ isti tip v razširjenem kontekstu.
		Po pravilu \rulename{TyVal-Inr} velja $\Gamma_1, \Gamma_3, \Gamma_2 \types V \of A$.
		
		\item Zadnje pravilo ne more biti \rulename{TyVal-Fun}, saj v tem primeru $A$ ni prenosljiv.
		
		\item Zadnje pravilo ne more biti \rulename{TyVal-Fun-Rec}, saj v tem primeru $A$ ni prenosljiv.
		
		\item Zadnje pravilo ne more biti \rulename{TyVal-Promise}, saj v tem primeru $A$ ni prenosljiv.
		
		\item Če je zadnje pravilo \rulename{TyVal-Boxed}, potem je $V = \tmboxed{V'}$.
		Po lemi~\ref{lem:weakening-values-bb} ima $V'$ isti tip v razširjenem kontekstu.
		Po pravilu \rulename{TyVal-Boxed} velja $\Gamma_1, \Gamma_3, \Gamma_2 \types V \of A$.
		
	\end{itemize}
\end{proof}

Sledeči lemi o substituciji za vrednosti in izračune se nanašata ena na drugo.

\begin{lema}[o substituciji za vrednosti]\label{lem:substitucija-vrednosti-2}
	Naj za vrednost $V$ velja $\Gamma_1 \types V \of A$ in za vrednost $W$ velja $\Gamma_1, x \of A, \Gamma_2 \types W \of B$. Potem velja $\Gamma_1, \Gamma_2 \types W[V/x] \of B$.
\end{lema}

\begin{proof}
	Dokazujemo z strukturno indukcijo na drevo izpeljave za $\Gamma_1, x \of A, \Gamma_2 \types W \of B$.
	Ločimo primere glede na zadnje uporabljeno pravilo.
	Primeri, ki so enaki kot v dokazu~\ref{lem:substitucija-vrednosti} so izpuščeni.
	
	\begin{itemize}
		\item Če je zadnje pravilo \rulename{TyVal-Var'}, potem je $W = y$.
		Ločimo dva primera.
		\begin{itemize}
			\item Če je $y = x$, potem je $A = B$ in $W[V/x] = x[V/x] = V$. Če je $A$ prenosljiv tip ima $W[V/x]$ tip $A$ po lemi~\ref{lem:weakening-values-prenosljiv}. Če $A$ ni prenosljiv tip, potem $\Gamma_2$ ne vsebuje $\bb$ in ima $W[V/x]$ tip $A$ po lemi~\ref{lem:weakening-values-2}.
			
			\item Če $Y \neq x$, potem je $W[V/x] = y[V/x] = y = W$. Posledično ima $W[V/x]$ tip $B$.
		\end{itemize}
		
		\item Če je zadnje pravilo \rulename{TyVal-Boxed}, potem je $W=\tmboxed{W'}$ in $B=\tyboxed{B'}$. Po indukcijski predpostavki ima $W'[V/x]$ tip $B'$. Ker je $W[V/x] = \tmboxed{W'[V/x]}$, ima $W[V/x]$ tip $B$, po pravilu \rulename{TyVal-Boxed}.
		
	\end{itemize}
\end{proof}

\begin{lema}[o substituciji za izračune]\label{lem:substitucija-izračuni-2}
	Naj za vrednost $V$ velja $\Gamma_1 \types V \of A$ in za izračun $M$ velja $\Gamma_1, x \of A, \Gamma_2 \types M \of \tycomp{B}{\o, \i}$. Potem velja $\Gamma_1, \Gamma_2 \types M[V/x] \of \tycomp{B}{\o, \i}$.
\end{lema}

\begin{proof}
	Dokazujemo z strukturno indukcijo na drevo izpeljave za $\Gamma_1, x \of A, \Gamma_2 \types M \of \tycomp{B}{\o, \i}$.
	Ločimo primere glede na zadnje uporabljeno pravilo.
	Primeri, ki so enaki kot v dokazu~\ref{lem:substitucija-izračuni} so izpuščeni.
	
	\begin{itemize}
		\item Če je zadnje pravilo \rulename{TyComp-Unbox}, potem je $M$ enak $\tmunbox{W}{x}{N}$. Vrednost $W$ ima tip $B$ in izračun $N$ tip $\tycomp{B}{\o, \i}$.
		Po indukcijski predpostavki velja, da ima $N[V/x]$ isti tip. Po lemi~\ref{lem:substitucija-vrednosti-2} ima $W[V/x]$ isti tip. Ker je $M[V/x]$ enak $\tmunbox{W[V/x]}{x}{N[V/x]}$, ima izračun $M[V/x]$ tip $\tycomp{B}{\o, \i}$ po pravilu \rulename{TyComp-Unbox}.
		
		\item Če je zadnje pravilo \rulename{TyComp-RecPromise}, potem je $M$ enak $\tmwithrec{op}{r}{x}{N_1}{p}{N_2}$. Izračun $N_1$ ima tip $\tycomp{C}{\o', \i'}$ in $N_2$ tip $\tycomp{B}{\o, \i}$.
		Po indukcijski predpostavki velja, da imata $N_1[V/x]$ in $N_2[V/x]$ isti tip. Ker je $M[V/x]$ enak $\tmwithrec{op}{r}{x}{N_1[V/x]}{p}{N_2[V/x]}$, ima izračun $M[V/x]$ tip $\tycomp{B}{\o, \i}$ po pravilu \rulename{TyComp-RecPromise}.
		
		\item Če je zadnje pravilo \rulename{TyComp-Spawn}, potem je $M$ enak $\tmspawn{N_1}{N_2}$. Izračun $N_1$ ima tip $\tycomp{C}{\o', \i'}$ in $N_2$ tip $\tycomp{B}{\o, \i}$.
		Po indukcijski predpostavki velja, da imata $N_1[V/x]$ in $N_2[V/x]$ isti tip. Ker je $M[V/x]$ enak $\tmspawn{N_1[V/x]}{N_2[V/x]}$, ima izračun $M[V/x]$ tip $\tycomp{B}{\o, \i}$ po pravilu \rulename{TyComp-Spawn}.	
	\end{itemize}
	
\end{proof}

Sledeči lemi o krepitvi konteksta za vrednosti in izračune se nanašata ena na drugo.

\begin{lema}\label{lem:strengthening-values-promise}
	Naj za vrednost $V$ velja $\Gamma_1, p \of \typromise{A}, \Gamma_2 \types V \of B$, kjer $\Gamma_2$ vsebuje $\bb$. Potem velja $\Gamma_1, \Gamma_2 \types V \of B$.
\end{lema}

\begin{proof}
	Dokazujemo z strukturno indukcijo na drevo izpeljave za $\Gamma_1, p \of \typromise{A}, \Gamma_2 \types V \of B$.
	Ločimo primere glede na zadnje uporabljeno pravilo.
	
	\begin{itemize}
		\item[\sitem] Če je zadnje pravilo \rulename{TyVal-Var'}, potem je $V = x$.
		\begin{itemize}
			\item Če $x \in \Gamma_1$ ali $x \in \Gamma_2$, potem po pravilu \rulename{TyVal-Var'} velja $\Gamma_1, \Gamma_2 \types V \of B$.
			
			\item Primera ko je $x = p$ ne moremo imeti saj tip $\typromise{A}$ ni prenosljiv, v $\Gamma_2$ pa imamo $\bb$.
		\end{itemize}
		
		\item Če je zadnje pravilo \rulename{TyVal-Cons}, potem je $V = n$. Po pravilu \rulename{TyVal-Cons} velja $\Gamma_1, \Gamma_2 \types V \of B$.
%		\item Če je zadnje pravilo \rulename{Ty-Cons-T}, potem je $V = true$. Po pravilu \rulename{Ty-Cons-T} velja $\Gamma_1, \Gamma_2, \bb \types V \of B$.
%		\item Če je zadnje pravilo \rulename{Ty-Cons-F}, potem je $V = false$. Po pravilu \rulename{Ty-Cons-F} velja $\Gamma_1, \Gamma_2, \bb \types V \of B$.

		\item Če je zadnje pravilo \rulename{TyVal-Unit}, potem je $V = ()$. Po pravilu \rulename{TyVal-Unit} velja $\Gamma_1, \Gamma_2 \types V \of B$.
		
		\item Če je zadnje pravilo \rulename{TyVal-Pair}, potem je $V =\tmpair{V_1}{V_2}$.
		Po indukcijski predpostavki imata $V_1$ in $V_2$ isti tip v skrčenem kontekstu.
		Po pravilu \rulename{TyVal-Pair} velja $\Gamma_1, \Gamma_2 \types V \of B$.
		
		\item Če je zadnje pravilo \rulename{TyVal-Inl}, potem je $V = \tminl{V_1}$.
		Po indukcijski predpostavki ima $V_1$ isti tip v skrčenem kontekstu.
		Po pravilu \rulename{TyVal-Inl} velja $\Gamma_1, \Gamma_2 \types V \of B$.
		
		\item Če je zadnje pravilo \rulename{TyVal-Inr}, potem je $V = \tminl{V_2}$.
		Po indukcijski predpostavki ima $V_2$ isti tip v skrčenem kontekstu.
		Po pravilu \rulename{TyVal-Inr} velja $\Gamma_1, \Gamma_2 \types V \of B$.
		
		\item Če je zadnje pravilo \rulename{TyVal-Fun}, potem je $V = \tmfunano{x}{M}$.
		Po lemi~\ref{lem:strengthening-comp-promise} ima $M$ isti tip v skrčenem kontekstu.
		Po pravilu \rulename{TyVal-Fun} velja $\Gamma_1, \Gamma_2 \types V \of B$.
		
		\item Če je zadnje pravilo \rulename{TyVal-Fun-Rec}, potem je $V = \tmfunrecano{f}{x}{M}$.
		Po lemi~\ref{lem:strengthening-comp-promise} ima $M$ isti tip v skrčenem kontekstu.
		Po pravilu \rulename{TyVal-Fun-Rec} velja $\Gamma_1, \Gamma_2 \types V \of B$.
		
		\item Če je zadnje pravilo \rulename{TyVal-Promise}, potem je $V = \tmpromise{V'}$.
		Po indukcijski predpostavki ima $V'$ isti tip v skrčenem kontekstu.
		Po pravilu \rulename{TyVal-Promise} velja $\Gamma_1, \Gamma_2 \types V \of B$.
		
		\item Če je zadnje pravilo \rulename{TyVal-Boxed}, potem je $V = \tmboxed{V'}$.
		Po indukcijski predpostavki ima $V'$ isti tip v skrčenem kontekstu.
		Po pravilu \rulename{TyVal-Boxed} velja $\Gamma_1, \Gamma_2 \types V \of B$.
		
	\end{itemize}
\end{proof}

\begin{lema}\label{lem:strengthening-comp-promise}
	Naj za izračun $M$ velja $\Gamma_1, p \of \typromise{A}, \Gamma_2 \types M \of \tycomp{B}{\o, \i}$, kjer $\Gamma_2$ vsebuje $\bb$. Potem velja $\Gamma_1, \Gamma_2 \types M \of \tycomp{B}{\o, \i}$.
\end{lema}

\begin{proof}
	Dokazujemo z strukturno indukcijo na drevo izpeljave za $\Gamma_1, p \of \typromise{A}, \Gamma_2 \types M \of \tycomp{B}{\o, \i}$.
	Ločimo primere glede na zadnje uporabljeno pravilo.
	
	\begin{itemize}
		\item Če je zadnje pravilo \rulename{TyComp-Return}, potem je $M$ enak $\tmreturn{V}$.
		Po lemi~\ref{lem:strengthening-values-promise} ima $V$ isti tip v skrčenem kontekstu.
		Po pravilu \rulename{TyComp-Return} velja $\Gamma_1, \Gamma_2 \types M \of \tycomp{B}{\o, \i}$.
		
		\item Če je zadnje pravilo \rulename{TyComp-Let}, potem je $M$ enak $\tmlet{x}{N_1}{N_2}$.
		Po indukcijski predpostavki imata $N_1$ in $N_2$ isti tip v skrčenem kontekstu.
		Po pravilu \rulename{TyComp-Let} velja $\Gamma_1, \Gamma_2 \types M \of \tycomp{B}{\o, \i}$.
		
		\item Če je zadnje pravilo \rulename{TyComp-Apply}, potem je $M$ enak $V_1 V_2$.
		Po lemi~\ref{lem:strengthening-values-promise} imata $V_1$ in $V_2$ isti tip v skrčenempotem je bilo zadnje pravilo za določitev tipa $P$ \rulename{TyProc-Run}. Tip $C$ je oblike $\tyrun{A}{\o, \i}$.
		Izračun $M$ ima tip $\tycomp{B}{\o', \i'}$ in $N$ tip $\tycomp{A}{\o, \i}$.
		Po pravilu \rulename{TyProc-Par} ima $P'$ tip $\typar{\tycomp{A}{\o, \i}}{\tycomp{B}{\o', \i'}}$, kjer je $C$ naredil korak \rulename{TyRedu-Spawn-R}. kontekstu.
		Po pravilu \rulename{TyComp-Apply} velja $\Gamma_1, \Gamma_2 \types M \of \tycomp{B}{\o, \i}$.
		
		\item Če je zadnje pravilo \rulename{TyComp-MatchPair}, potem je $M$ enak $\tmmatch{V}{(x_1,x_2) \mapsto N}$. 
		Po lemi~\ref{lem:strengthening-values-promise} ima $V$ isti tip v skrčenem kontekstu.
		Po indukcijski predpostavki ima $N$ isti tip v skrčenem kontekstu.
		Po pravilu \rulename{TyComp-MatchPair} velja $\Gamma_1, \Gamma_2 \types M \of \tycomp{B}{\o, \i}$.
		
		\item Če je zadnje pravilo \rulename{TyComp-MatchEmpty}, potem je $M$ enak $\tmmatch{V}{}$.
		Po lemi~\ref{lem:strengthening-values-promise} ima $V$ isti tip v skrčenem kontekstu.
		Po pravilu \rulename{TyComp-MatchEmpty} velja $\Gamma_1, \Gamma_2 \types M \of \tycomp{B}{\o, \i}$.
		
		\item Če je zadnje pravilo \rulename{TyComp-MatchSum}, potem je $M$ enak $\tmmatch{V}{\tminl{x_1} \mapsto N_1, \tminr{x_2} \mapsto N_2}$.
		Po lemi~\ref{lem:strengthening-values-promise} ima $V$ isti tip v skrčenem kontekstu.
		Po indukcijski predpostavki imata $N_1$ in $N_2$ isti tip v skrčenem kontekstu.
		Po pravilu \rulename{TyComp-MatchSum} velja $\Gamma_1, \Gamma_2 \types M \of \tycomp{B}{\o, \i}$.
		
		\item Če je zadnje pravilo \rulename{TyComp-Signal}, potem je $M$ enak $\tmopout{op}{V}{N}$.
		Po lemi~\ref{lem:strengthening-values-promise} ima $V$ isti tip v skrčenem kontekstu.
		Po indukcijski predpostavki ima $N$ isti tip v skrčenem kontekstu.
		Po pravilu \rulename{TyComp-Signal} velja $\Gamma_1, \Gamma_2 \types M \of \tycomp{B}{\o, \i}$.
		
		\item Če je zadnje pravilo \rulename{TyComp-Interrupt}, potem je $M$ enak $\tmopin{op}{V}{N}$.
		Po lemi~\ref{lem:strengthening-values-promise} ima $V$ isti tip v skrčenem kontekstu.
		Po indukcijski predpostavki ima $N$ isti tip v skrčenem kontekstu.
		Po pravilu \rulename{TyComp-Interrupt} velja $\Gamma_1, \Gamma_2 \types M \of \tycomp{B}{\o, \i}$.
		
		\item Če je zadnje pravilo \rulename{TyComp-RecPromise}, potem je $M$ enak $\tmwithrec{op}{x}{r}{N_1}{p}{N_2}$.
		Po indukcijski predpostavki imata $N_1$ in $N_2$ isti tip v skrčenem kontekstu.
		Po pravilu \rulename{TyComp-RecPromise} velja $\Gamma_1, \Gamma_2 \types M \of \tycomp{B}{\o, \i}$.
		
		\item Če je zadnje pravilo \rulename{TyComp-Await}, potem je $M$ enak $\tmawait{V}{y}{N}$.
		Po lemi~\ref{lem:strengthening-values-promise} ima $V$ isti tip v skrčenem kontekstu.
		Po indukcijski predpostavki ima $N$ isti tip v skrčenem kontekstu.
		Po pravilu \rulename{TyComp-Await} velja $\Gamma_1, \Gamma_2 \types M \of \tycomp{B}{\o, \i}$.
		
		\item Če je zadnje pravilo \rulename{TyComp-Subsume}, ima po indukcijski predpostavki izračun $M$ isti tip v skrčenem kontekstu.
		
	\end{itemize}
\end{proof}


\begin{lema}\label{lem:tovor-osnovni-tip-skrcitev-2}
	Naj za vrednost $V$ velja $\Gamma_1, x \of \typromise{A}, \Gamma_2 \types V \of B$, kjer je $B$ prenosljiv tip. Potem velja $\Gamma_1, \Gamma_2 \types V \of B$.
\end{lema}

\begin{proof}
	Dokazujemo z strukturno indukcijo na drevo izpeljave za $\Gamma_1, p \of \typromise{A}, \Gamma_2 \types V \of B$.
	Ločimo primere glede na zadnje uporabljeno pravilo.
	
	\begin{itemize}
		\item[\sitem] Če je zadnje pravilo \rulename{TyVal-Var'}, potem je $V = y$. Ker $\typromise{A}$ ni prenosljiv tip, velja $y \in \Gamma_1$ ali $y \in \Gamma_2$.
		Po pravilu \rulename{TyVal-Var'} velja $\Gamma_1, \Gamma_2, \bb \types V \of B$.
		
		\item Če je zadnje pravilo \rulename{TyVal-Cons}, potem je $V = n$. Po pravilu \rulename{TyVal-Cons} velja $\Gamma_1, \Gamma_2 \types V \of B$.
%		\item Če je zadnje pravilo \rulename{Ty-Cons-T}, potem je $V = true$. Po pravilu \rulename{Ty-Cons-T} velja $\Gamma_1, \Gamma_2 \types V \of B$.
%		\item Če je zadnje pravilo \rulename{Ty-Cons-F}, potem je $V = false$. Po pravilu \rulename{Ty-Cons-F} velja $\Gamma_1, \Gamma_2 \types V \of B$.

		\item Če je zadnje pravilo \rulename{TyVal-Unit}, potem je $V = ()$. Po pravilu \rulename{TyVal-Unit} velja $\Gamma_1, \Gamma_2 \types V \of B$.
		
		\item Če je zadnje pravilo \rulename{TyVal-Pair}, potem je $V =\tmpair{V_1}{V_2}$.
		Po indukcijski predpostavki imata $V_1$ in $V_2$ isti tip v skrčenem kontekstu.
		Po pravilu \rulename{TyVal-Pair} velja $\Gamma_1, \Gamma_2, \bb \types V \of B$.
		
		\item Če je zadnje pravilo \rulename{TyVal-Inl}, potem je $V = \tminl{V_1}$.
		Po indukcijski predpostavki ima $V_1$ isti tip v skrčenem kontekstu.
		Po pravilu \rulename{TyVal-Inl} velja $\Gamma_1, \Gamma_2, \bb \types V \of B$.
		
		\item Če je zadnje pravilo \rulename{TyVal-Inr}, potem je $V = \tminl{V_2}$.
		Po indukcijski predpostavki ima $V_2$ isti tip v skrčenem kontekstu.
		Po pravilu \rulename{TyVal-Inr} velja $\Gamma_1, \Gamma_2, \bb \types V \of B$.
		
		\item Zadnje pravilo ne more biti \rulename{TyVal-Fun}, saj v tem primeru $B$ ni prenosljiv.
		
		\item Zadnje pravilo ne more biti \rulename{TyVal-Fun-Rec}, saj v tem primeru $B$ ni prenosljiv.
		
		\item Zadnje pravilo ne more biti \rulename{TyVal-Promise}, saj v tem primeru $B$ ni prenosljiv.
		
		\item Če je zadnje pravilo \rulename{TyVal-Boxed}, potem je $V = \tmboxed{V'}$.
		Po lemi~\ref{lem:strengthening-values-promise} ima $V'$ isti tip v skrčenem kontekstu.
		Po pravilu \rulename{TyVal-Boxed} velja $\Gamma_1, \Gamma_2, \bb \types V \of B$.
		
	\end{itemize}
\end{proof}

Sledeči lemi o krepitvi konteksta za vrednosti in izračune se nanašata ena na drugo.

\begin{lema}\label{lem:strengthening-values-bb}
	Naj za vrednost $V$ velja $\Gamma_1, \bb, \Gamma_2 \types V \of A$. Tedaj velja $\Gamma_1, \Gamma_2 \types V \of A$.
\end{lema}


\begin{proof}
	Dokazujemo z strukturno indukcijo na drevo izpeljave za $\Gamma_1, \bb, \Gamma_2 \types V \of A$.
	Ločimo primere glede na zadnje uporabljeno pravilo.
	
	\begin{itemize}
		\item[\sitem] Če je zadnje pravilo \rulename{TyVal-Var'}, potem je $V = x$ in velja $x \in \Gamma_1$ ali $x \in \Gamma_2$.
		Po pravilu \rulename{TyVal-Var'} velja $\Gamma_1, \Gamma_2 \types V \of A$.

		\item Če je zadnje pravilo \rulename{TyVal-Boxed}, potem je $V = \tmboxed{V'}$.
		Po indukcijski predpostavki ima $V'$ isti tip v skrčenem kontekstu.
		Po pravilu \rulename{TyVal-Boxed} velja $\Gamma_1, \Gamma_2 \types V \of A$.
		
		\item Če je zadnje pravilo \rulename{TyVal-Cons}, potem je $V = n$. Po pravilu \rulename{TyVal-Cons} velja $\Gamma_1, \Gamma_2 \types V \of A$.
		
		\item Če je zadnje pravilo \rulename{TyVal-Unit}, potem je $V = ()$. Po pravilu \rulename{TyVal-Unit} velja $\Gamma_1, \Gamma_2 \types V \of A$.
		
		\item Če je zadnje pravilo \rulename{TyVal-Pair}, potem je $V =\tmpair{V_1}{V_2}$.
		Po indukcijski predpostavki imata $V_1$ in $V_2$ isti tip v skrčenem kontekstu.
		Po pravilu \rulename{TyVal-Pair} velja $\Gamma_1, \Gamma_2 \types V \of A$.
		
		\item Če je zadnje pravilo \rulename{TyVal-Promise}, potem je $V = \tmpromise{V'}$.
		Po indukcijski predpostavki ima $V'$ isti tip v skrčenem kontekstu.
		Po pravilu \rulename{TyVal-Promise} velja $\Gamma_1, \Gamma_2 \types V \of A$.
		
		\item Če je zadnje pravilo \rulename{TyVal-Inl}, potem je $V = \tminl{V_1}$.
		Po indukcijski predpostavki ima $V_1$ isti tip v skrčenem kontekstu.
		Po pravilu \rulename{TyVal-Inl} velja $\Gamma_1, \Gamma_2 \types V \of A$.
		
		\item Če je zadnje pravilo \rulename{TyVal-Inr}, potem je $V = \tminl{V_2}$.
		Po indukcijski predpostavki ima $V_2$ isti tip v skrčenem kontekstu.
		Po pravilu \rulename{TyVal-Inr} velja $\Gamma_1, \Gamma_2 \types V \of A$.
		
		\item Če je zadnje pravilo \rulename{TyVal-Fun}, potem je $V = \tmfunano{x}{M}$.
		Po lemi~\ref{lem:strengthening-comp-bb} ima $M$ isti tip v skrčenem kontekstu.
		Po pravilu \rulename{TyVal-Fun} velja $\Gamma_1, \Gamma_2 \types V \of A$.
		
		\item Če je zadnje pravilo \rulename{TyVal-Fun-Rec}, potem je $V = \tmfunrecano{f}{x}{M}$.
		Po lemi~\ref{lem:strengthening-comp-bb} ima $M$ isti tip v skrčenem kontekstu.
		Po pravilu \rulename{TyVal-Fun-Rec} velja $\Gamma_1, \Gamma_2 \types V \of A$.
		
	\end{itemize}
\end{proof}

\begin{lema}\label{lem:strengthening-comp-bb}
	Naj za izračun $M$ velja $\Gamma_1, \bb, \Gamma_2 \types M \of \tycomp{A}{\o, \i}$. Tedaj velja $\Gamma_1, \Gamma_2 \types M \of \tycomp{A}{\o, \i}$.
\end{lema}

\begin{proof}
	Dokazujemo z strukturno indukcijo na drevo izpeljave za $\Gamma_1, \bb, \Gamma_2 \types M \of \tycomp{A}{\o, \i}$.
	Ločimo primere glede na zadnje uporabljeno pravilo.
	
	\begin{itemize}
		\item Če je zadnje pravilo \rulename{TyComp-Return}, potem je $M$ enak $\tmreturn{V}$.
		Po lemi~\ref{lem:strengthening-values-bb} ima $V$ isti tip v skrčenem kontekstu.
		Po pravilu \rulename{TyComp-Return} velja $\Gamma_1, \Gamma_2 \types M \of \tycomp{A}{\o, \i}$.
		
		\item Če je zadnje pravilo \rulename{TyComp-Let}, potem je $M$ enak $\tmlet{x}{N_1}{N_2}$.
		Po indukcijski predpostavki imata $N_1$ in $N_2$ isti tip v skrčenem kontekstu.
		Po pravilu \rulename{TyComp-Let} velja $\Gamma_1, \Gamma_2 \types M \of \tycomp{A}{\o, \i}$.
		
		\item Če je zadnje pravilo \rulename{TyComp-Apply}, potem je $M$ enak $V_1 V_2$.
		Po lemi~\ref{lem:strengthening-values-bb} imata $V_1$ in $V_2$ isti tip v skrčenem kontekstu.
		Po pravilu \rulename{TyComp-Apply} velja $\Gamma_1, \Gamma_2 \types M \of \tycomp{A}{\o, \i}$.
		
		\item Če je zadnje pravilo \rulename{TyComp-MatchPair}, potem je $M$ enak $\tmmatch{V}{(x_1,x_2) \mapsto N}$. 
		Po lemi~\ref{lem:strengthening-values-bb} ima $V$ isti tip v skrčenem kontekstu.
		Po indukcijski predpostavki ima $N$ isti tip v skrčenem kontekstu.
		Po pravilu \rulename{TyComp-MatchPair} velja $\Gamma_1, \Gamma_2 \types M \of \tycomp{A}{\o, \i}$.
		
		\item Če je zadnje pravilo \rulename{TyComp-MatchEmpty}, potem je $M$ enak $\tmmatch{V}{}$.
		Po lemi~\ref{lem:strengthening-values-bb} ima $V$ isti tip v skrčenem kontekstu.
		Po pravilu \rulename{TyComp-MatchEmpty} velja $\Gamma_1, \Gamma_2 \types M \of \tycomp{A}{\o, \i}$.
		
		\item Če je zadnje pravilo \rulename{TyComp-MatchSum}, potem je $M$ enak $\tmmatch{V}{\tminl{x_1} \mapsto N_1, \tminr{x_2} \mapsto N_2}$.
		Po lemi~\ref{lem:strengthening-values-bb} ima $V$ isti tip v skrčenem kontekstu.
		Po indukcijski predpostavki imata $N_1$ in $N_2$ isti tip v skrčenem kontekstu.
		Po pravilu \rulename{TyComp-MatchSum} velja $\Gamma_1, \Gamma_2 \types M \of \tycomp{A}{\o, \i}$.
		
		\item Če je zadnje pravilo \rulename{TyComp-Signal}, potem je $M$ enak $\tmopout{op}{V}{N}$.
		Po lemi~\ref{lem:strengthening-values-bb} ima $V$ isti tip v skrčenem kontekstu.
		Po indukcijski predpostavki ima $N$ isti tip v skrčenem kontekstu.
		Po pravilu \rulename{TyComp-Signal} velja $\Gamma_1, \Gamma_2 \types M \of \tycomp{A}{\o, \i}$.
		
		\item Če je zadnje pravilo \rulename{TyComp-Interrupt}, potem je $M$ enak $\tmopin{op}{V}{N}$.
		Po lemi~\ref{lem:strengthening-values-bb} ima $V$ isti tip v skrčenem kontekstu.
		Po indukcijski predpostavki ima $N$ isti tip v skrčenem kontekstu.
		Po pravilu \rulename{TyComp-Interrupt} velja $\Gamma_1, \Gamma_2 \types M \of \tycomp{A}{\o, \i}$.
		
		\item Če je zadnje pravilo \rulename{TyComp-RecPromise}, potem je $M$ enak $\tmwithrec{op}{x}{r}{N_1}{p}{N_2}$.
		Po indukcijski predpostavki imata $N_1$ in $N_2$ isti tip v skrčenem kontekstu.
		Po pravilu \rulename{TyComp-RecPromise} velja $\Gamma_1, \Gamma_2 \types M \of \tycomp{A}{\o, \i}$.
		
		\item Če je zadnje pravilo \rulename{TyComp-Await}, potem je $M$ enak $\tmawait{V}{y}{N}$.
		Po lemi~\ref{lem:strengthening-values-bb} ima $V$ isti tip v skrčenem kontekstu.
		Po indukcijski predpostavki ima $N$ isti tip v skrčenem kontekstu.
		Po pravilu \rulename{TyComp-Await} velja $\Gamma_1, \Gamma_2 \types M \of \tycomp{A}{\o, \i}$.
		
		\item Če je zadnje pravilo \rulename{TyComp-Subsume}, ima po indukcijski predpostavki izračun $M$ isti tip v skrčenem kontekstu.
		
	\end{itemize}
\end{proof}

Tudi sedaj bomo potrebovali lemo o inverziji. Vse točke iz leme~\ref{lem:inversion-lema} veljajo tudi sedaj, vendar jih ne naštejemo še enkrat.

\begin{lema}\label{lem:inversion-lema-2}
	Naj za izračun $M$ velja $\Gamma \types M \of \tycomp{A}{\o, \i}$. Potem obstaja $(\o', \i') \order{O, I} (\o, \i)$.
	\begin{itemize}
		\item Če je $M$ oblike $\tmwithrec{op}{x}{r}{N_1}{p}{N_2}$, potem velja $\Gamma, x \of B_{op}, r \of \tyfun{\tyunit}{\tycomp{\typromise{B}}{\o'', \i''}} \types N_1 \of \tycomp{\typromise{B}}{\o'', \i''}$ in $\Gamma, p \of \typromise{B} \types N_1 \of \tycomp{A}{\o', \i'}$.
		
		\item Če je $M$ oblike $\tmunbox{V}{x}{N}$, potem velja $\Gamma \types V \of \tyboxed{B}$ in $\Gamma, x \of A \types N \of \tycomp{A}{\o', \i'}$.
		
		\item Če je $M$ oblike $\tmspawn{N_1}{N_2}$, potem velja $\Gamma, \bb \types N_1 \of \tycomp{B}{\o'', \i''}$ in $\Gamma \types N_2 \of \tycomp{A}{\o', \i'}$.
	\end{itemize}
\end{lema}

\begin{proof}
	Dokazujemo z strukturno indukcijo na drevo izpeljave za $\Gamma \types M \of \tycomp{A}{\o, \i}$.
	Ločimo primere glede na zadnje uporabljeno pravilo.
	
	\begin{itemize}
		\item Če je zadnje pravilo \rulename{TyComp-RecPromise}, \rulename{TyComp-Unbox} ali \rulename{TyComp-Spawn}, potem ustrezna točka leme sledi za $(\o', \i') = (\o, \i)$.
		
		\item Če je zadnje uporabljeno pravilo \rulename{TyComp-Subsume}, potem ima $M$ tudi tip $\tycomp{A}{\o', \i'}$. Po indukcijski predpostavki direktno na $M$ lema sledi, saj je $\order{O, I}$ tranzitiven.
	\end{itemize}
\end{proof}


\begin{trditev}[o ohranitvi]\label{trd:ohranitev-izracuni-2}
	Naj za izračun $M$ velja $\Gamma \types M \of \tycomp{A}{\o, \i}$. Če izračun $M$ naredi korak $M \reduces M'$, potem velja $\Gamma \types M' \of \tycomp{A}{\o, \i}$.
\end{trditev}

\begin{proof}
	Dokazujemo z strukturno indukcijo na drevo izpeljave koraka $M \reduces M'$.
	Ločimo primere glede na zadnje uporabljeno pravilo.
	Primeri, ki so enaki kot v dokazu~\ref{trd:ohranitev-izracuni} so izpuščeni.

	\begin{itemize}
		\item Če je zadnje pravilo $$\tmunbox{[V]}{x}{N} \reduces N[V/x],$$ potem ima po lemi~\ref{lem:inversion-lema-2} vrednost $\tmboxed{V}$ tip $\tyboxed{B}$ in $N$ tip $\tycomp{A}{\o', \i'}$.
		Po lemi~\ref{lem:substitucija-izračuni-2} ima $M' = N[V/x]$ tip $\tycomp{A}{\o', \i'}$.
		Po pravilu \rulename{TyComp-Subsume} ima $M'$ tip $\tycomp{A}{\o, \i}$.
		
		\item Če je zadnje pravilo $$\tmlet{x}{(\tmwithrec{op}{y}{r}{N_1}{p}{N_2})}{N_3} \reduces $$ $$ \tmwithrec{op}{y}{r}{N_1}{p}{(\tmlet{x}{N_2}{N_3})},$$ potem ima po lemi~\ref{lem:inversion-lema-2} izračun $N_1$ tip $\tycomp{B_1}{\o'', \i''}$, $N_2$ tip $\tycomp{B_2}{\o', \i'}$ in $N_3$ tip $\tycomp{A}{\o', \i'}$.
		Po pravilu \rulename{TyComp-Let} in lemi~\ref{lem:weakening-comp-2} ima $\tmlet{x}{N_2}{N_3}$ tip $\tycomp{A}{\o', \i'}$.
		Po pravilu \rulename{TyComp-RecPromise} ima $M'$ tip $\tycomp{A}{\o', \i'}$.
		
		\item Če je zadnje pravilo $$\tmwithrec{op}{x}{r}{N_1}{p}{\tmopout{op'}{V}{N_2}} \reduces $$ $$\tmopout{op'}{V}{\tmwithrec{op}{x}{r}{N_1}{p}{N_2}},$$ potem ima po lemi~\ref{lem:inversion-lema-2} izračun $N_1$ tip $\tycomp{B}{\o'', \i''}$, $V$ tip $C_{op'}$ in $N_2$ tip $\tycomp{A}{\o', \i'}$.
		Po pravilu \rulename{TyComp-RecPromise} ima $\tmwithrec{op}{x}{r}{N_1}{p}{N_2}$ tip $\tycomp{A}{\o', \i'}$.
		Po lemi~\ref{lem:tovor-osnovni-tip-skrcitev-2} ima $V$ še zmeraj tip $C_{op}$ tudi v manjšem kontekstu.
		Po pravilu \rulename{TyComp-Signal} ima $M'$ tip $\tycomp{A}{\o', \i'}$.
		
		\item Če je zadnje pravilo $$\tmopin{op}{V}{\tmwithrec{op}{x}{r}{N_1}{p}{N_2}} \reduces $$ $$ \tmlet{p}{N_1[V/x, R/r]}{\tmopin{op}{V}{N_2}},$$ potem ima po lemi~\ref{lem:inversion-lema-2} vrednost $V$ tip $B_{op}$, $N_1$ tip $\tycomp{\typromise{B}}{\o'', \i''}$, $N_2$ tip $\tycomp{A}{\o', \i'}$.
		Po pravilu \rulename{TyComp-Interrupt} ima $\tmopin{op}{V}{N_2}$ tip $\tycomp{A}{\opincomp{op}{(\o', \i')}}$.
		Ker velja $(\o'', \i'') \order{O, I} \i'(op)$, velja $(\o'', \i'') \order{O \times I} \opincomp{op}{(\o', \i')}$ in posledično ima po pravilu \rulename{TyComp-Subsume} izračun $N_1[V/x]$ tip $\tycomp{\typromise{B}}{\opincomp{op}{(\o', \i')}}$
		Po pravilu \rulename{TyComp-Let} ima $M'$ tip $\tycomp{A}{\opincomp{op}{(\o', \i')}}$.
		Po pravilu \rulename{TyComp-Subsume} ima $M'$ tip $\tycomp{A}{\o, \i}$.
		
		\item Če je zadnje pravilo $$\tmopin{op'}{V}{\tmwithrec{op}{x}{r}{N_1}{p}{N_2}} \reduces $$ $$ \tmwithrec{op}{x}{r}{N_1}{p}{\tmopin{op'}{V}{N_2}},$$ potem ima po lemi~\ref{lem:inversion-lema-2} vrednost $V$ tip $B_{op}$, $N_1$ tip $\tycomp{\typromise{B}}{\o'', \i''}$, $N_2$ tip $\tycomp{A}{\o', \i'}$.
		Po pravilu \rulename{TyComp-Interrupt} ima $\tmopin{op}{V}{N_2}$ tip $\tycomp{A}{\opincomp{op}{(\o', \i')}}$.
		Po pravilu \rulename{TyComp-RecPromise} ima $M'$ tip $\tycomp{A}{\opincomp{op}{(\o', \i')}}$.
		Po pravilu \rulename{TyComp-Subsume} ima $M'$ tip $\tycomp{A}{\o, \i}$.

		\item Če je zadnje pravilo $$\tmlet{x}{(\tmspawn{N_1}{N_2})}{N_3} \reduces \tmspawn{N_1}{\tmlet{x}{N_2}{N_3}},$$ potem ima po lemi~\ref{lem:inversion-lema-2} izračun $N_1$ tip $\tycomp{\typromise{B'}}{\o'', \i''}$, $N_2$ tip $\tycomp{B}{\o', \i'}$ in $N_3$ tip $\tycomp{A}{\o', \i'}$.
		Po pravilu \rulename{TyComp-Let} ima $\tmlet{x}{N_2}{N_3}$ tip $\tycomp{A}{\o', \i'}$.
		Po pravilu \rulename{TyComp-Spawn} ima $M'$ tip $\tycomp{A}{\o', \i'}$.
		Po pravilu \rulename{TyComp-Subsume} ima $M'$ tip $\tycomp{A}{\o, \i}$.

		\item Če je zadnje pravilo $$\tmwithrec{op}{x}{r}{N_1}{p}{\tmspawn{N_2}{N_3}} \reduces $$ $$ \tmspawn{N_2}{\tmwithrec{op}{x}{r}{N_1}{p}{N_3}},$$ potem ima po lemi~\ref{lem:inversion-lema-2} izračun $N_1$ tip $\tycomp{\typromise{B}}{\o'', \i''}$, $N_2$ tip $\tycomp{C}{\o''', \i'''}$ in $N_3$ tip $\tycomp{A}{\o', \i'}$.
		Po pravilu \rulename{TyComp-RecPromise} ima $\tmwithrec{op}{x}{r}{N_1}{p}{N_3}$ tip $\tycomp{A}{\o', \i'}$.
		Po lemi~\ref{lem:strengthening-comp-promise} ima $N_2$ tip $\tycomp{C}{\o''', \i'''}$ tudi v manjšem kontekstu.
		Po pravilu \rulename{TyComp-Spawn} ima $M'$ tip $\tycomp{A}{\o', \i'}$.
		Po pravilu \rulename{TyComp-Subsume} ima $M'$ tip $\tycomp{A}{\o, \i}$.

		\item Če je zadnje pravilo $$\tmopin{op}{V}{\tmspawn{N_1}{N_2}} \reduces \tmspawn{N_1}{\tmopin{op}{V}{N_2}},$$ potem ima po lemi~\ref{lem:inversion-lema-2} vrednost $V$ tip $B$, $N_1$ tip $\tycomp{C}{\o'', \i''}$ in $N_2$ tip $\tycomp{A}{\o', \i'}$.
		Po pravilu \rulename{TyComp-Interrupt} ima $\tmopin{op}{V}{N_2}$ tip $\tycomp{A}{\o', \i'}$.
		Po pravilu \rulename{TyComp-Spawn} ima $M'$ tip $\tycomp{A}{\o', \i'}$.
		Po pravilu \rulename{TyComp-Subsume} ima $M'$ tip $\tycomp{A}{\o, \i}$.
		
	\end{itemize}
	
\end{proof}


Ko iz enega procesa ustvarimo dva paralelna procesa se temu primerno tip spremeni v paralelni tip. 
Redukcijskim pravilom na sliki~\ref{fig:process-type-reductions}, dodamo pravili na sliki~\ref{fig:process-type-reductions-spawn}.

\begin{figure}[H]
	\centering
	\begin{mathpar}
		\coopinfer{TyRedu-Spawn-L}{
		}{
			\tyrun{A}{\o, \i} \tyreduces \typar{\tyrun{A'}{\o', \i'}}{\tyrun{A}{\o, \i}}
		}	
		\quad
		\coopinfer{TyRedu-Spawn-R}{
		}{
			\tyrun{A}{\o, \i} \tyreduces \typar{\tyrun{A}{\o, \i}}{\tyrun{A'}{\o', \i'}}
		}
	\end{mathpar}
	\caption{Dodatna pravila za redukcijo tipov.}
	\label{fig:process-type-reductions-spawn}
	
\end{figure}

\begin{izrek}[o ohranitvi]\label{izr:ohranitev-2}
	Naj za proces $P$ velja $\Gamma \types P \of C$. Če proces $P$ naredi korak $P \reduces P'$, potem obstaja tak $C'$, da velja $C \tyreduces C'$ in $\Gamma \types P' \of C'$.
\end{izrek}


\begin{proof}
	Dokazujemo z strukturno indukcijo na drevo izpeljave koraka $P \reduces P'$.
	Ločimo primere glede na zadnje uporabljeno pravilo.
	Primeri, ki so enaki kot v dokazu~\ref{izr:ohranitev} so izpuščeni.
	
	\begin{itemize}
		
		\item Če je zadnje uporabljeno pravilo
		$$\tmrun{(\tmspawn{M}{N})} \reduces \tmpar{\tmrun{M}}{\tmrun{N}},$$
		potem je bilo zadnje pravilo za določitev tipa $P$ \rulename{TyProc-Run}. Tip $C$ je oblike $\tyrun{A}{\o, \i}$.
		Izračun $M$ ima tip $\tycomp{B}{\o', \i'}$ in $N$ tip $\tycomp{A}{\o, \i}$.
		Po lemi~\ref{lem:strengthening-comp-bb} ima $M$ isti tip tudi v kontekstu brez $\bb$.
		Po pravilu \rulename{TyProc-Par} ima $P'$ tip $\typar{\tycomp{B}{\o', \i'}}{\tycomp{A}{\o, \i}}$, kjer je $C$ naredil korak \rulename{TyRedu-Spawn-L}.
		
		\item Če je zadnje uporabljeno pravilo
		$$\tmrun{(\tmspawn{M}{N})} \reduces \tmpar{\tmrun{N}}{\tmrun{M}},$$
		je dokaz podoben, kot v prejšnji točki.
	\end{itemize}

\end{proof}





