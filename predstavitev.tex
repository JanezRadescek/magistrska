\documentclass{beamer}
\usepackage[slovene]{babel}
\usepackage[utf8]{inputenc}
\usepackage[T1]{fontenc}

\usetheme{Montpellier} %beamer
\usecolortheme{beaver}

\usepackage{listings} %koda
\usepackage{mathpartir} % inference rules
\usepackage{mathtools} % mathllap


% bold matematika znotraj \textbf{ }, tudi v naslovih, kot \omega spodaj
\makeatletter \g@addto@macro\bfseries{\boldmath} \makeatother



% ukazi za matematična okolja
\theoremstyle{definition} % tekst napisan pokončno
\newtheorem{definicija}{Definicija}[section]
\newtheorem{primer}[definicija]{Primer}
\newtheorem{opomba}[definicija]{Opomba}
\newtheorem{aksiom}{Aksiom}

\theoremstyle{plain} % tekst napisan poševno
\newtheorem{lema}[definicija]{Lema}
\newtheorem{izrek}[definicija]{Izrek}
\newtheorem{trditev}[definicija]{Trditev}
\newtheorem{posledica}[definicija]{Posledica}

% !TEX root = paper.tex

% Any macro that is actually used should have a comment explaining what it is for.
% Please fight macro pollution and remove the macros that are not used.

\newcommand{\defeq}{\mathrel{\overset{\text{\tiny def}}{=}}} % Definitional equality

\newcommand{\pl}[1]{\textsc{#1}} % the name of a programming language

\newcommand{\lambdaAEff}{$\lambda_{\text{\ae}}$} % the name of the calculus

\newcommand{\lae}{$\lambda_{\text{\ae}}$} %shorter name
\newcommand{\aeff}{Æff} % the name of the language

% BNF grammars
\newcommand{\bnfis}{\mathrel{\;{:}{:}{=}\ }}
\newcommand{\bnfor}{\mathrel{\;\big|\ \ }}

%%%%% Semantic concepts

%%% Sets

\newcommand{\One}{\mathbb{1}} % singleton set as denotation of unit type
\newcommand{\one}{\star} % canonical element of the singleton set
\newcommand{\Zero}{\mathbb{0}} % empty set as denotation of empty type

\newcommand{\Bool}{\mathbb{B}} % two-element set of booleans
\newcommand{\true}{\mathbf{true}} % constant true
\newcommand{\false}{\mathbf{false}} % constant false

\newcommand{\expto}{\Rightarrow} % set exponentiation
\newcommand{\lam}[1]{\lambda #1 \,.\,} % lambda abstraction
\newcommand{\pair}[2]{\langle #1 , #2 \rangle} % pairing

\newcommand{\lifted}[1]{#1_\bot} % lifting monad
\newcommand{\idte}[4]{\mathbf{ifdef}~#1~\mathbf{then}~#2 \mapsto #3~\mathbf{else}~#4} % test if element of a lifted set is defined (non-bottom) or not, and then use it in the then branch

\newcommand{\ite}[3]{\mathbf{if}~#1~\mathbf{then}~#2~\mathbf{else}~#3} % if-then-else used in semantic definitions


%%% Signatures

\newcommand{\Tree}[2]{\mathrm{Tree}_{#1}\left(#2\right)} % The tree algebra for an operation signature
\newcommand{\retTree}[1]{\mathsf{return}\,#1} % the inclusion of generators into trees

\newcommand{\opsym}[1]{\mathsf{#1}} % a custom operation symbol
\newcommand{\op}{\opsym{op}} % a generic operation symbol

\newcommand{\sig}{\Sigma} % the global signature of signal and interrupt names

\renewcommand{\o}{o} % effect annotation describing possible outgoing operations
\renewcommand{\i}{\iota} % effect annotation describing possible incoming operations

\newcommand{\opincomp}[2]{{\mathsf{#1}}\,{\tmkw{\downarrow}}\,#2} % action of incoming interrupt on computation types
\newcommand{\opincompp}[2]{{\mathsf{#1}}\,{\tmkw{\downarrow\downarrow}}\,#2} % action of a list of incoming interrupts on computation types

%%% Theories
\newcommand{\eq}{\mathrm{Eq}} % a set of equations

\newcommand{\FreeAlg}[2]{\mathrm{Free}_{#1}\left(#2\right)} % Free algebra for a signature generated by a set
\newcommand{\lift}[1]{#1^\dagger} % the Kleisli lifting of a map
\newcommand{\freelift}[1]{#1^\ddagger} % the lifting of a map induced by the free model property

\newcommand{\M}{\mathcal{M}} % a generic model for a theory
\newcommand{\Mcarrier}{\vert \mathcal{M} \vert} % the carrier of a generic model

\newcommand{\T}{T} % A generic monad


%%% Example effect theories

\newcommand{\sigget}{\mathsf{get}}
\newcommand{\sigset}{\mathsf{set}}


%%%%% Types

\newcommand{\at}{\mathbin{!}} % the ! sign, with proper spacing
\newcommand{\att}{\mathbin{!!}} % the !! sign, with proper spacing

%% Value types

\newcommand{\tysym}[1]{\mathsf{#1}}
\newcommand{\tybase}{\tysym{b}} % a base type
\newcommand{\tyunit}{\tysym{1}} % the unit ground type
\newcommand{\tyint}{\tysym{int}} % the integer ground type
\newcommand{\tystring}{\tysym{string}} % the integer ground type
\newcommand{\tylist}[1]{\tysym{list}~\tysym{#1}} % the list ground type
\newcommand{\tyempty}{\tysym{0}} % the empty ground type
\newcommand{\typrod}[2]{#1 \times #2} % product type
\newcommand{\tysum}[2]{#1 + #2} % sum type
\newcommand{\tyfun}[2]{#1 \to #2} % user function type
\newcommand{\typromise}[1]{\langle #1 \rangle} % type of promises

%% Computation types

%\newcommand{\tycomp}[2]{#1 \at #2} % computation type
\newcommand{\tycomp}[1]{#1 \at} % computation type. No o/i

%% Process types

%\newcommand{\tyrun}[3]{#1 \att (#2,#3)} % type of the run M process
\newcommand{\tyrun}[1]{#1 \att} % type of the run M process
\newcommand{\typar}[2]{#1 \mathbin{\tmkw{\vert\vert}}  #2} % type of parallel processes
\newcommand{\tyC}{C} % meta variable ranging over process types
\newcommand{\tyD}{D} % meta variable ranging over process types

%%%%% Display of source code in math mode

\newcommand{\tm}[1]{\mathsf{#1}} % the source code font

%
\definecolor{keywordColor}{cmyk}{0.9, 0.4, 0.1, 0.2} % i dont have that color otherwise
\newcommand{\tmkw}[1]{\tm{\color{keywordColor}#1}} % source code keyword, colored

\newcommand{\tmpromise}[1]{\langle #1 \rangle} % completed promise

\newcommand{\tmconst}[1]{\tm{#1}}
\newcommand{\tmunit}{()} % the element of the unit type
\newcommand{\tmpair}[2]{( #1 , #2 )} % ordered pair
\newcommand{\tminl}[2][]{\tmkw{inl}_{#1}\,#2} % left injection
\newcommand{\tminr}[2][]{\tmkw{inr}_{#1}\,#2} % right injection
\newcommand{\tmfun}[2]{{\mathop{\tmkw{fun}}}\; (#1) \mapsto #2} % function abstraction
\newcommand{\tmfunano}[2]{{\mathop{\tmkw{fun}}}\; #1 \mapsto #2} % function abstraction (no type annotation expected)
%\newcommand{\tmfun}[2]{{\mathop{\tmkw{fun}}}\; #1 \mapsto #2} % no annotation is now standard. Use annotated type if you want annotation.
\newcommand{\tmfunrec}[3]{\tmkw{fun}\; \tmkw{rec}\; #1\; (#2) \mapsto #3} % recursive abstraction
\newcommand{\tmfunrecano}[3]{\tmkw{fun}\; \tmkw{rec}\; #1\; #2 \mapsto #3} % recursive abstraction (no type annotation expected)
\newcommand{\tmapp}[2]{#1\,#2} % application

\newcommand{\tmboxed}[1]{[#1]} % boxed value
\newcommand{\tmunbox}[3]{\tmkw{unbox}\; \tmboxed{#1}\; \tmkw{as} \; #2 \; \tmkw{in}\;#3} % unbox comand

\newcommand{\tmreturn}[2][]{\tmkw{return}_{#1}\, #2} % pure computation
\newcommand{\tmlet}[3]{\tmkw{let}\; #1 = #2 \;\tmkw{in}\; #3} % let-binding
\newcommand{\tmletrec}[5][]{\tmkw{let}\;\tmkw{rec}\; #2\; #3 #1 = #4 \;\tmkw{in}\; #5} % recursive definitions

\newcommand{\tmop}[4]{\tm{#1}\;(#2, #3. #4)} % operation call
\newcommand{\tmopin}[3]{\tmkw{\downarrow}\, \tm{#1}\,(#2, #3)} % incoming interrupt
\newcommand{\tmopout}[3]{\tmkw{\uparrow}\,\tm{#1}\, (#2, #3)} % outgoing signal
\newcommand{\tmopoutbig}[3]{\tmkw{\uparrow}\,\tm{#1}\, \big(#2, #3\big)} % outgoing signal with big brackets
\newcommand{\tmopoutgen}[2]{\tmkw{\uparrow}\,\tm{#1}\, #2} % generic variant of outgoing signal

\newcommand{\tmmatch}[3][]{\tmkw{match}\;#2\;\tmkw{with}\;\{#3\}_{#1}} % match statement

\newcommand{\tmawait}[3]{\tmkw{await}\;#1\;\tmkw{until}\;\tmpromise{#2}\;\tmkw{in}\;#3} % awaiting for a promise to be completed

\newcommand{\tmwith}[5]{\tmkw{promise}\; (\tm{#1}\; #2 \mapsto #3)\; \tmkw{as}\; #4\; \tmkw{in}\; #5} % interrupt hook
\newcommand{\tmwithrec}[6]{\tmkw{promise}\; (\tm{#1}\; #2\; (#3) \mapsto (#4))\; \tmkw{as}\; #5\; \tmkw{in}\; #6} % interrupt hook
%\newcommand{\tmwith}[6]{\tmkw{promise}\; (\tm{#1}\; #2 \mapsto #3)\; \tmkw{as}\; #4 \of \typromise{#5}\; \tmkw{in}\; #6} % interrupt hook

\newcommand{\tmrun}[1]{\tmkw{run}\; #1} % running a computation as a process
\newcommand{\tmpar}[2]{#1 \mathbin{\tmkw{\vert\vert}} #2} % parallel composition of processes

%%% Operational semantics

\newcommand{\reduces}{\leadsto} % small-step reduction
\newcommand{\tyreduces}{\rightsquigarrow} % reduction of process types

\newcommand{\E}{\mathcal{E}} % evaluation context for computations
\renewcommand{\H}{\mathcal{H}} % signal hoisting context
\newcommand{\F}{\mathcal{F}} % evaluation context for processes

%%% Typing rules

\newcommand{\types}{\vdash} % typing judgement
\newcommand{\of}{\mathinner{:}} % the colon in a typing judgement

\newcommand{\sub}{\sqsubseteq} % subtyping relation

\definecolor{rulenameColor}{cmyk}{0.1, 0.1, 0.1, 0.4} % i dont have that color otherwise
\newcommand{\coopinfer}[3]{\inferrule*[Lab={\color{rulenameColor}#1}]{#2}{#3}}

%%% Meta-theory

\makeatletter
\newcommand{\hourglass}{}                  % hourglass symbol for classifying temporarity blocked computations
\DeclareRobustCommand{\hourglass}{\mathrel{\mathpalette\hour@glass\relax}}

\newcommand\hour@glass[2]{%
  \vcenter{\hbox{%
    \rotatebox[origin=c]{90}{\scalebox{0.8}{$\m@th#1\bowtie$}}%
  }}%
}
\makeatother

\newcommand{\awaiting}[2]{#1 \hourglass #2} % computations blocked on awaiting a particular promise variable to be fulfilled

\newcommand{\CompResult}[2]{\mathsf{CompRes}\langle#1 \,\vert\, #2\rangle} % top-level result forms of individual computations
\newcommand{\RunResult}[2]{\mathsf{RunRes}\langle#1 \,\vert\, #2\rangle} % local (under-signal) result forms of individual computations

\newcommand{\Result}[2]{\mathsf{Res}\langle#1 \,\vert\, #2\rangle} % top-level result forms of computations

\newcommand{\ProcResult}[1]{\mathsf{ProcRes}\langle #1 \rangle} % top-level result forms of parallel processes
\newcommand{\ParResult}[1]{\mathsf{ParRes}\langle #1 \rangle} % intermediate result forms of parallel processes

%%% Maths

\newcommand{\cond}[3]{\mathsf{if}\;#1\;\mathsf{then}\;#2\;\mathsf{else}\;#3} % single line conditional

\newcommand{\carrier}[1]{\vert #1 \vert} % carrier of a cpo
\newcommand{\order}[1]{\sqsubseteq_{#1}} % partial order of a cpo
\newcommand{\lub}[1]{\bigsqcup_n \langle #1 \rangle} % least upper bound of an omega-chain

\newcommand{\Pow}[1]{\mathcal{P}(#1)} % powerset
\newcommand{\sem}[1]{[\![#1]\!]} % semantic bracket

\makeatletter
\providecommand*{\cupdot}{%     % disjoint union of sets
  \mathbin{%
    \mathpalette\@cupdot{}%
  }%
}
\newcommand*{\@cupdot}[2]{%
  \ooalign{%
    $\m@th#1\cup$\cr
    \hidewidth$\m@th#1\cdot$\hidewidth
  }%
}
\makeatother


%%% Redex highlighting

\definecolor{redexColor}{rgb}{0.83, 0.83, 0.83} % the color of highlighted redexes
\newcommand{\highlightgray}[1]{{\setlength{\fboxsep}{1.5pt}\colorbox{redexColor}{$#1$}}} % highlight redexes with gray(ish) background
\newcommand{\highlightwhite}[1]{{\setlength{\fboxsep}{1.5pt}\colorbox{white}{$#1$}}} % highlight redexes with white background




% todo notes to comment the code when working in a group
\usepackage{todonotes}
\definecolor{jcyan}{cmyk}{1, 0, 0.15, 0.05}
\newcommand\mP[1]{\todo[inline,color=red]{#1 -MP}}	% comments by matija
\newcommand\jR[1]{\todo[inline,color=jcyan]{#1 -JR}} % comments by janez


\begin{document}
	
	\title{Asinhroni algebrajski učinki}
	\author[Janez Radešček]{Janez Radešček \\[3mm] Mentor: doc.~dr.~Pretnar~Matija\\[3mm] Somentor: asist.~raz.~dr.~Ahman~Danel}
	\institute{FMF, Univerza v Ljubljani}
	\date{\today}
	
	\frame{\titlepage}

	
	
	\begin{frame}[fragile]{Uvod}
		\begin{enumerate}
			\item Običajen program je sekvenčen. 
			%Imamo neke korake, ki jih izvajamo v določenem zaporedju enega za drugim.
			
			\begin{columns}[T]
				\begin{column}{0.2\textwidth}
					\begin{align*}
					&1\ a \gets 2  \\
					&2\ b \gets 5 \\
					&3\ c \gets b + 1 \\
					&4\ d \gets a + c \\
					&5\ return\ d 
					\end{align*}
					\vspace{0.1ex}
				\end{column}
				\begin{column}{0.5\textwidth}
				\end{column}
			\end{columns}
		
			\item Da bi pohitrili izvajanje, bi radi vzporeden program.
			%Namesto, da naenkrat izvajamo samo en korak, jih lahko več hkrati.
			%Običajno ne moremo izvesti vseh korakov hkrati.
			%Določene korake moramo izvesti kasneje kot določene druge korake, ker so med sabo odvisni. 
			%Recimo koraka 3 ne moremo izvesti dokler se ni izvedel korak 2.
			%Podobno 4
			%V splšnem ni enostavno ugotoviti kateri koraki so med sabo odvisni. 

		\end{enumerate}	
	\end{frame}


	\begin{frame}[fragile]{Uvod}
		
		Standardni pristop je zato sledeč.
		\begin{enumerate}
			\item Korake razporedimo v procese.
			\item Vsi koraki znotraj enega procesa se smatrajo za odvisne in se zato izvedejo sekvenčno.
			\item Za korake iz različnih procesov smatramo da so neodvisni, razen če jih posebej označimo.
		\end{enumerate}
	
		\begin{columns}[T]
			\begin{column}{0.15\textwidth}
				\begin{align*}
				&1\ a \gets 2  \\
				& \\
				& \\
				&4\ d \gets a + c\ \ : 3 \\
				&5\ return\ d 
				\end{align*}
				\vspace{0.1ex}
			\end{column}
			\begin{column}{0.15\textwidth}
				\begin{align*}
				& &  \\
				&               &2\ b \gets 5 \\
				&               &3\ c \gets b + 1 \\
				&& \\
				&&\\
				\end{align*}
			\end{column}
			\begin{column}{0.05\textwidth}
			\end{column}
		\end{columns}
	
		%Problem ugotoviti kateri koraki so odvisni med sabo, smo prevedli na problem delitve korakov v procese.
		%Če korake napačno razdelimo v procese lahko dobimo program z drugačnim rezultatom ali celo takega ki ima napako.
		
	\end{frame}

	\begin{frame}[fragile]{\aeff{}}

		\begin{columns}[T]
			\begin{column}{0.49\textwidth}
				\begin{lstlisting}[basicstyle=\tiny]
operation signal_c : int		

run let a = 2 in
    promise(signal_c c -> 
      return <c>)
    as p in
    await p until <c'> in
    let d = a + c' in
    return d 

run let b = 5 in
    let c = b + 1 in
    send singal_c c
				\end{lstlisting}
			\end{column}
		
		
			\begin{column}{0.49\textwidth}
				\begin{lstlisting}[basicstyle=\tiny]
operation signal_c : int		

run let a = 2 in
    promise(signal_c c -> 
      let d = a + c in
      return <d>)
    as p in
    await p until <d'> in
    return d'

run let a = 5 in
    let c = a + 1 in
    send singal_c c
				\end{lstlisting}
			\end{column}
		\end{columns}
	
		
		\vspace{5ex}
		\begin{columns}[T]
			\begin{column}{0.49\textwidth}
				%Nedelujoč primer
				\begin{lstlisting}[basicstyle=\tiny]
run let a = 2 in
    let d = a + c in
    return d 

run let b = 5 in
    let c = b + 1
				\end{lstlisting}
			\end{column}
			
			
			\begin{column}{0.49\textwidth}
				%V lae
				\begin{lstlisting}[basicstyle=\tiny]

				\end{lstlisting}
			\end{column}
		\end{columns}


		%Da se izognemo problemu kaj z 4. korakom aeff ne omogoča uporabe spremenljivk iz drugih procesov.
		%Kadar bi radi da nek korak le je odvisen od koraka iz drugega procesa to dosežemo s pomočjo efektov. 
		
	\end{frame}



	\begin{frame}{Izrazi}


		\begin{columns}[T]
			\begin{column}{0.27\textwidth}
				\begin{figure}[hp]
					\parbox{\textwidth}{
						\centering
						\tiny
						\begin{align*}
						\intertext{\textbf{Vrednosti}}
						V, W
						\bnfis& n \bnfor\! \true \bnfor\! \false        \\
						\bnfor& x                                       \\
						\bnfor& \tmunit \bnfor\! \tmpair{V}{W}          \\
						\bnfor& \tminl[Y]{V} \bnfor\! \tminr[X]{V}      \\
						\bnfor& \tmfun{x}{M}                        \\
						\bnfor& \tmfunrec{f}{x}{M}                   \\
						\bnfor& \tmpromise V                           
						\end{align*}
					} 
				\end{figure}
			\end{column}
		
			\begin{column}{0.32\textwidth}
				\begin{figure}[hp]
					\parbox{\textwidth}{
						\centering
						\tiny
						\begin{align*}
						\intertext{\textbf{Izračuni}}
						M, N
						\bnfis& \tmreturn{V}                             \\
						\bnfor& \tmlet{x}{M}{N}                          \\
						\bnfor& V\,W                                   \\
						\bnfor& \tmmatch{V}{\tmpair{x}{y} \mapsto M}    \\
						\bnfor& \tmmatch[]{V}{}                         \\
						\bnfor& \tmmatch{V}{\tminl{x} \mapsto M, \tminr{y} \mapsto N}	\\
						\bnfor& \tmopout{op}{V}{M}       \\
						\bnfor& \tmopin{op}{V}{M}          \\
						\bnfor& \tmwith{op}{x}{M}{p}{N}      \\
						\bnfor& \tmawait{V}{x}{M}           
						\end{align*}
					} 
				\end{figure}
			\end{column}
		
			\begin{column}{0.32\textwidth}
				\begin{figure}[hp]
					\parbox{\textwidth}{
						\centering
						\tiny
						\begin{align*}
						\intertext{\textbf{Procesi}}
						P, Q
						\bnfis & \tmrun M &  \\
						\bnfor & \tmpar P Q &  \\
						\bnfor & \tmopout{op}{V}{P} &  \\
						\bnfor & \tmopin{op}{V}{P}  & 
						\end{align*}
					} 
				\end{figure}
			\end{column}
		\end{columns}
		
	\end{frame}



	\begin{frame}{Standardna operacijska semantika izračunov}
		\begin{figure}[tp]
			\tiny
			\begin{align*}
			\tmapp{(\tmfun{x \of X}{M})}{V} &\reduces M[V/x]
			\\
			\tmapp{(\tmfunrec{f}{x \of X}{M})}{V} &\reduces M[V/x, (\tmfunrec{f}{x \of X}{M})/f]
			\\
			\tmlet{x}{(\tmreturn V)}{N} &\reduces N[V/x]
			\\
			\tmmatch{\tmpair{V}{W}}{\tmpair{x}{y} \mapsto M} &\reduces M[V/x, W/y]
			\\
			\mathllap{\tmmatch{(\tminl[Y]{V})}{\tminl{x} \mapsto M, \tminr{y} \mapsto N}} &\reduces	M[V/x]
			\\
			\mathllap{\tmmatch{(\tminr[X]{W})}{\tminl{x} \mapsto M, \tminr{y} \mapsto N}} &\reduces	N[W/y]
			\end{align*}
		\end{figure}
	\end{frame}


	\begin{frame}{\lae{} operacijska semantika izračunov}
		\begin{figure}[tp]
			\tiny
			\begin{align*}
			\intertext{\textbf{Algebraičnost signala in prekinitve}}
			\tmlet{x}{(\tmopout{op}{V}{M})}{N} &\reduces \tmopout{op}{V}{\tmlet{x}{M}{N}}
			\\
			\tmlet{x}{(\tmwith{op}{y}{M}{p}{N_1})}{N_2} &\reduces \tmwith{op}{y}{M}{p}{(\tmlet{x}{N_1}{N_2})}
			\\[1ex]
			\intertext{\textbf{Komutativnost signala in prestreznika}}
			\tmwith{op}{x}{M}{p}{\tmopout{op'}{V}{N}} &\reduces \tmopout{op'}{V}{\tmwith{op}{x}{M}{p}{N}}
			\\[1ex]
			\intertext{\textbf{Širitev prekinitve}}
			\tmopin{op}{V}{\tmreturn W} &\reduces \tmreturn W
			\\
			\tmopin{op}{V}{\tmopout{op'}{W}{M}} &\reduces \tmopout{op'}{W}{\tmopin{op}{V}{M}}
			\\
			\tmopin{op}{V}{\tmwith{op}{x}{M}{p}{N}} &\reduces \tmlet{p}{M[V/x]}{\tmopin{op}{V}{N}}
			\\
			\tmopin{op'}{V}{\tmwith{op}{x}{M}{p}{N}} &\reduces \tmwith{op}{x}{M}{p}{\tmopin{op'}{V}{N}}
			\quad {\color{rulenameColor}(\op \neq \op')}
			\\[1ex]
			\intertext{\quad\,\textbf{Čakanje na izpolnitev obljube}}
			\tmawait{\tmpromise V}{x}{M} &\reduces M[V/x]
			\end{align*}
		\end{figure}
	\end{frame}

	
	\begin{frame}{Evalvacija v okolju}
		\vspace{-2ex}
		\begin{minipage}[t]{\textwidth}
			\tiny
			\centering
			\begin{align*}
			\shortintertext{\textbf{Evalvacija v okolju}}
			\coopinfer{}{
				M \reduces N
			}{
				\E[M] \reduces \E[N]
			}
			\end{align*}
			\vspace{-10ex}
		\end{minipage}
		
		\begin{figure}[tp]
			\tiny
			\begin{align*}
				\shortintertext{\textbf{kjer}}
				\text{$\E$}
				\bnfis [~]
				\bnfor \tmlet{x}{\E}{N}
				\bnfor \tmopout{op}{V}{\E}
				\bnfor \tmopin{op}{V}{\E}
				\bnfor \tmwith{op}{x}{M}{p}{\E}
			\end{align*}
			\vspace{-10ex}
		\end{figure}
	\end{frame}


	\begin{frame}{\lae{} operacijska semantika procesov}
		\begin{figure}[tp]
			\parbox{\textwidth}{
				\centering
				\tiny
				\begin{minipage}[t]{0.4\textwidth}
					\centering
					\begin{align*}
					\shortintertext{\textbf{Posamezen proces}}
					\coopinfer{}{
						M \reduces N
					}{
						\tmrun M \reduces \tmrun N
					}
					\end{align*}
				\end{minipage}
				\qquad
				\begin{minipage}[t]{0.4\textwidth}
					\centering
					\begin{align*}
					\shortintertext{\textbf{Širitev prekinitve}}
					\tmopin{op}{V}{\tmrun M} &\reduces \tmrun {(\tmopin{op}{V}{M})}
					\\
					\tmopin{op}{V}{\tmpar P Q} &\reduces \tmpar {\tmopin{op}{V}{P}} {\tmopin{op}{V}{Q}}
					\\
					\tmopin{op}{V}{\tmopout{op'}{W}{P}} &\reduces \tmopout{op'}{W}{\tmopin{op}{V}{P}}
					\end{align*}
				\end{minipage}
			
				
				
				%%%%
				
				\begin{minipage}[t]{0.4\textwidth}
					\centering
					\begin{align*}
					\shortintertext{\textbf{Dvig signala}}
					\tmrun {(\tmopout{op}{V}{M})}  &\reduces \tmopout{op}{V}{\tmrun M}
					\end{align*}
				\end{minipage}
				\qquad
				\begin{minipage}[t]{0.4\textwidth}
					\centering
					\begin{align*}
					\shortintertext{\textbf{Oddajanje signala}}
					\tmpar{\tmopout{op}{V}{P}}{Q} &\reduces \tmopout{op}{V}{\tmpar{P}{\tmopin{op}{V}{Q}}}
					\\
					\tmpar{P}{\tmopout{op}{V}{Q}} &\reduces \tmopout{op}{V}{\tmpar{\tmopin{op}{V}{P}}{Q}}
					\end{align*}
				\end{minipage}
				
				%%%%
		
				\begin{align*}
				\shortintertext{\quad\textbf{Vrednotenje v kontekstu.}}
				\quad
				\coopinfer{}{
					P \reduces Q
				}{
					\F[P] \reduces \F[Q]
				}
				\end{align*}
				\vspace{-8ex}
				\begin{align*}
				\shortintertext{\textbf{kjer}\vspace{1ex}}
				\text{$\F$}
				\bnfis& [~]
				\bnfor \tmpar \F Q \bnfor\! \tmpar P \F
				\bnfor \tmopout{op}{V}{\F}
				\bnfor \tmopin{op}{V}{\F}
				\end{align*}
			} 
		\end{figure}
	\end{frame}



	\begin{frame}{Tipi in operacije}
		\begin{figure}[tb]
			\parbox{\textwidth}{
				\centering
				\tiny
				\begin{align*}
				\text{Osnovni tipi vrednosti $\bar{A}$, $\bar{B}$}
				\bnfis & \tysym{int} \,\bnfor\! \tysym{bool} \,\bnfor\! \tyunit \,\bnfor\! \tyempty \, 
				          \bnfor\! \typrod{\bar{A}}{\bar{B}} \,\bnfor\! \tysum{\bar{A}}{\bar{B}}
				\\
				\text{Tipi vrednosti $A$, $B$}
				\bnfis & \bar{A} \, \bnfor\! \typrod{A}{B} \,\bnfor\! \tysum{A}{B} \,\bnfor\! \tyfun{A}{B} \,\bnfor\! \typromise{A}
				\\
				\text{Tip izračuna} \bnfis& \tycomp{A}
				\\
				\text{Tip procesa \tyC, \tyD}  \bnfis & \tyrun{A} \,\bnfor\! \typar{\tyC}{\tyD}
				\end{align*}
			} 
		\end{figure}
	
		\begin{figure}
			\centering
			\tiny
			\begin{align*}
			\intertext{Operacije}
			(op_1, \bar{A}_{op_1}),\, (op_2, \bar{A}_{op_2}),\, ... ,\, (op_n, \bar{A}_{op_k})
			\end{align*}
			\vspace{-15ex}
		\end{figure}
	\end{frame}

	\begin{frame}{Pravila za tipe vrednosti}
		\begin{figure}[tp]
			\centering
			\tiny
			\begin{mathpar}
				\coopinfer{}{
				}{
					\Gamma \types n : \tysym{int}
				}
				\qquad
				\coopinfer{}{
				}{
					\Gamma \types \true : \tysym{bool}
				}
				\qquad
				\coopinfer{}{
				}{
					\Gamma \types \false : \tysym{bool}
				}
				\quad
				\coopinfer{}{
				}{
					\Gamma, x \of A, \Gamma' \types x : A
				}
				\quad
				\coopinfer{}{
				}{
					\Gamma \types \tmunit : \tyunit
				}
				\\
				\coopinfer{}{
					\Gamma \types V : A \\
					\Gamma \types W : B
				}{
					\Gamma \types \tmpair{V}{W} : \typrod{A}{B}
				}
				\quad
				\coopinfer{}{
					\Gamma \types V : A
				}{
					\Gamma \types \tmpromise V : \typromise A
				}
				\quad
				\coopinfer{}{
					\Gamma \types V : A
				}{
					\Gamma \types \tminl[B]{V} : A + B
				}
				\quad
				\coopinfer{}{
					\Gamma \types W : B
				}{
					\Gamma \types \tminr[A]{W} : A + B
				}
				\\
				\coopinfer{}{
					\Gamma, x \of A \types M : B
				}{
					\Gamma \types \tmfun{x : A}{M} : \tyfun{A}{B}
				}
				\quad
				\coopinfer{}{
					\Gamma, x \of A \types M : B
				}{
					\Gamma \types \tmfunrec{f}{x : A}{M} : \tyfun{A}{B}
				}
			\end{mathpar}
		\end{figure}
	\end{frame}

	\begin{frame}{Pravila za tipe računov}
		\begin{figure}[tp]
			\centering
			\tiny
			\begin{mathpar}
				\coopinfer{}{
					\Gamma \types V : X
				}{
					\Gamma \types \tmreturn{V} : X
				}
				\qquad
				\coopinfer{}{
					\Gamma \types M : X
					\\
					\Gamma, x \of X \types N : Y
				}{
					\Gamma \types
					\tmlet{x}{M}{N} : Y
				}
				\qquad
				\coopinfer{}{
					\Gamma \types V : \tyfun{X}{Y} \\
					\Gamma \types W : X
				}{
					\Gamma \types \tmapp{V}{W} : Y
				}
				\\
				\coopinfer{}{
					\Gamma \types V : \typrod{X}{Y} \\
					\Gamma, x \of X, y \of Y \types M : Z
				}{
					\Gamma \types \tmmatch{V}{\tmpair{x}{y} \mapsto M} : Z
				}
				\qquad
				\coopinfer{}{
					\Gamma \types V : \tyempty
				}{
					\Gamma \types \tmmatch[Z]{V}{} : Z
				}
				\\
				\coopinfer{}{
					\Gamma \types V : X + Y \\\\
					\Gamma, x \of X \types M : Z \\
					\Gamma, y \of Y \types N : Z \\
				}{
					\Gamma \types \tmmatch{V}{\tminl{x} \mapsto M, \tminr{y} \mapsto N} : Z
				}
				\\
				\coopinfer{}{
					\Gamma \types V : A_\op \\
					\Gamma \types M : X 
				}{
					\Gamma \types \tmopout{op}{V}{M} : X
				}
				\qquad
				\coopinfer{}{
					\Gamma \types V : A_\op \\
					\Gamma \types M : X 
				}{
					\Gamma \types \tmopin{op}{V}{M} : X
				}
				\\
				\coopinfer{}{
					\Gamma, x \of A_\op \types M : \typromise X \\
					\Gamma, p \of \typromise X \types N : Y 
				}{
					\Gamma \types \tmwith{op}{x}{M}{p}{N} : Y
				}
				\\
				\coopinfer{}{
					\Gamma \types V : \typromise X \\
					\Gamma, x \of X \types M : Y 
				}{
					\Gamma \types \tmawait{V}{x}{M} : Y
				}
			\end{mathpar}
		\end{figure}
	\end{frame}


	\begin{frame}{Pravila za tipe procesov}
		\begin{figure}[tp]
			\centering
			\small
			\begin{mathpar}
				\coopinfer{}{
					\Gamma \types M : \tycomp{A}
				}{
					\Gamma \types \tmrun{M} : \tyrun{A}
				}
				\quad
				\coopinfer{}{
					\Gamma \types P : \tyrun{C} \\
					\Gamma \types Q : \tyrun{D}
				}{
					\Gamma \types \tmpar{P}{Q} : \typar{\tyC}{\tyD}
				}
				\\
				\coopinfer{}{
					\Gamma \types V : A_\op \\
					\Gamma \types P : \tyrun{C} 
				}{
					\Gamma \types \tmopout{op}{V}{P} : \tyrun{C}
				}
				\quad
				\coopinfer{}{
					\Gamma \types V : A_\op \\
					\Gamma \types P : \tyrun{C} 
				}{
					\Gamma \types \tmopin{op}{V}{P} : \tyrun{C}
				}  
			\end{mathpar}
		\end{figure}
	\end{frame}


	\begin{frame}{Rezultati}
			
		%Običajno bi bil izračun return V rezultat. 
		%Izračun await blokira dokler ustrezen prestreznik ne izpolni pripadajoče obljube. Ni nujno da bo vsak prestreznik ipolnil obljubo. Zato rezultatom dodamo izračune, ki čakajo izpolnitev obljube.
		%Izračun je lahko v postopku izpolnitve obljube na katero čakamo. Takih izračunov ne damo med rezultatov.
		
		\begin{figure}
			\centering
			\tiny
			\textbf{Čakajoči izrazi}
			\begin{mathpar}
				\coopinfer{}{
				}{
					\awaiting p {\tmawait p x M}
				}
				\quad
				\coopinfer{}{
					\awaiting p M
				}{
					\awaiting p {\tmlet x M N}
				}
				\quad
				\coopinfer{}{
					\awaiting p M
				}{
					\awaiting p {\tmopin{op}{V}{M}}
				}
			\end{mathpar}
		
			\textbf{Delni rezultati}
			\begin{mathpar}
				\coopinfer{}{
					\CompResult {\Psi} {M}
				}{
					\CompResult {\Psi} {\tmopout {op} V M}
				}
				\quad
				\coopinfer{}{
					\RunResult {\Psi} {M}
				}{
					\CompResult {\Psi} {M}
				}
				\vspace{-1ex}
				\\
				\coopinfer{}{
				}{
					\RunResult {\Psi} {\tmreturn V}
				}
				\quad
				\coopinfer{}{
					\RunResult {\Psi \cup \{p\}} {N}
				}{
					\RunResult {\Psi} {\tmwith {op} x M p N}
				}
				\quad
				\coopinfer{}{
					p \in \Psi \\
					\awaiting p M
				}{
					\RunResult {\Psi} {M}
				}
			\end{mathpar}
		
			%Proces je rezultat če so vsi signali prišli do korenin drevesa procesov in so vsi procesi
		
			\textbf{Rezultati}
			\begin{mathpar}
				\coopinfer{}{
					\ProcResult {P}
				}{
					\ProcResult {\tmopout {op} V P}
				}
				\qquad
				\coopinfer{}{
					\ParResult {P}
				}{
					\ProcResult {P}
				}
				\qquad
				\coopinfer{}{
					\RunResult {\emptyset} {M}
				}{
					\ParResult {\tmrun M}
				}
				\qquad
				\coopinfer{}{
					\ParResult P \\
					\ParResult Q
				}{
					\ParResult {\tmpar P Q}
				}
			\end{mathpar}
		
		\end{figure}
		
	\end{frame}




	

	
	\begin{frame}{Napredek}
		\begin{lema}[1]
			Če ima izraz $M$ v okolju $\Gamma$ tip $A$, je $M$ delni rezultat ali pa lahko naredi korak v izraz $M'$.
		\end{lema}
		
		%glede na čas dokaz tudi leme, vsaj deloma.
		
					
		\begin{izrek}[o napredku]
			Če ima izraz $M$ v okolju $\Gamma$ tip $A$, je $M$ rezultat ali pa lahko naredi korak v izraz $M'$.
		\end{izrek}
		
		\begin{proof}
			... 
			%Predpostavimo da M ni rezultat. Hočemo dokazati da obstaja M* v katerega lahko naredimo korak. Sedaj dokazujemo z indukcijo na globino izpeljave tipa $A$. Recimo da smo na globini 1. Edina možnost, da smo dokazali A, je da smo uporabili pravilo tip run. Se pravi M'' ima tip A. Po lemmi 1 je M'' rezultat ali pa lahko naredi korak. Če bi bil M'' rezultat bi bil po pravilu rezultat run tudi M rezultat. Se pravi da M'' lahko naredi korak. Po pravilu za posamezen proces pa to pomeni da tudi M lahko naredi korak. Naj bo globina sedaj n>1. Tedaj smo A dokazali z pravilom tip vzporedna, tip proces signal ali tip proces prekinitev. Recimo da smo dokazali z pravilom tip vzporedna. A = (C || D). Če sta P in Q rezultata bi bil po pravilu rezultat vzporedna tudi M rezultat. Se pravi da eden izmed P ali Q ni rezultat. Po indukcijski predpostavki lahko tisti, ki ni rezultat naredi korak. Po pravilu za vrednotenje v kontekstu to pomeni da M lahko naredi korak. Podobno za  tip proces signal in tip proces prekinitev.
		\end{proof}
		%Kako lepo definirati globino iz dokaza??
		
	\end{frame}

	
		
	\begin{frame}{Ohranitev}
				
		\begin{izrek}[o ohranitvi]
			Če ima izraz $M$ v okolju $\Gamma$ tip $A$ in naredi korak v izraz $M^*$, ima $M^*$ v okolju $\Gamma$ tip $A$.
		\end{izrek}
		
		\begin{proof}
			%Left for the reader as an exercise :D
		\end{proof}
		
	\end{frame}



	\begin{frame}[fragile]{Primer}
		\begin{columns}[T]
			\begin{column}{0.49\textwidth}
				\centering
				\aeff
				\begin{lstlisting}[basicstyle=\tiny]
operation signal_c : int		

run let a = 2 in
    promise(signal_c c -> return <c>)
    as p in
    await p until <c'> in
    let d = a + c' in
    return d 

run let b = 5 in
    let c = b + 1 in
    send singal_c c
				\end{lstlisting}
			\end{column}
			
			%prevedi primer v lae, izračunaj tip in nato izračun vrednost
			%Preskoči če bo zmanjkovalo časa.
			\begin{column}{0.49\textwidth}
				\centering
				\lae
				\begin{lstlisting}
          ...
				\end{lstlisting}
			\end{column}
		\end{columns}
	\end{frame}



	\begin{frame}{Rekurzivne obljube}
		\begin{figure}[hp]
			\parbox{\textwidth}{
				\centering
				\tiny
				\begin{align*}
				\shortintertext{\textbf{Izračuni}}
				M, N
				\bnfis& ... \,\bnfor \tmwithrec{op}{f}{x \of A}{M \of \typromise{B}}{p}{N}                   
				\end{align*}
			} 
		\end{figure}
	
		\begin{figure}[tp]
			\centering
			\tiny
			\begin{align*}
			\shortintertext{\textbf{Operacijska semantika}}
			\tmopin{op}{V}{\tmwith{op}{x}{M}{p}{N}} &\reduces \tmlet{p}{M[V/x]}{\tmopin{op}{V}{N}}
			\\
			\tmopin{op'}{V}{\tmwith{op}{x}{M}{p}{N}} &\reduces \tmwith{op}{x}{M}{p}{\tmopin{op'}{V}{N}}
			\quad {\color{rulenameColor}(\op \neq \op')}
			\end{align*}
		\end{figure}
		
		\begin{figure}[tp]
			\centering
			\tiny
			\textbf{Pravila za tipe računov}
			\begin{mathpar}
				\coopinfer{}{
					\Gamma, f \of \tyfun{A_\op}{\typromise{B}}, x \of A_\op \types M : \typromise B \\
					\Gamma, p \of \typromise B \types N : C 
				}{
					\Gamma \types \tmwithrec{op}{f}{x : A}{M : \typromise{B}}{p}{N} : C
				}
	
			\end{mathpar}
		\end{figure}
		
	\end{frame}




	\begin{frame}{Prenosljivi tipi}

		\begin{figure}[hp]
			\parbox{\textwidth}{
				\centering
				\tiny
				\begin{align*}
				\shortintertext{\textbf{Vrednosti}}
				V, W
				\bnfis & ... \,\bnfor\! \tmboxed{V}     
				\\[5ex]
				\shortintertext{\textbf{Izračuni}}
				M, N
				\bnfis & ... \,\bnfor\! \tmunbox{V}{x}{M}
				\end{align*}
			} 
		\end{figure}
		
		\begin{figure}[tp]
			\centering
			\tiny
			\begin{align*}
			\shortintertext{\textbf{Operacijska semantika}}
			\tmunbox{V}{u}{M} & \reduces M[V/u]
			\end{align*}
		\end{figure}
	
		\begin{figure}[tb]
			\parbox{\textwidth}{
				\centering
				\tiny
				\begin{align*}
				\text{Prenosljivi tipi vrednosti $\bar{A}$, $\bar{B}$}
				\bnfis & ... \,\bnfor\! \tyboxed{A}
				\end{align*}
			} 
		\end{figure}
	
		
		\begin{figure}[tp]
			\centering
			\tiny
			\textbf{Pravila za tipe vrednosti}
			\begin{mathpar}
				\coopinfer{}{\text{A prenosljiv ali } \blacksquare \notin \Gamma'
				}{
					\Gamma, x \of A, \Gamma' \types x : A
				}
				\coopinfer{}{\Gamma, \blacksquare \types V : A
				}{
					\Gamma \types \tmboxed{V} : \tyboxed{A}
				}
			\end{mathpar}
		\end{figure}
		
	\end{frame}


	\begin{frame}{Spawn}
		
		\begin{figure}[hp]
			\parbox{\textwidth}{
				\centering
				\tiny
				\begin{align*}
				\shortintertext{\textbf{Izračuni}}
				M, N
				\bnfis & ... \,\bnfor\! \tmspawn{M}{N}
				\end{align*}
			} 
		\end{figure}
		
		\begin{figure}[tp]
			\centering
			\tiny
			\begin{align*}
			\shortintertext{\textbf{Operacijska semantika}}
			\tmrun{(\tmspawn{M}{N})} & \reduces \tmpar{\tmrun{M}}{\tmrun{N}}
			\end{align*}
		\end{figure}
	
	
		\begin{figure}[tp]
			\centering
			\tiny
			\textbf{Pravila za tipe računov}
			\begin{mathpar}
				\coopinfer{}{\Gamma, \blacksquare \types M : B \\ \Gamma \types N : A
				}{
					\Gamma \types \tmspawn{M}{N} : A
				}
			\end{mathpar}
		\end{figure}
		
	\end{frame}
	
	
	
	
	\begin{frame}{Implementacija}
		\begin{enumerate}
			\item dvosmerni sistem tipov
			%Je kakšen boljši razlog za uvedbo bidirectional type sistem v trenutni verziji kot malo boljši izpis napake??
			\item operacijska semantika
		\end{enumerate}
	\end{frame}
	
	
	
	
	\begin{frame}{Literatura}
		\phantom{\cite{aeff}}
				
		\bibliographystyle{fmf-sl}
		\bibliography{literatura.bib}
	\end{frame}
	
	
\end{document}