\section{Implementacija}\label{sec:impl}


Kot sintaktični sladkor dodamo tudi varovan prestreznik, ki se sproži le kadar je poleg prave operacije tudi pravi pripadajoči tovor.
\begin{figure}[h]
	\centering
	\small
	\begin{align*}
	\intertext{\textbf{Sintaksa}}
	M, N
	\bnfis& ...                            & &\text{obstoječi izračuni} \\
	\bnfor& \tmwithrecgu{op}{x}{f}{P(x)}{M}{p}{N}  & &\text{varovan rekurzivni prestreznik}
	\end{align*}
	
	\begin{align*}
	\intertext{\textbf{Semantika}}
	\tmopin{op}{V}{\tmwithrecgu{op}{f}{x}{P}{M}{p}{N}} \\ \reduces \tmlet{p}{(\ite{P[V/x]}{M[V/x, F/f]}{F\,()})}{\tmopin{op}{V}{N}} \\
	{\color{rulenameColor}F = \tmfunano{y}{(\tmwithrec{op}{f}{x}{M}{p}{\tmreturn{p}})}}
	\end{align*}
	
	\caption{Varovan rekurzivni prestreznik}
	\label{fig:izrazi-prestreznik}
\end{figure}

\subsection{Sistem tipov}\label{sec:tipi}

\subsubsection{Hindley–Milner}

\subsubsection{Bidirectional}


\subsection{Tolmač}\label{sec:interpreter}

\subsubsection{prej}

\subsubsection{potem}

Pokazat da, če naredimo korak v nek izračun, obstaja zaporedje korakov v prejšnji, ki vodijo v isti izračun.

Pokazati da potrebujemo manj korakov