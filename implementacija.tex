\section{Implementacija}\label{sec:implementacija}

Medtem ko je cilj \lae-računa, da pokaže pravilnost asinhronih konceptov prikazanih v prejšnjih poglavjih, je cilj jezika \aeff{}, da te koncepte lahko tudi preizkusimo in uporabimo. Zato ima jezik \aeff{} določene stvari, ki jih \lae-račun nima.

Jezik \aeff{} je implementiran v jeziku Ocaml. Implementacija je sestavljena iz treh glavnih delov.
Parserja, ki prevede tekst v abstraktno sintaktično drevo (AST) in preveri ustreznost sintakse.  
Pregledovalnika tipov, ki preveri, da ima dani program tip.
In tolmača, ki izvaja korake, dokler ne pride do rezultata.


Da \aeff{} postane bolj uporaben so mu dodane konstantni vrednosti $\true$ in $\false$, funkcije kot so $+, -, *, / ...$, reference in globalne spremenljivke.

Izračun $\uparrow\hspace{-1ex}\op (V);M$ je sintaktičen sladkor za $\tmlet{\_}{\tmopout{op}{V}{return ()}}{M}$.
Podobno je $\tmkw{spawn}(M);N$ sintaktičen sladkor za $\tmlet{\_}{\tmspawn{M}{return ()}}{N}$. 
%Izračun $M;N$ je sintaktičen sladkor za $\tmlet{\_}{M}{N}$, ki ga lahko uporabimo kadar rezultat izračuna $M$ ni uporabljen v $N$.

Varovan prestreznik $$\tmwithrecgu{op}{x}{r}{P(x)}{M}{p}{N},$$ je sintaktičen sladkor za $$\tmwithrec{op}{x}{r}{\tmif{P(x)}{M}{r()}}{p}{N}.$$
Nanj lahko gledamo kot na prestreznik, ki se sproži, le kadar je poleg pravega imena operacije tudi pravi tovor.

Dodamo tudi seznam. Za prazen seznam uporabimo konstruktor []. Elementa mu lahko dodajamo z $::$, recimo 1::2::3::[].
Sedaj imamo problem, ker uporabljamo [] tako za sezname kot za zavite vrednosti.
V poglavju~\ref{sec:razsirjen-lae}, kjer v dveh primerih uporabimo seznam, nadomestimo konstruktor za prazen seznam z $\mathsf{nil}$ in ohranimo [] za zavite vrednosti, ki so bistvene v tem delu.
Nasprotno v dejanski implementaciji \aeff{} raje ohranimo [] za prazen seznam in spremenimo sintakso zavite vrednosti v $[|V|]$. 

Prvotna implementacija jezika \aeff{} izračuna tip programa z Hindley–Milnerjevim algoritmom. 
Kasneje je bil typechecker spremenjen, da je uporabljal dvosmerni sistem tipov (bidirectional typing), ki temelji na algoritmu iz članka~\cite{bidirectional}.
Glavna ideja dvosmernega sistema tipov je, da lahko dinamično izmenjuje med preverjanjem in računanjem tipa. Kadar so tipi pripisani izrazom, tip le preverimo, kar je enostavneje, kot ga izračunati in v primeru napake lahko izpiše boljše poročilo o napaki. Vendar pa pripisovanje tipov ni obvezno in ga v tem primeru lahko izračunamo. Posledično lahko uporabnik, poda tip le nekaterim delom programa, preostalim pa ne.
Tako prva kot druga verzija ne preverjata tipe učinkov, ampak le standardni tip.

Čeprav ima \aeff{} možnost vzporednega izvajanja in smo to do sedaj v primerih programov predpostavljali, sama implementacije zaenkrat le simulira vzporedno izvajanje.

Dodana je kratka standardna knjižnica v kateri lahko najdemo osnovne funkcije kot so $\mathsf{nth}$, $\mathsf{map}$, $\mathsf{filter}$, $\mathsf{min}$, $\mathsf{max}$, $\mathsf{length}$, $\mathsf{fold\_left}$, $\mathsf{fold\_right}$...