\section{Implementacija}\label{sec:impl}

Jezik \aeff{} je implementiran v jeziku Ocaml. Implementacija je sestavljena iz treh glavnih delov.
Parserja, ki prevede tekst v abstraktno sintaktično drevo (AST) in preveri ustreznost sintakse.  
Type checkerja, ki preveri, da ima dani program tip.
Tolmača, ki izvaja korake, dokler ne pridemo do rezultata.

\subsection{Dodatni izrazi}

Da \aeff{} postane bolj uporaben so mu dodane konstantni vrednosti $\true$ in $\false$, funkcije kot so $+, -, *, / ...$, reference in globalne spremenljivke.

Izračun $\tmkw{if}\ V\ \tmkw{then}\ M\ \tmkw{else}\ N$ je sintaktičen sladkor za $\tmmatch{V}{\true \mapsto M, \false \mapsto N}$.

Izračun $M;N$ je sintaktičen sladkor za $\tmlet{\_}{M}{N}$, ki ga lahko uporabimo kadar rezultat izračuna $M$ ni uporabljen v $N$.

Kot sintaktični sladkor je dodan varovan prestreznik, ki se sproži le kadar je poleg prave operacije tudi pravi pripadajoči tovor.
\begin{figure}[h]
	\centering
	\small
	\begin{align*}
	\intertext{\textbf{Sintaksa}}
	M, N
	\bnfis& ...                            & &\text{obstoječi izračuni} \\
	\bnfor& \tmwithrecgu{op}{x}{r}{P(x)}{M}{p}{N}  & &\text{varovan rekurzivni prestreznik}
	\end{align*}
	
	\begin{align*}
	\intertext{\textbf{Semantika}}
	\tmopin{op}{V}{\tmwithrecgu{op}{x}{r}{P}{M}{p}{N}} \\ \reduces \tmlet{p}{(\ite{P[V/x]}{M[V/x, R/r]}{R\,()})}{\tmopin{op}{V}{N}} \\
	{\color{rulenameColor}R = \tmfunano{y}{(\tmwithrecgu{op}{x}{r}{P}{M}{p}{\tmreturn{p}})}}
	\end{align*}
	
	\caption{Varovan rekurzivni prestreznik}
	\label{fig:izrazi-varovan-prestreznik}
\end{figure}

\subsection{Sistem tipov}\label{sec:tipi}

Prvotna implementacija izračuna tip programa z Hindley–Milnerjevim algoritmom. 
Kasneje je bil typechecker spremenjen, da je uporabljal dvosmerni sistem tipov (bidirectional typing), ki temelji na algoritmu iz članka~\cite{bidirectional}.
Glavna ideja dvosmernega sistema tipov je da lahko izmenjujemo ali tip preverjamo ali ga izračunamo. Kadar so tipi pripisani izrazom, tip le preverimo. V nasprotnem primeru ga lahko izračunamo. Ena izmed prednosti dvosmernega algoritma je, da v primeru napake lahko izpiše boljše poročilo o napaki. 

\subsection{Tolmač}\label{sec:interpreter}

Čeprav ima \aeff{} možnost vzporednega izvajanja in smo to do sedaj predpostavljali, sama implementacije le simulira vzporedno izvajanje.
