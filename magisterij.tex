% !TeX spellcheck = sl_SI
% vim: set spell spelllang=sl:
% za preverjanje črkovanja, če se uporablja Texstudio ali vim
\documentclass[12pt,a4paper,twoside]{article}
\usepackage[utf8]{inputenc}  % pravilno razpoznavanje unicode znakov

% NASLEDNJE UKAZE USTREZNO POPRAVI
\newcommand{\program}{Matematika} % ime studijskega programa
\newcommand{\imeavtorja}{Janez~Radešček} % ime avtorja
\newcommand{\imementorja}{doc.~dr.~Pretnar~Matija} % akademski naziv in ime mentorja, uporabi poln naziv, prof.~dr.~, doc.~dr., ali izr.~prof.~dr.
\newcommand{\imesomentorja}{asist.~raz.~dr.~Danel~Ahman} % akademski naziv in ime somentorja, če ga imate
\newcommand{\naslovdela}{Asinhroni algebrajski učinki}
\newcommand{\letnica}{2021} % letnica magistriranja
\newcommand{\opis}{Delo obravnava asinhrono izvajanje v navezi z algebrajskimi učinki.}  % Opis dela v eni povedi. Ne sme vsebovati matematičnih simbolov v $ $.
\newcommand{\kljucnebesede}{asinhrono izvajanje\sep algebrajski učinki\sep vzporedno izvajanje\sep prestrezniki\sep signali} % ključne besede, ločene z \sep, da se PDF metapodatki prav procesirajo
\newcommand{\keywords}{asynchronous execution\sep algebraic effects\sep parallel execution\sep handlers\sep signals} % ključne besede v angleščini
\newcommand{\organization}{Univerza v Ljubljani, Fakulteta za matematiko in fiziko} % fakulteta
\newcommand{\literatura}{literatura}  % pot do datoteke z literaturo (brez .bib končnice)
\newcommand{\sep}{, }  % separator med ključnimi besedami v besedilu
% KONEC PODATKOV

\usepackage{bibentry}         % za navajanje literature v programu dela s celim imenom
\nobibliography{\literatura}
\newcommand{\plancite}[1]{\item[\cite{#1}] \bibentry{#1}} % citiranje v programu dela

\usepackage{filecontents}  % za pisanje datoteke s PDF metapodatki
\usepackage{silence} \WarningFilter{latex}{Overwriting file}  % odstrani annoying warning o obstoju datoteke
% datoteka s PDF metapodatki, zgenerira se kot magisterij.xmpdata
\begin{filecontents*}{\jobname.xmpdata}
  \Title{\naslovdela}
  \Author{\imeavtorja}
  \Keywords{\kljucnebesede}
  \Subject{matematika}
  \Org{\organization}
\end{filecontents*}

\usepackage[a-1b]{pdfx}  % zgenerira PDF v tem PDF/A-1b formatu, kot zahteva knjižnica
\hypersetup{bookmarksopen, bookmarksdepth=3, colorlinks=true,
  linkcolor=black, anchorcolor=black, citecolor=black, filecolor=black,
  menucolor=black, runcolor=black, urlcolor=black, pdfencoding=auto,
  breaklinks=true, psdextra}

\usepackage[slovene]{babel}  % slovenščina
\usepackage[T1]{fontenc}     % naprednejše kodiranje fonta
\usepackage{amsmath,amssymb,amsfonts,amsthm} % matematični paketi
\usepackage{graphicx}     % za slike
\usepackage{emptypage}    % prazne strani so neoštevilčene, ampak so štete
\usepackage{makeidx}      % za stvarno kazalo, lahko zakomentiraš, če ne rabiš
\makeindex                % za stvarno kazalo, lahko zakomentiraš, če ne rabiš
% oblika strani
\usepackage[
  top=3cm,
  bottom=3cm,
  inner=3.5cm,      % margini za dvostransko tiskanje
  outer=2.5cm,
  footskip=40pt     % pozicija številke strani
]{geometry}



% DRUGI TVOJI PAKETI:

\usepackage{mathpartir} % inference rules
\usepackage{mathtools} % mathllap

\usepackage{listings}
\lstset{
	escapeinside={(*@}{@*)},          % if you want to add LaTeX within your code
}

\renewcommand{\lstlistingname}{Program}% Listing -> Program
\renewcommand{\lstlistlistingname}{Kazalo Programov}% List of Listings -> Kazalo Programov

\usepackage{float}

\usepackage{xcolor} % higlight important part of the proof.

\usepackage{relsize} % slightly decrease font size to remove black blocks


% !TEX root = paper.tex

% Any macro that is actually used should have a comment explaining what it is for.
% Please fight macro pollution and remove the macros that are not used.

\newcommand{\defeq}{\mathrel{\overset{\text{\tiny def}}{=}}} % Definitional equality

\newcommand{\defiff}{\mathrel{\overset{\text{\tiny def}}{\Longleftrightarrow}}} % Definitional iff

\newcommand{\pl}[1]{\textsc{#1}} % the name of a programming language

\newcommand{\lambdaAEff}{$\lambda_{\text{\ae}}$} % the name of the calculus

\newcommand{\lae}[0]{$\lambda_{\text{\ae}}$} %shorter name
\newcommand{\aeff}[0]{Æff} % the name of the language

\newcommand{\rulename}[1]{\textsc{#1}} % For typing/results rules 

% BNF grammars
\newcommand{\bnfis}{\mathrel{\;{:}{:}{=}\ }}
\newcommand{\bnfor}{\mathrel{\;\big|\ \ }}

%%%%% Semantic concepts

%%% Sets

\newcommand{\One}{\mathbb{1}} % singleton set as denotation of unit type
\newcommand{\one}{\star} % canonical element of the singleton set
\newcommand{\Zero}{\mathbb{0}} % empty set as denotation of empty type

\newcommand{\Bool}{\mathbb{B}} % two-element set of booleans
\newcommand{\true}{\mathbf{true}} % constant true
\newcommand{\false}{\mathbf{false}} % constant false

\newcommand{\expto}{\Rightarrow} % set exponentiation
\newcommand{\lam}[1]{\lambda #1 \,.\,} % lambda abstraction
\newcommand{\pair}[2]{\langle #1 , #2 \rangle} % pairing

\newcommand{\lifted}[1]{#1_\bot} % lifting monad
\newcommand{\idte}[4]{\mathbf{ifdef}~#1~\mathbf{then}~#2 \mapsto #3~\mathbf{else}~#4} % test if element of a lifted set is defined (non-bottom) or not, and then use it in the then branch

\newcommand{\ite}[3]{\mathbf{if}~#1~\mathbf{then}~#2~\mathbf{else}~#3} % if-then-else used in semantic definitions


%%% Signatures

\newcommand{\Tree}[2]{\mathrm{Tree}_{#1}\left(#2\right)} % The tree algebra for an operation signature
\newcommand{\retTree}[1]{\mathsf{return}\,#1} % the inclusion of generators into trees

\newcommand{\opsym}[1]{\mathsf{#1}} % a custom operation symbol
\newcommand{\op}{\opsym{op}} % a generic operation symbol

\newcommand{\sig}{\Sigma} % the global signature of signal and interrupt names

\renewcommand{\o}{o} % effect annotation describing possible outgoing operations
\renewcommand{\i}{\iota} % effect annotation describing possible incoming operations

\newcommand{\opincomp}[2]{{\mathsf{#1}}\,{\tmkw{\downarrow}}\,#2} % action of incoming interrupt on computation types
\newcommand{\opincompp}[2]{{\mathsf{#1}}\,{\tmkw{\downarrow\downarrow}}\,#2} % action of a list of incoming interrupts on computation types

\newcommand{\bb}[0]{\blacksquare}   %black box needed for mobile types


%%% Theories
\newcommand{\eq}{\mathrm{Eq}} % a set of equations

\newcommand{\FreeAlg}[2]{\mathrm{Free}_{#1}\left(#2\right)} % Free algebra for a signature generated by a set
\newcommand{\lift}[1]{#1^\dagger} % the Kleisli lifting of a map
\newcommand{\freelift}[1]{#1^\ddagger} % the lifting of a map induced by the free model property

\newcommand{\M}{\mathcal{M}} % a generic model for a theory
\newcommand{\Mcarrier}{\vert \mathcal{M} \vert} % the carrier of a generic model

\newcommand{\T}{T} % A generic monad


%%% Example effect theories

\newcommand{\sigget}{\mathsf{get}}
\newcommand{\sigset}{\mathsf{set}}


%%%%% Types

\newcommand{\at}{\mathbin{!}} % the ! sign, with proper spacing
\newcommand{\att}{\mathbin{!!}} % the !! sign, with proper spacing

%% Value types

\newcommand{\tysym}[1]{\mathsf{#1}}
\newcommand{\tybase}{\tysym{b}} % a base type
\newcommand{\tyunit}{\tysym{1}} % the unit ground type
\newcommand{\tyint}{\tysym{int}} % the integer ground type
\newcommand{\tystring}{\tysym{string}} % the integer ground type
\newcommand{\tylist}[1]{\tysym{list}~\tysym{#1}} % the list ground type
\newcommand{\tyempty}{\tysym{0}} % the empty ground type
\newcommand{\typrod}[2]{#1 \times #2} % product type
\newcommand{\tysum}[2]{#1 + #2} % sum type
\newcommand{\tyfun}[2]{#1 \to #2} % user function type
\newcommand{\typromise}[1]{\langle #1 \rangle} % type of promises
\newcommand{\tyboxed}[1]{[#1]} % type of boxed value
\newcommand{\tymobile}[1]{\widetilde{#1}} % type of boxed value

%% Computation types

\newcommand{\tycomp}[2]{#1 \at (#2)} % computation type

%% Process types

\newcommand{\tyrun}[2]{#1 \att (#2)} % type of the run M process
\newcommand{\typar}[2]{(#1 \mathbin{\tmkw{\vert\vert}}  #2)} % type of parallel processes
\newcommand{\tyC}{C} % meta variable ranging over process types
\newcommand{\tyD}{D} % meta variable ranging over process types

%%%%% Display of source code in math mode

\newcommand{\tm}[1]{\mathsf{#1}} % the source code font

%
\definecolor{keywordColor}{cmyk}{0.9, 0.4, 0.1, 0.2} % i dont have that color otherwise
\newcommand{\tmkw}[1]{\tm{\color{keywordColor}#1}} % source code keyword, colored

\newcommand{\tmpromise}[1]{\langle #1 \rangle} % completed promise

%% values
\newcommand{\tmconst}[1]{\tm{#1}}
\newcommand{\tmunit}{()} % the element of the unit type
\newcommand{\tmpair}[2]{( #1 , #2 )} % ordered pair
\newcommand{\tminl}[2][]{\tmkw{inl}_{#1}\,#2} % left injection
\newcommand{\tminr}[2][]{\tmkw{inr}_{#1}\,#2} % right injection
\newcommand{\tmfun}[2]{{\mathop{\tmkw{fun}}}\; (#1) \mapsto #2} % function abstraction
\newcommand{\tmfunano}[2]{{\mathop{\tmkw{fun}}}\; #1 \mapsto #2} % function abstraction (no type annotation expected)
\newcommand{\tmfunrec}[3]{\tmkw{fun}\; \tmkw{rec}\; #1\; (#2) \mapsto #3} % recursive abstraction
\newcommand{\tmfunrecano}[3]{\tmkw{fun}\; \tmkw{rec}\; #1\; #2 \mapsto #3} % recursive abstraction (no type annotation expected)
\newcommand{\tmapp}[2]{#1\,#2} % application
\newcommand{\tmboxed}[1]{[#1]} % boxed value

%% computations
\newcommand{\tmunbox}[3]{\tmkw{unbox}\; #1\; \tmkw{as} \; \tmboxed{#2} \; \tmkw{in}\;#3} % unbox comand
\newcommand{\tmreturn}[2][]{\tmkw{return}_{#1}\, #2} % pure computation
\newcommand{\tmlet}[3]{\tmkw{let}\; #1 = #2 \;\tmkw{in}\; #3} % let-binding
\newcommand{\tmletrec}[5][]{\tmkw{let}\;\tmkw{rec}\; #2\; #3 #1 = #4 \;\tmkw{in}\; #5} % recursive definitions

\newcommand{\tmop}[4]{\tm{#1}\;(#2, #3. #4)} % operation call
\newcommand{\tmopin}[3]{\tmkw{\downarrow}\, \tm{#1}\,(#2, #3)} % incoming interrupt
\newcommand{\tmopout}[3]{\tmkw{\uparrow}\,\tm{#1}\, (#2, #3)} % outgoing signal
\newcommand{\tmopoutbig}[3]{\tmkw{\uparrow}\,\tm{#1}\, \big(#2, #3\big)} % outgoing signal with big brackets
\newcommand{\tmopoutgen}[2]{\tmkw{\uparrow}\,\tm{#1}\, #2} % generic variant of outgoing signal

\newcommand{\tmmatch}[3][]{\tmkw{match}\;#2\;\tmkw{with}\;\{#3\}_{#1}} % match statement

\newcommand{\tmawait}[3]{\tmkw{await}\;#1\;\tmkw{until}\;\tmpromise{#2}\;\tmkw{in}\;#3} % awaiting for a promise to be completed

\newcommand{\tmwith}[5]{\tmkw{promise}\; (\tm{#1}\; #2 \mapsto #3)\; \tmkw{as}\; #4\; \tmkw{in}\; #5} % interrupt hook
\newcommand{\tmwithrec}[6]{\tmkw{promise}\; (\tm{#1}\; #2\; #3 \mapsto #4)\; \tmkw{as}\; #5\; \tmkw{in}\; #6} % rec interrupt hook
\newcommand{\tmwithrecgu}[7]{\tmkw{promise}\; (\tm{#1}\; #2\; #3\; \tmkw{when}\; #4 \mapsto #5)\; \tmkw{as}\; #6\; \tmkw{in}\; #7} % rec interrupt hook with guard
%\newcommand{\tmwith}[6]{\tmkw{promise}\; (\tm{#1}\; #2 \mapsto #3)\; \tmkw{as}\; #4 \of \typromise{#5}\; \tmkw{in}\; #6} % interrupt hook
\newcommand{\tmspawn}[2]{\tmkw{spawn} (#1, #2)} % Spawn new process

%% process

\newcommand{\tmrun}[1]{\tmkw{run}\; #1} % running a computation as a process
\newcommand{\tmpar}[2]{#1 \mathbin{\tmkw{\vert\vert}} #2} % parallel composition of processes

%%% Operational semantics

\newcommand{\reduces}{\leadsto} % small-step reduction
\newcommand{\tyreduces}{\rightsquigarrow} % reduction of process types

\newcommand{\E}{\mathcal{E}} % evaluation context for computations
\renewcommand{\H}{\mathcal{H}} % signal hoisting context
\newcommand{\F}{\mathcal{F}} % evaluation context for processes

%%% Typing rules

\newcommand{\types}{\vdash} % typing judgement
\newcommand{\of}{\mathinner{:}} % the colon in a typing judgement

\newcommand{\sub}{\sqsubseteq} % subtyping relation

\definecolor{rulenameColor}{cmyk}{0.1, 0.1, 0.1, 0.4} % i dont have that color otherwise
\newcommand{\coopinfer}[3]{\inferrule*[Lab={\color{rulenameColor}#1}]{#2}{#3}}

%%% Meta-theory

\makeatletter
\newcommand{\hourglass}{}                  % hourglass symbol for classifying temporarity blocked computations
\DeclareRobustCommand{\hourglass}{\mathrel{\mathpalette\hour@glass\relax}}

\newcommand\hour@glass[2]{%
  \vcenter{\hbox{%
    \rotatebox[origin=c]{90}{\scalebox{0.8}{$\m@th#1\bowtie$}}%
  }}%
}
\makeatother

\newcommand{\awaiting}[2]{#1 \hourglass #2} % computations blocked on awaiting a particular promise variable to be fulfilled

\newcommand{\CompResult}[2]{\mathsf{CompRes}\langle#1 \,\vert\, #2\rangle} % top-level result forms of individual computations
\newcommand{\RunResult}[2]{\mathsf{RunRes}\langle#1 \,\vert\, #2\rangle} % local (under-signal) result forms of individual computations

\newcommand{\Result}[2]{\mathsf{Res}\langle#1 \,\vert\, #2\rangle} % top-level result forms of computations

\newcommand{\ProcResult}[1]{\mathsf{ProcRes}\langle #1 \rangle} % top-level result forms of parallel processes
\newcommand{\ParResult}[1]{\mathsf{ParRes}\langle #1 \rangle} % intermediate result forms of parallel processes

%%% Maths

\newcommand{\cond}[3]{\mathsf{if}\;#1\;\mathsf{then}\;#2\;\mathsf{else}\;#3} % single line conditional

\newcommand{\carrier}[1]{\vert #1 \vert} % carrier of a cpo
\newcommand{\order}[1]{\sqsubseteq_{#1}} % partial order of a cpo
\newcommand{\lub}[1]{\bigsqcup_n \langle #1 \rangle} % least upper bound of an omega-chain

\newcommand{\Pow}[1]{\mathcal{P}(#1)} % powerset
\newcommand{\sem}[1]{[\![#1]\!]} % semantic bracket

\makeatletter
\providecommand*{\cupdot}{%     % disjoint union of sets
  \mathbin{%
    \mathpalette\@cupdot{}%
  }%
}
\newcommand*{\@cupdot}[2]{%
  \ooalign{%
    $\m@th#1\cup$\cr
    \hidewidth$\m@th#1\cdot$\hidewidth
  }%
}
\makeatother


%%% Redex highlighting

\definecolor{redexColor}{rgb}{0.83, 0.83, 0.83} % the color of highlighted redexes
\newcommand{\highlightgray}[1]{{\setlength{\fboxsep}{1.5pt}\colorbox{redexColor}{$#1$}}} % highlight redexes with gray(ish) background
\newcommand{\highlightwhite}[1]{{\setlength{\fboxsep}{1.5pt}\colorbox{white}{$#1$}}} % highlight redexes with white background


%%%%%% Highlighting important parts of proofs
\definecolor{kost}{RGB}{240, 220, 180}
\newcommand{\sitem}{\colorbox{kost}{$\odot$}}
\newcommand{\sitemm}{\textcolor{red}{$\odot$}}

% ukaz za slovarsko geslo / angleško-slovenski slovar
\newcommand{\geslo}[2]{\noindent\textbf{#1}\hspace*{3mm}\hangindent=\parindent\hangafter=1 #2\par}



%% for aeff coloring
\def\lstlanguagefiles{aeff}
\lstset{language=aeff,upquote=true}
\let\ls\lstinline



% todo notes to comment the code when working in a group
\usepackage{todonotes}
\definecolor{jcyan}{cmyk}{1, 0, 0.15, 0.05}
\newcommand\mP[1]{\todo[inline,color=red]{#1 -MP}}	% comments by matija
\newcommand\jR[1]{\todo[inline,color=jcyan]{#1 -JR}} % comments by janez


\setlength{\overfullrule}{50pt} % označi predlogo vrstico
\pagestyle{plain}               % samo številka strani na dnu, nobene glave / noge

% ukazi za matematična okolja
\theoremstyle{definition} % tekst napisan pokončno
\newtheorem{definicija}{Definicija}[section]
\newtheorem{primer}[definicija]{Primer}
\newtheorem{opomba}[definicija]{Opomba}
\newtheorem{aksiom}{Aksiom}

\theoremstyle{plain} % tekst napisan poševno
\newtheorem{lema}[definicija]{Lema}
\newtheorem{izrek}[definicija]{Izrek}
\newtheorem{trditev}[definicija]{Trditev}
\newtheorem{posledica}[definicija]{Posledica}

\numberwithin{equation}{section}  % števec za enačbe zgleda kot (2.7) in se resetira v vsakem poglavju



% bold matematika znotraj \textbf{ }, tudi v naslovih, kot \omega spodaj
\makeatletter \g@addto@macro\bfseries{\boldmath} \makeatother

% Poimenuj kazalo slik kot ``Kazalo slik'' in ne ``Slike''
\addto\captionsslovene{
  \renewcommand{\listfigurename}{Kazalo slik}%
}

% če želiš, da se poglavja začnejo na lihih straneh zgoraj
% \let\oldsection\section
% \def\section{\cleardoublepage\oldsection}

%%%%%%%%%%%%%%%%%%%%%%%%%%%%%%%%%%%%%%%%%%
%%%%%%           DOCUMENT           %%%%%%
%%%%%%%%%%%%%%%%%%%%%%%%%%%%%%%%%%%%%%%%%%

\begin{document}

\pagenumbering{roman} % začnemo z rimskimi številkami
\thispagestyle{empty} % ampak na prvi strani ni številke

\noindent{\large
UNIVERZA V LJUBLJANI\\[1mm]
FAKULTETA ZA MATEMATIKO IN FIZIKO\\[5mm]
\program\ -- 2.~stopnja}
% ustrezno dopolni za IŠRM
\vfill

\begin{center}
  \large
  \imeavtorja\\[3mm]
  \Large
  \textbf{\MakeUppercase{\naslovdela}}\\[10mm]
  \large
  Magistrsko delo \\[1cm]
  Mentor: \imementorja \\[2mm] % ustrezno popravi spol
  Somentor: \imesomentorja   % dodaj, če potrebno
\end{center}
\vfill

\noindent{\large Ljubljana, \letnica}

\cleardoublepage

%% sem pride IZJAVA O AVTORSTVU  -- SE NATISNE V VIS

% zahvala
\pdfbookmark[1]{Zahvala}{zahvala} %
\section*{Zahvala}
Neobvezno.
Zahvaljujem se \dots
% end zahvala -- izbriši vse med zahvala in end zahvala, če je ne rabiš

\cleardoublepage

\pdfbookmark[1]{\contentsname}{kazalo-vsebine}
\tableofcontents

% list of figures
% \cleardoublepage
% \pdfbookmark[1]{\listfigurename}{kazalo-slik}
% \listoffigures
% end list of figures

\cleardoublepage

\section*{Program dela}
\addcontentsline{toc}{section}{Program dela} % dodajmo v kazalo
%Mentor naj napiše program dela skupaj z osnovno literaturo. Na literaturo se
%lahko sklicuje kot~\cite{lebedev2009introduction}, \cite{gurtin1982introduction},
%\cite{zienkiewicz2000finite}, \cite{STtemplate}.
V današnjem času se vedno več programov na lokalni napravi izvaja usklajeno z oddaljenim strežnikom.
Sinhrono izvajanje, pri katerem klicoči računalnik počaka na odziv klicanega, je sicer enostavno za implementacijo, vendar je za uporabnika praktično neuporabno, saj bi na primer pametni telefon postal neodziven, dokler bi čakal na zahtevane podatke s strežnika.
Za dobro uporabniško izkušnjo moramo vložiti veliko dodatnega truda in programe pisati \emph{asinhrono}, pri čemer nam lahko precej pomaga dobro zasnovan programski jezik.

V delu predstavite algebrajski pristop k asinhronemu izvajanju~\cite{aeff} ter prototipni programski jezik Æff~\cite{webaeff}.
Izpostavite težave, zaradi katerih je treba omejiti vrste podatkov, ki jih lahko prenašamo, ter raziščite razširitev z modalnimi tipi~\cite{fitch}. Razširjeni jezik tudi implementirajte.

\section*{Osnovna literatura}
%Literatura mora biti tukaj posebej samostojno navedena (po pomembnosti) in ne
%le citirana. V tem razdelku literature ne oštevilčimo po svoje, ampak uporabljamo
%okolje itemize in ukaz plancite, saj je celotna literatura oštevilčena na koncu.
\begin{itemize}
  \plancite{aeff}
  \plancite{webaeff}
  \plancite{fitch}
\end{itemize}

\vspace{2cm}
\hspace*{\fill} Podpis mentorja: \phantom{prostor za podpis}

 \vspace{2cm}
 \hspace*{\fill} Podpis somentorja: \phantom{prostor za podpis}

\cleardoublepage
\pdfbookmark[1]{Povzetek}{abstract}

\begin{center}
\textbf{\naslovdela} \\[3mm]
\textsc{Povzetek} \\[2mm]
\end{center}
%Tukaj napišemo povzetek vsebine. Sem sodi razlaga vsebine in ne opis tega, kako je delo
%organizirano.
V delu si pogledamo programski jezik \aeff{} in \lae{}-račun na katerem temelji. Glavna lastnost \lae{}-računa je možnost asinhronega izvajanja v navezi z vzporednim izvajanjem. To dosežemo tako, da prestreznik naprej izvaja svoje nadaljevanje, medtem ko čaka, da dobi odgovor v obliki učinka iz drugega procesa. Ko prestreže želeni učinek, ga primerno obdela in izvajanje se spet nadaljuje. Osnovno verzijo \lae{}-računa razširimo z rekurzivnimi obljubami, mobilnimi tipi in dinamičnimi procesi. Rekurzivne obljube omogočijo, da bo program lahko reagiral na več učinkov z istim imenom operacije. Mobilni tipi omogočijo pošiljanje vrednosti višjega reda, predvsem funkcij. Dinamični procesi omogočijo ustvarjanje novih procesov sproti po potrebi. 
Dokažemo izreka o napredku in ohranitvi za osnovni in razširjen \lae{}-račun.
Omenimo tudi nekatere razlike med \lae{}-računom in dejansko implementacijo jezika \aeff{}.



\vfill
\begin{center}
\textbf{Asynchronous algebraic effects} \\[3mm] % prevod slovenskega naslova dela
\textsc{Abstract}\\[2mm]
\end{center}
%An abstract of the work is written here. This includes a short description of
%the content and not the structure of your work.
We look at programming language \aeff{} and \lae{}-calculus on which \aeff{} is based. The main feature of the \lae-calculus is the ability to run asynchronously in conjunction with parallel execution.
This is accomplished by continuing to perform handlers continuation while it waits to receive a response in the form of an effect from another process. Once it captures the desired effect, it processes it appropriately and execution resumes.
We extend the basic version of the \lae{}-calculus with recursive promises, mobile types, and dynamic processes.
Recursive promises allow the program to be able to respond to multiple effects with the same operation name.
Mobile types allow you to send higher-order values, especially functions.
Dynamic processes allow you to create new processes on the fly as needed.
We prove the progress and preservation theorem for the basic and extended \lae{}-calculus.
We also mention some differences between the \lae{}-calculus and the actual implementation of the \aeff{} language. 

\vfill\noindent
\textbf{Math.~Subj.~Class.~(2010):} 68M99, 68N15, 68N18, 68N19, 68Q85 \\[1mm]
\textbf{Ključne besede:} \kljucnebesede \\[1mm]
\textbf{Keywords:} \keywords

\cleardoublepage

\setcounter{page}{1}    % od sedaj naprej začni zopet z 1
\pagenumbering{arabic}  % in z arabskimi številkami


\section{Uvod} \label{sec:uvod}
%Napišite kratek zgodovinski in matematični uvod.  Pojasnite motivacijo za problem, kje
%nastopa, kje vse je bil obravnavan. Na koncu opišite tudi organizacijo dela -- kaj je v
%katerem razdelku.


Praktično vsak moderen programski jezik je ekvivalenten Turingovemu stroju. Kljub temu se lahko zelo razlikujejo v raznih objektivnih lastnostih, kot so poraba pomnilnika, hitrosti izvajanja, dolžina izvorne kode itd. Prav tako se lahko razlikujejo tudi v bolj subjektivnih lastnostih kot je programska paradigma, berljivost in razumljivost izvorne kode, enostavnost implementacije algoritmov, kvaliteti dokumentacije itd.

Eden izmed možnih pristop k pohitritvi izvajanja programa je vzporedno računanje. Namesto da korake v programu izvajamo enega za drugim, jih poskusimo čim več izvajati hkrati. Določeni koraki so lahko odvisni od drugih korakov in je zato zelo pomembno da jih izvedemo v pravem vrstnem redu. Sicer se lahko njihov pomen spremeni. Ugotoviti kateri koraki so med sabo odvisni in kateri ne je le redko enostavno. Zato je standarden pristop k temu, da programer korake razdeli v procese. Koraki znotraj istega procesa se smatrajo za odvisne, tudi če so neodvisni. Koraki iz različnih procesov se smatrajo za neodvisne (tudi če so odvisni!), razen kadar jih posebej označimo da so odvisni med sabo.

Na programerju je da se pri delitvi in označevanju ne zmoti. Zato je želeno od programskega jezika, da ima korake, ki naredijo ta postopek čim bolj enostaven in razumljiv.

Kljub temu, da sedaj načeloma lahko procese izvajamo hkrati, se mora izvajanje kakšnega procesa včasih začasno ustaviti, ker je nek korak v procesu odvisen od nekega drugega koraka v drugem procesu. Le to ima lahko dve negativni posledici. Lahko sproži verižno reakcijo, ki ustavi še več procesov. Lahko pa se je ustavil kritičen proces, za katerega si iz različni razlogov želimo, da se nebi ustavil ali pa vsaj minimalno.


Ta problem lahko deloma rešimo tako, da sprostimo predpostavko, da so vsi koraki znotraj procesa odvisni med sabo. Kadar bi se zaradi trenutnega koraka moral proces ustaviti, poskusimo izvajati preostale korake v procesu, kadar je to mogoče. Problematičen korak pa izvedemo šele ko so se izvedli vsi koraki od katerih je odvisen. Takemu izvajanju, kjer medtem, ko čakamo na nek dogodek, izvajamo nek drug del, pravimo asinhrono izvajanje. 

%Asinhrono izvajanje ni nujno povezano z vzporednim izvajanjem. 



Nekateri programski jeziki, predvsem starejše verzije, nimajo direktno podpore za vzporedno izvajanje programa. 
Veliko modernih jezikov, če ne drugega omogočajo tako imenovan POSIX. POSIX je aplikacijski vmesnik, ki na operacijskih sistemih, ki ga podpirajo, omogoča uporabo niti~\cite{posix}.
Drugi programski jeziki kot so C++, Java... imajo kot del standardne knjižnice konstrukte, ki jim omogočajo vzporedno izvajanje programa. Čeprav v ozadju mnogi jeziki uporabljajo POSIX, so detajli skriti in zato bolj naravni za programerja. 
Nekateri programski jeziki omogočajo mnoge konstrukte, ki jih povezujemo s vzporednim programiranjem, vendar direktno ne omogočajo vzporednega izvajanja. Na primer CPython ima tako imenovan GIL (global  interpreter lock)~\cite{gil}, ki preprečuje tolmaču, da bi izvajal korake iz več kot ene nit na enkrat.
Le redki programski jeziki, recimo Go, pa so bili zasnovani z vzporednim programiranjem v mislih in imajo zato konstrukte potrebne za vzporedno izvajanje vgrajene v jezik.

Asinhrono izvajanje programa podpirajo mnogi programski jeziki namenjeni izdelovanje spletnih aplikacij. Povezovanje z neko drugo napravo preko interneta je relativno počasno in zato je pomembno, da namesto, da čakamo, opravljamo medtem nekaj drugega. JavaScript je primer takega jezika. 



Eden izmed jezikov, ki podpira asinhrono izvajanje, je jezik \aeff{}. Jezik \aeff{} je statično tipiziran tolmačen funkcijski jezik. Program napisan v \aeff{} je razdeljen v procese. Procesi lahko med sabo komunicirajo s pomočjo učinkov. V trenutni verziji tolmač le simulira vzporedno izvajanje in posledično prednosti asinhronega in vzporednega izvajanja ne pridejo do izraza. To je izključno omejitev implementacije in bi se jo v prihodnosti lahko odpravilo.

\aeff{} sloni na \lae{}-računu~\cite{aeff}. Namen \lae{}-računa je pokazati, da v ozadju kombiniranja asinhronega vzporednega izvajanja z učinki stoji lepa teorija in so posledično koncepti uporabljeni v jeziku \aeff{} smiselni.


Najprej si bomo v poglavju \ref{sec:primeri-aeff} pogledali nekaj primerov napisanih v jeziku \aeff{}, na katerih bo prišlo do izraza asinhrono vzporedno izvajanje.
Nato si bomo v poglavju \ref{sec:lae} pogledali že obstoječ \lae{}-račun. V poglavju \ref{sec:razsirjen-lae} ga bomo razširili z prenosljivimi vrednostmi, rekurzivnimi obljubami in dinamičnimi procesi. V poglavju \ref{sec:impl} si bomo pogledali implementacijo \aeff{} in nekatere razlike med \aeff{} in \lae{}-računom. Obstoječ Hindley–Milnerjev sistem tipov bomo zamenjali z dvosmernim sistemom tipov in obstoječ tolmač spremenili, da omogoča nekoliko bolj optimalno izvajanje. Vse dele implementacije bomo prilagodili tudi za razširjen \lae-račun.


\section{Primeri v \aeff{}} \label{sec:primeri-aeff}

Poglejmo si sledeči sekvenčni sinhroni primer. Poračunati moramo zelo zahtevno in počasno točno funkcijo in prikazati njen rezultat na zaslonu. Ker je tako zelo počasna, preden se je dejansko lotimo, poračunamo še hitrejši približek. Pseudo kodo vidimo na programu~\ref{prog:primer-0}

\begin{lstlisting}[caption={Sinhron sekvečni primer.},label={prog:primer-0},float,floatplacement=h]
x' = priblizek(42)
prikazi_na_zaslon(x')
x = tocna_funkcija(42)
prikazi_na_zaslon(x)

kontinuacija()
\end{lstlisting}

Ta primer ima dve težavi, ki sta posledici sekvenčnega sinhronega izvajanja. Prvi problem je da, ker je program sekvenčni, ne moremo računati točne funkcije hkrati z približkom in posledično se program izvaja dlje časa. Drugi problem pa je, da dokler se bo računala točna funkcija, ne moremo izvajati kontinuacije, kar pomeni da se program ne more odzivati na uporabnikov vhod.  

Obema težavama se zlahka izognemo če program napišemo v jeziku \aeff, kot vidimo v programu~\ref{prog:primer-1}. Najprej v prvem procesu poračunamo približek in ga prikažemo. Nato s pomočjo efekta v drug proces pošljemo vhod točne funkcije. V drugem procesu prestrežemo število $42$, začnemo računati točno funkcijo in ko končamo pošljemo rezultat nazaj v glavni proces. Medtem ko se računa točna vrednost se glavni proces ne zaustavi, ampak nadaljuje z izvajanjem kontinuacije. Ko glavni proces prestreže rezultat, ga izpiše na zaslon in veže v spremenljivko p, da ga po potrebi lahko uporabi tudi kontinuacija.

\begin{lstlisting}[caption={Asinhron Vzporeden primer.},label={prog:primer-1},float,floatplacement=h]
operation call : int
operation result : int
 
run x' = priblizek(42)
	prikazi_na_zaslon(x')
	send call 42;
	promise (result x ->
		prikazi_na_zaslon(x);
		<x>
	) as p in
	kontinuacija()

run promise(call a ->
		let x = tocna_funkcija(a) in
		send result x;
		<()>
	) as _ in
	()
\end{lstlisting}


\section{Račun \lae{}}\label{sec:lae}

Račun $\lambda$ je preprost teoretičen programski jezik. Leta 1930 ga je uvedel Alonzo Chuch z namenom formalizacije koncepta izračunljivosti~\cite{rojas2015tutorial}. Račun lambda je Turingovo poln.


Račun \lae{} je razširitev računa $\lambda$. Posledično je tudi \lae{}-račun Turingovo poln. Računu \lae{} dodamo izraze, ki poosebijo bistvo asinhrono vzporednega programiranja z pomočjo učinkov. Da \lae{}-račun čim bolj približamo vsakdanjim programskim jezikom, mu dodamo tudi izraze za  naravna števila, par, ujemanje vzorca...


\subsection{Izrazi}

Izraze v \lae{}-računu razdelimo na vrednosti, izračune in procese. 


Vrednosti so sledeče.
Konstante vrednosti naravna števila in logični vrednosti resnica in neresnica.
% TOLE DODAJ ČE BO MAGISTRSKA PREKRATKA DA BODO DALJŠI DOKAZI in nekatere aritmetične in logične funkcije kot so $+$, $-$, $*$, $/$, $=$, $<$...
Spremenljivke, ki so simbolična imena povezana z vrednostmi v danem kontekstu.
Enote in pare.
leve in desne inkluzije vsote.
Lambda abstrakcije.
Rekurzivne lambda abstrakcije.
Izpolnjena obljuba. Ta vrednost je edina zares nova. Je vrednost, ki jo je vrnil prestreznik in jo posebej označimo, da bomo kasneje lahko pravilno določili tip.


Izračuni so sledeči.
Vrni ki drži neko vrednost.
Zaporedje dveh izračunov.
Aplikacija ki v prvi izraz substituira drugi izraz.
Ujemanje ki glede na vzorec izraza izbere izračun. 
Signal vsebuje operacijo, pripadajočo vrednost imenovano tovor in izračun.
Prekinitev vsebuje operacijo, pripadajočo vrednost imenovano tovor in izračun.
Prestreznik vsebuje ime operacije, ime spremeljivke, Izračun M in izračun N.
blokada


Pripadajočo sintakso vidimo na~\ref{fig:izrazi} v Backus-Naurjevi obliki (BNF).



\begin{figure}[h]
	\parbox{\textwidth}{
		\centering
		\small
		\begin{align*}
		\intertext{\textbf{Vrednosti}}
		V, W
		\bnfis& n \bnfor\! \true \bnfor\! \false        & &\text{konstantne vrednosti} \\
		\bnfor& x                                       & &\text{spremenljivka} \\
		\bnfor& \tmunit \bnfor\! \tmpair{V}{W}          & &\text{enota in par} \\
		\bnfor& \tminl[Y]{V} \bnfor\! \tminr[X]{V}      & &\text{leva in desna inkluzija} \\
		\bnfor& \tmfun{x}{M}                        & &\text{lambda} \\
		\bnfor& \tmfunrec{f}{x : X}{M}                        & &\text{rekurzivna lambda} \\
		\bnfor& \tmpromise V                            & &\text{izpolnjena obljuba}
		\\[1ex]
		\intertext{\textbf{Izračuni}}
		M, N
		\bnfis& \tmreturn{V}                            & &\text{vrnjena vrednost} \\
		\bnfor& \tmlet{x}{M}{N}                         & &\text{zaporedje} \\
	%	\bnfor& \tmletrec[: \tyfun{X}{Y}]{f}{x}{M}{N} & &\text{rekurzivna definicija} \\
		\bnfor& V\,W                                    & &\text{aplikacija} \\
		\bnfor& \tmmatch{V}{\tmpair{x}{y} \mapsto M}    & &\text{ujemanje produkta} \\
		\bnfor& \tmmatch[]{V}{}                        & &\text{prazno ujemanje} \\
		\bnfor& \tmmatch{V}{\tminl{x} \mapsto M, \tminr{y} \mapsto N}	& &\text{ujemanje vsote} \\
		\bnfor& \tmopout{op}{V}{M}       & &\text{signal} \\
		\bnfor& \tmopin{op}{V}{M}          & &\text{prekinitev} \\
		\bnfor& \tmwith{op}{x}{M}{p}{N}      & &\text{prestreznik} \\
		\bnfor& \tmawait{V}{x}{M}             & &\text{blokada}
			\\[1ex]
		\intertext{\textbf{Procesi}}
	%	P \bnfis & ...
		  P, Q
		\bnfis & \tmrun M & & \text{run} \\
		\bnfor & \tmpar P Q & & \text{vzporedna procesa} \\
		\bnfor & \tmopout{op}{V}{P} & & \text{proces signal} \\
		\bnfor & \tmopin{op}{V}{P} & & \text{proces prekinitev}
		\end{align*}
	} 
	\caption{Vrednosti, izračuni in procesi.}
	\label{fig:izrazi}
\end{figure}


\subsection{Operacijska semantika}

Račun \lae\ opremimo z operacijsko semantiko malih korakov, ki je definirana z relacijo korak $M \reduces N$. Redukcijska pravila za izračune so podana na sliki~\ref{fig:small-step-semantics-of-computations}. Redukcijska pravila za procese so podana na sliki~\ref{fig:small-step-semantics-of-processes}.

\begin{figure}[h]
	\centering
	\small
	\begin{align*}
	\intertext{\textbf{Pravila osnovnih izračunov}}
	\tmapp{(\tmfun{x \of X}{M})}{V} &\reduces M[V/x]
	\\
	\tmapp{(\tmfunrec{f}{x \of X}{M})}{V} &\reduces M[V/x, (\tmfunrec{f}{x \of X}{M})/f]
	\\
	\tmlet{x}{(\tmreturn V)}{N} &\reduces N[V/x]
	\\
	\tmmatch{\tmpair{V}{W}}{\tmpair{x}{y} \mapsto M} &\reduces M[V/x, W/y]
	\\
	\mathllap{\tmmatch{(\tminl[Y]{V})}{\tminl{x} \mapsto M, \tminr{y} \mapsto N}} &\reduces	M[V/x]
	\\
	\mathllap{\tmmatch{(\tminr[X]{W})}{\tminl{x} \mapsto M, \tminr{y} \mapsto N}} &\reduces	N[W/y]
	\\[1ex]
	\intertext{\textbf{Algebraičnost signala in prestreznika}}
	\tmlet{x}{(\tmopout{op}{V}{M})}{N} &\reduces \tmopout{op}{V}{\tmlet{x}{M}{N}}
	\\
	\tmlet{x}{(\tmwith{op}{y}{M}{p}{N_1})}{N_2} &\reduces \tmwith{op}{y}{M}{p}{(\tmlet{x}{N_1}{N_2})}
	\\[1ex]
	\intertext{\textbf{Komutativnost operacij}}
	\tmwith{op}{x}{M}{p}{\tmopout{op'}{V}{N}} &\reduces \tmopout{op'}{V}{\tmwith{op}{x}{M}{p}{N}}
	\\
	\tmopin{op}{V}{\tmopout{op'}{W}{M}} &\reduces \tmopout{op'}{W}{\tmopin{op}{V}{M}}
	\\[1ex]
	\intertext{\textbf{Širitev prekinitve}}
	\tmopin{op}{V}{\tmreturn W} &\reduces \tmreturn W
	\\
	\tmopin{op}{V}{\tmwith{op}{x}{M}{p}{N}} &\reduces \tmlet{p}{M[V/x]}{\tmopin{op}{V}{N}}
	\\
	\tmopin{op'}{V}{\tmwith{op}{x}{M}{p}{N}} &\reduces \tmwith{op}{x}{M}{p}{\tmopin{op'}{V}{N}} \\
	&\qquad {\color{rulenameColor}(\op \neq \op')}
	\\[1ex]
	\intertext{\textbf{Čakanje na izpolnitev obljube}}
	\tmawait{\tmpromise V}{x}{M} &\reduces M[V/x]
	\\[-8ex]
	\end{align*}
	
	\begin{align*}
	\intertext{\textbf{Evalvacija v okolju}}
	\coopinfer{}{
		M \reduces N
	}{
		\E[M] \reduces \E[N]
	}
	\end{align*}
	\vspace{-6ex}
	\begin{align*}
	\intertext{\textbf{kjer}}
	\text{$\E$}
	\bnfis [~]
	\bnfor \tmlet{x}{\E}{N}
	\bnfor \tmopout{op}{V}{\E}
	\bnfor \tmopin{op}{V}{\E} 
	\bnfor \tmwith{op}{x}{M}{p}{\E}
	\end{align*}
	
	\caption{Operacijska semantika malih korakov za izračune.}
	\label{fig:small-step-semantics-of-computations}
\end{figure}

\begin{figure}[h]
    \centering
	\small
	\begin{minipage}[t]{0.4\textwidth}
		\centering
		\begin{align*}
		\intertext{\textbf{Posamezen proces}}
		\coopinfer{}{
			M \reduces N
		}{
			\tmrun M \reduces \tmrun N
		}
		\end{align*}
	\end{minipage}
	\qquad
	\begin{align*}
	\intertext{\textbf{Prehod}}
	\tmrun {(\tmopout{op}{V}{M})}  &\reduces \tmopout{op}{V}{\tmrun M}
	\\
	\tmopin{op}{V}{\tmrun M} &\reduces \tmrun {(\tmopin{op}{V}{M})}
	\\[1ex]
	\intertext{\textbf{Oddajanje signala}}
	\tmpar{\tmopout{op}{V}{P}}{Q} &\reduces \tmopout{op}{V}{\tmpar{P}{\tmopin{op}{V}{Q}}}
	\\
	\tmpar{P}{\tmopout{op}{V}{Q}} &\reduces \tmopout{op}{V}{\tmpar{\tmopin{op}{V}{P}}{Q}}
	\\[1ex]
	\intertext{\textbf{Širitev prekinitve}}
	\tmopin{op}{V}{\tmpar P Q} &\reduces \tmpar {\tmopin{op}{V}{P}} {\tmopin{op}{V}{Q}}
	\\[1ex]
	\intertext{\textbf{Komutativnost signala in prekinitve}}
	\tmopin{op}{V}{\tmopout{op'}{W}{P}} &\reduces \tmopout{op'}{W}{\tmopin{op}{V}{P}}
	\end{align*}
	\vspace{-4ex}
	\begin{align*}
	\shortintertext{\quad\textbf{Evalvacija v okolju}}
	\quad
	\coopinfer{}{
		P \reduces Q
	}{
		\F[P] \reduces \F[Q]
	}
	\end{align*}
	\vspace{-6ex}
	\begin{align*}
	\intertext{\textbf{kjer}}
	\text{$\F$}
	\bnfis& [~]
	\bnfor \tmpar \F Q \bnfor\! \tmpar P \F
	\bnfor \tmopout{op}{V}{\F}
	\bnfor \tmopin{op}{V}{\F}
	\end{align*}
	
	\caption{Operacijska semantika malih korakov za procese.}
	\label{fig:small-step-semantics-of-processes}
\end{figure}

Poleg teh pravil, ki so identična pravilom iz \cite{aeff}, dodamo še dve novi.
Prvo pravilo potegne blokado ven iz zaporedja. Drugo pravilo prestavi prekinitev takoj za blokado. Obe pravili skupaj dosežeta dvoje. Malenkost večji del prvotnega izračuna je postal asinhron. Kar nam lahko v primeru, da bo obljuba izpolnjena in bomo nadaljevali z izvajanjem tega dela, nekoliko pohitri izvajanje. Druga prednost pa je, da izračun, ki je v čakajočem stanju, se vedno začne z izračunom $\tmkw{Await}$. Posledično je prepoznati ali je izračun v čakajočem stanju trivialno in se bodo delni rezultati nekoliko poenostavili. 

Rezultati ostanejo enaki kot v \cite{aeff}.

\begin{figure}[h]
	\centering
	\small
	\begin{align*}
		\intertext{\textbf{Algebraičnost blokade}}
		\tmlet{x}{(\tmawait{V}{y}{M})}{N} & \reduces \tmawait{V}{y}{(\tmlet{x}{M}{N})}
		\\[1ex]
		\intertext{\textbf{Komutativnost blokade in prekinitve}}
		\tmopin{op}{V}{\tmawait{W}{x}{M}} &\reduces \tmawait{W}{x}{\tmopin{op}{V}{M}}
	\end{align*}
	
	\caption{Dodatni pravili operacijske semantike.}
	\label{fig:operacijska-semantika-poenostavitev}
\end{figure}

\subsection{Sistem tipov}


Da se izognemo nekaterim napakam ob izvajanju uvedemo sistem tipov.
Tipe ločimo na tipe za vrednosti, izračune in procese.
Večina vrednosti dobi standardne tipe, kot so naravno število, boolean, enota, par, vsota in funkcijski tip. 


%%Predolga/grda poved
Ker funkcijski tip označuje funkcijo, ki vzame argument in mu priredi izračun, le tej pa imajo, kot bomo kasneje videli, poleg standardnih tipov, še tipe za učinke, ima tudi funkcijski tip dodane tipe za učinke.


Dodatno za vrednost obljuba uvedemo tip \emph{obljuba} $\typromise{A}$. 
Tipom za vrednosti, ki ne vsebujejo funkcijskega tipa ali obljube, pravimo osnovni tipi. 


Izračunom poleg standardnega tipa dodamo še tipe signalov, ki jih lahko sprožimo, označene z $\o$ in tipe prekinitev, ki jih lahko prestrežemo, označene z $\i$.
Tipi učinkov $\o$ in $\i$ so elementi množice $O$ oziroma $I$.
Množica $\sig$ je množica vseh operacij, ki jih imamo na voljo.
Množica $O$ je preprosto potenčna množica množice $\sig$ in posledično $\o$ predstavlja množico signalov, ki jih izračun lahko sproži.

Ko prestrežemo neko prekinitev lahko ustrezen izračun obljuba začne sprožati nove signale in prestrezati nove prekinitve. Zato $I$ definiramo kot največjo fiksno točko preslikave $\omega$ definirane kot 
$$\Omega(X) = \sig \Rightarrow (O \times X)_\bot $$,
kjer je $\Rightarrow$ potenciranje, $\times$ je kartezični produkt in $(-)_\bot$ je dvig.


Tipi procesov so odvisni od tipov izračunov in posledično vsebujejo tipe učinkov.

Pravila za dodelitev tipa vidimo na sliki~\ref{fig:value-typing-rules} in~\ref{fig:computation-typing-rules}.


\begin{figure}[h]
	\parbox{\textwidth}{
		\centering
		\small
		\begin{align*}
		\text{Osnovni tipi vrednosti $\bar{A}$, $\bar{B}$}
		\bnfis & \tysym{int} \,\bnfor\! \tysym{bool} \,\bnfor\! \tyunit \,\bnfor\! \tyempty \,\bnfor\! \typrod{\bar{A}}{\bar{B}} \,\bnfor\! \tysum{\bar{A}}{\bar{B}}
		\\%[1ex]
		\text{Tipi vrednosti $A$, $B$}
		\bnfis & \bar{A} \, \bnfor\! \typrod{A}{B} \,\bnfor\! \tysum{A}{B} \,\bnfor\! \tyfun{A}{\tycomp{B}{\o,\i}} \,\bnfor\! \typromise{A}
		\\
		\text{Tip izračuna} \bnfis& \tycomp{A}{\o,\i}
		\\
		\text{Tip procesa \tyC, \tyD}  \bnfis & \tyrun{A}{\o, \i} \,\bnfor\! \typar{\tyC}{\tyD}
		\end{align*}
	} 
	\caption{Tipi izrazov}
	\label{fig:tipi}
\end{figure}

Vsaki operaciji priredimo nek tip, kot vidimo na sliki~\ref{fig:operacije}. Ker izračun 
$$ \tmwith{op}{x}{M}{p}{\tmopout{op'}{V}{N}} $$
lahko naredi korak v izračun
$$ \tmopout{op'}{V}{\tmwith{op}{x}{M}{p}{N}} $$,
je ključno da vrednost $V$ ne vsebuje spremenljivke $p$, saj v drugem izračunu spremenljivka $p$ v vrednosti $V$ ni več dobro definiran. Da zagotovimo, da vrednost $V$ ne vsebuje spremenljivke $p$, omejimo tipe, ki pripadajo operacijam, na osnovne tipe.

\begin{figure}[t]
	\centering
	\small
	\begin{align*}
	(op_1, \bar{A}_{op_1}),\, (op_2, \bar{A}_{op_2}),\, ... ,\, (op_n, \bar{A}_{op_k})
	\end{align*}
\vspace{-5ex}
	\caption{Operacije in pripadajoči osnovni tipi.}
	\label{fig:operacije}
\end{figure}



\begin{figure}[h]
	\centering
	\small
	\begin{mathpar}
		\coopinfer{Tip-Cons}{
		}{
			\Gamma \types n : int
		}
		\qquad
		\coopinfer{Tip-Cons}{
		}{
			\Gamma \types true : bool
		}
		\qquad
		\coopinfer{Tip-Cons}{
		}{
			\Gamma \types false : bool
		}
		\quad
		\coopinfer{Tip-Var}{
		}{
			\Gamma, x \of X, \Gamma' \types x : X
		}
		\quad
		\coopinfer{Tip-Unit}{
		}{
			\Gamma \types \tmunit : \tyunit
		}
		\\
		\coopinfer{Tip-Pair}{
			\Gamma \types V : X \\
			\Gamma \types W : Y
		}{
			\Gamma \types \tmpair{V}{W} : \typrod{X}{Y}
		}
		\quad
		\coopinfer{Tip-Promise}{
			\Gamma \types V : X
		}{
			\Gamma \types \tmpromise V : \typromise X
		}
		\quad
		\coopinfer{Tip-Inl}{
			\Gamma \types V : X
		}{
			\Gamma \types \tminl[Y]{V} : X + Y
		}
		\quad
		\coopinfer{Tip-Inr}{
			\Gamma \types W : Y
		}{
			\Gamma \types \tminr[X]{W} : X + Y
		}
		\\
		\coopinfer{Tip-Fun}{
			\Gamma, x \of X \types M : \tycomp{Y}{\o,\i}
		}{
			\Gamma \types \tmfun{x : X}{M} : \tyfun{X}{\tycomp{Y}{\o,\i}}
		}
		\quad
		\coopinfer{Tip-Fun-Rec}{
			\Gamma,f \of \tyfun{X}{\tycomp{Y}{\o,\i}}, x \of X \types M : \tycomp{Y}{\o,\i}
		}{
			\Gamma \types \tmfunrec{f}{x : X}{M} : \tyfun{X}{\tycomp{Y}{\o,\i}}
		}
	\end{mathpar}
	\caption{Pravila za izračun tipov za vrednosti.}
	\label{fig:value-typing-rules}
\end{figure}

\begin{figure}[h]
	\centering
	\small
	\begin{mathpar}
		\coopinfer{TyComp-Return}{
			\Gamma \types V : X
		}{
			\Gamma \types \tmreturn{V} : \tycomp{X}{(\o,\i)} 
		}
		\qquad
		\coopinfer{TyComp-Let}{
			\Gamma \types M : \tycomp{X}{(\o,\i)}
			\\
			\Gamma, x \of X \types N : \tycomp{Y}{(\o,\i)} 
		}{												
			\Gamma \types
			\tmlet{x}{M}{N} : \tycomp{Y}{(\o,\i)}       
		}											
		\\
		\coopinfer{TyComp-Apply}{
			\Gamma \types V : \tyfun{X}{\tycomp{Y}{(\o,\i)}} \\
			\Gamma \types W : X
		}{
			\Gamma \types \tmapp{V}{W} : \tycomp{Y}{(\o,\i)}
		}
		\quad
		\coopinfer{TyComp-MatchPair}{
			\Gamma \types V : \typrod{X}{Y} \\
			\Gamma, x \of X, y \of Y \types M : \tycomp{Z}{(\o,\i)}
		}{
			\Gamma \types \tmmatch{V}{\tmpair{x}{y} \mapsto M} : \tycomp{Z}{(\o,\i)}
		}
		\\
		\coopinfer{TyComp-MatchEmpty}{
			\Gamma \types V : \tyempty
		}{
			\Gamma \types \tmmatch[\tycomp{Z}{(\o,\i)}]{V}{} : \tycomp{Z}{(\o,\i)}
		}
		\quad
		\coopinfer{TyComp-MatchSum}{
			\Gamma \types V : X + Y \\\\
			\Gamma, x \of X \types M : \tycomp{Z}{(\o,\i)} \\
			\Gamma, y \of Y \types N : \tycomp{Z}{(\o,\i)} \\
		}{
			\Gamma \types \tmmatch{V}{\tminl{x} \mapsto M, \tminr{y} \mapsto N} : \tycomp{Z}{(\o,\i)}
		}
		\\
		\coopinfer{TyComp-Signal}{
			\op \in \o \\
			\Gamma \types V : A_\op \\
			\Gamma \types M : \tycomp{X}{(\o,\i)} 
		}{
			\Gamma \types \tmopout{op}{V}{M} : \tycomp{X}{(\o,\i)}
		}
		\qquad
		\coopinfer{TyComp-Interrupt}{
			\Gamma \types V : A_\op \\
			\Gamma \types M : \tycomp{X}{(\o,\i)} 
		}{
			\Gamma \types \tmopin{op}{V}{M} : \tycomp{X}{\opincomp {op} (\o,\i)}
		}
		\\
		\coopinfer{TyComp-Promise}{
			\i\, (\op) = ({\o'} , {\i'}) \\
			\Gamma, x \of A_\op \types M : \tycomp{\typromise X}{(\o',\i')} \\
			\Gamma, p \of \typromise X \types N : \tycomp{Y}{(\o,\i)} 
		}{
			\Gamma \types \tmwith{op}{x}{M}{p}{N} : \tycomp{Y}{(\o,\i)}
		}
		\\
		\coopinfer{TyComp-Await}{
			\Gamma \types V : \typromise X \\
			\Gamma, x \of X \types M : \tycomp{Y}{(\o,\i)} 
		}{
			\Gamma \types \tmawait{V}{x}{M} : \tycomp{Y}{(\o,\i)}
		}
		\qquad
		\coopinfer{TyComp-Subsume}{
			\Gamma \types M : \tycomp{X}{(\o, \i)} \\
			(\o,\i) \order {O \times I} (\o',\i')
		}{
			\Gamma \types M : \tycomp{X}{(\o', \i')}
		}
	\end{mathpar}
	\caption{Pravila za izračun tipov za izračune.}
	\label{fig:computation-typing-rules}
\end{figure}

\begin{figure}[h]
	\centering
	\small
	\begin{mathpar}
		\coopinfer{TyProc-Run}{
			\Gamma \types M : \tycomp{X}{\o,\i}
		}{
			\Gamma \types \tmrun{M} : \tyrun{X}{\o, \i}
		}
		\quad
		\coopinfer{TyProc-Par}{
			\Gamma \types P : \tyC \\
			\Gamma \types Q : \tyD
		}{
			\Gamma \types \tmpar{P}{Q} : \typar{\tyC}{\tyD}
		}
		\\
		\coopinfer{TyProc-Signal}{
			\op \in \mathsf{signals\text{-}of}{(\tyC)} \\\\
			\Gamma \types V : A_\op \\
			\Gamma \types P : \tyC 
		}{
			\Gamma \types \tmopout{op}{V}{P} : \tyC
		}
		\quad
		\coopinfer{TyProc-Interrupt}{
			\Gamma \types V : A_\op \\
			\Gamma \types P : \tyC 
		}{
			\Gamma \types \tmopin{op}{V}{P} : \opincomp{op}{\tyC}
		}  
	\end{mathpar}
	\caption{Pravila za izračun tipov za procese.}
	\label{fig:process-typing-rules}
\end{figure}

\begin{figure}[h]
	\centering

%	\textbf{Čakajoči izrazi}
%	\begin{mathpar}
%		\coopinfer{}{
%		}{
%			\awaiting p {\tmawait p x M}
%		}
%		\quad
%		\coopinfer{}{
%			\awaiting p M
%		}{
%			\awaiting p {\tmlet x M N}
%		}
%		\quad
%		\coopinfer{}{
%			\awaiting p M
%		}{
%			\awaiting p {\tmopin{op}{V}{M}}
%		}
%	\end{mathpar}
	
	\textbf{Delni rezultati}
	\begin{mathpar}
		\coopinfer{}{
		}{
			\RunResult {\Psi} {\tmreturn V}
		}
		\\
		\coopinfer{}{
			\RunResult {\Psi \cup \{p\}} {N}
		}{
			\RunResult {\Psi} {\tmwith {op} x M p N}
		}
		\qquad
		\coopinfer{}{
			p \in \Psi
		}{
			\RunResult {\Psi} {\tmawait{p}{x}{M}}
		}
	\end{mathpar}

	\textbf{Rezultati}
	\begin{mathpar}
		\coopinfer{}{
			\ProcResult {P}
		}{
			\ProcResult {\tmopout {op} V P}
		}
		\qquad
		\coopinfer{}{
			\ParResult {P}
		}{
			\ProcResult {P}
		}
		\\
		\coopinfer{}{
			\ParResult P \\
			\ParResult Q
		}{
			\ParResult {\tmpar P Q}
		}
		\qquad
		\coopinfer{}{
			\RunResult {\emptyset} {M}
		}{
			\ParResult {\tmrun M}
		}
	\end{mathpar}
	\caption{Rezultati in delni rezultati.}
	\label{fig:results-rules}
\end{figure}


Izrek o varnosti, ki je sestavljen iz izreka o ohranitvi in izreka o napredku nam zagotavlja, da ne moremo dobiti runtime, če imamo tipe, kot smo jih definirali.



\begin{trditev}[o napredku]
	Naj za izračun $M$ velja $\emptyset \types M \of \tycomp{A}{\o, \i}$. Potem ali (i) obstaja izračun $M'$, tak da $M \reduces M'$, ali pa (ii) velja $\RunResult{\emptyset}{M}$.

	Naj za izračun $M$ velja $\Gamma \types M \of \tycomp{A}{\o, \i}$, kjer je $\Gamma = \{(x_1,\typromise{A_1}),(x_2,\typromise{A_2}),...,(x_i,\typromise{A_i})\}$. Potem ali (i) obstaja izračun $M'$, tak da $M \reduces M'$, ali pa (ii) velja $\RunResult{\emptyset}{M}$.
\end{trditev}

\begin{proof}
	Osnutek
	
	
	Ker ima izračun $M$ tip, obstaja drevo izpeljave za tip.
	Dokazujemo z indukcijo na globino drevesa izpeljave za izračune.
	
	Za $n$ je $1$ ločimo dva primera glede na uporabljeno pravilo v drevesu izpeljave.
	Če je $M = \tmreturn{V}$, potem po pravilu XYZ??? %Kako naj poimenujem pravila, korake... da se lahko kasneje sklicujem nanje???
	sledi, da je $M$ delni rezultat in velja (ii).
	Če je $M = V W$, potem mora biti $V$ oblike $\tmfunano{x}{M}$ ali $\tmfunrecano{f}{x}{M}$ in lahko naredimo korak v $M[W/x]$ ali $M[V/x, (\tmfunrec{f}{x \of X}{M})/f]$.
	
	Sedaj pokažimo da, če velja trditev za $n = 1,...,n$ velja tudi za $n+1$.
	Ločimo primere glede na zadnje uporabljeno pravilo.
	
	Če je zadnje pravilo $TyComp-Let$ potem ima $M$ %Kako naj ločim M iz trditve in M iz let M in N???
	tip in po $IP$ lahko naredi korak ali pa je delni rezultat. Če $M$ lahko naredi korak v $M'$ potem lahko po pravilu za evalvacijo v okolju tudi $M$ naredi korak.
	Če pa je $M$ delni rezultat ločimo tri možnosti. Če je $M = return V$ potem lahko naredimo korak v $N[V/x]$. V drugih dveh primerih lahko uporabimo algebraičnost prestreznika ali blokade in naredimo ustrezen korak. 
	
	Če je zadnje pravilo $TyComp-MatchPair$ potem je $V$ oblike $(V,W)$ in posledično lahko naredimo korak XYZ.
	
	Če je zadnje pravilo $TyComp-MatchEmpty$ %Nimamo pravila za V : 0: Zakaj bi potem imeli to pravilo???
	
	Če je zadnje pravilo $TyComp-MatchSum$ potem je $V$ ali oblike $inl_YV$ in lahko naredimo korak XYZ ali pa je oblike $inr_XV$ in lahko naredimo korak XYZ.
	
	Če je zadnje pravilo $TyComp-Signal$ %Tukaj bi morali signal prenesti iz izračuna na proces, ampak ker smo ločili izračune in procese tega ne moremo narediti? Naj združim nazaj v samo en izrek? Se da lepše formulirati trditev da bo vredu? Lahko v trditvi rečem da M ne sme biti signal. In potem v izreku za procese ta primer posebej obravnavam???
	
	Če je zadnje pravilo $TyComp-Interupt$ potem ima $M$ tip in po indukcijski predpostavki lahko $M$ naredi korak ali pa je delna vrednost. Če $M$ lahko naredi korak v $M'$ potem lahko po pravilu za evalvacijo v okolju tudi $M$ naredi korak. Če pa je $M$ delni rezultat ločimo tri možnosti. Če je $M = return V$ potem lahko naredimo korak v $M$. Če je $M = \tmwith{op'}{x}{M}{p}{N}$ naredimo ali korak XYZ ali XYZ glede na to ali velja $op = op'$.
	Tretja možnost se ne more zgodi saj je pripadajoči $\psi = \emptyset$.
	
	Če je zadnje pravilo $TyComp-promise$ potem ima $M$ tip in po indukcijski predpostavki lahko $M$ naredi korak ali pa je delna vrednost. Če $M$ lahko naredi korak v $M'$ potem lahko po pravilu za evalvacijo v okolju tudi $M$ naredi korak. Če pa je $M$ delni rezultat je tudi $M$ delni rezultat.
	
	Če je zadnje pravilo $TyComp-Await$, potem mora biti $V = <V>$ saj smo v praznem kontekstu. Tedaj lahko naredimo korak XYZ.
	
	Če je zadnje pravilo $TyComp-Subsume$ je $M$ po IP ali rezultat ali pa lahko naredi korak.
	
	%Nenehno se ponavlja "Če je zadnje pravilo...", "Ločimo primere ..." ... Glede na to da je to dokaz, ki mora biti čim bolj pregleden in razumljiv, ne pa estetsko oblikovan je to vredu??? 
	
	
	
	
\end{proof}


\begin{izrek}[o napredku]
	Naj za proces $P$ velja $\Gamma \types P \of C$. Potem ali (i) obstaja proces $P'$, tak da $P \reduces P'$, ali pa (ii) velja $\ProcResult{P}$.
\end{izrek}

\begin{proof}
	Če je izraz $M$ rezultat smo končali. Sicer dokazujemo 
\end{proof}



\begin{lema}[o substituciji]
	Naj za vrednost $V$ velja $\Gamma \types V \of A$ in za izračun $M$ velja $\Gamma, x \of A \types M \of \tycomp{B}{\o, \i}$. Potem velja $\Gamma \types M[V/x] \of \tycomp{B}{\o, \i}$.
\end{lema}

\begin{proof}
	...
\end{proof}

\begin{trditev}[o ohranitvi]
	Naj za izračun $M$ velja $\Gamma \types M \of \tycomp{A}{\o, \i}$. Če izračun $M$ naredi korak v izračun $M'$, potem velja $\Gamma \types M' \of A'$, kjer je $A' = $.
\end{trditev}

\begin{proof}
	...
\end{proof}

\begin{izrek}[o ohranitvi]
	Naj za proces $P$ velja $\Gamma \types P \of C$. Če proces $P$ naredi korak v proces $P'$, potem velja $\Gamma \types P' \of C'$, kjer je $C' = $.
\end{izrek}

\begin{proof}
	...
\end{proof}

\section{Razširjen \lae{}}\label{sec:razsirjen-lae}



\subsection{Rekurzivni prestreznik}


\mP{Tole še ni za oddati.}

Kadar pričakujemo več prekinitev z isto operacijo je lahko iz različnih razlogov za programerja priročno, če ima na voljo prestreznik, ki se po potrebi ponovno namesti. Lahko da je namen prestreznika, da na vsako prekinitev odgovori z signalom. Lahko nam prihajajoča prekinitev prinese pravo operacijo, vendar ne pravega pripadajočega tovora, zato obljube še nočemo izpolniti, in čakamo na naslednjo prekinitev.

To funkcionalnost smo do sedaj dosegli tako, da smo obljubo zapakirali v rekurzivno funkcijo in po potrebi znotraj obljube ponovno klicali to funkcij kot lahko vidimo v programu~\ref{prog:obljuba-v-rekurzivni-funkciji}.


\begin{lstlisting}[caption={Obljuba v rekurzivni funkciji.},label={prog:obljuba-v-rekurzivni-funkciji}]
	let rec f =
	promise (op x -> send op x; f(); return <<()>>) 
	as _ in ()
\end{lstlisting}

Predvsem z namenom bolj pregledne kode nadomestimo obstoječi prestreznik z rekurzivnim prestreznikom. Le ta ima poleg imena operacije $op$, tovora $V$, obljube $M$ in kontinuacije $N$ tudi spremenljivko $f$. Pripadajočo sintakso vidimo na sliki~\ref{fig:izrazi-prestreznik}.


\begin{figure}[h]
	\centering
	\small
	\begin{align*}
	M, N
	\bnfis& ...                            & &\text{obstoječi izračuni} \\
	\bnfor& \tmwithrec{op}{f}{V}{M}{p}{N}  & &\text{rekurzivni prestreznik}
	\end{align*}
 
	\caption{Izračuni z rekurzivnim prestreznikom}
	\label{fig:izrazi-prestreznik}
\end{figure}


Operacijska semantika rekurzivnega prestreznika je zelo podobna prejšnji verziji. Z to izjemo, da ko ustrezna prekinitev pride do prestreznika le ta v obljubi M naredi dve substituciji. Tako kot prej spremenljivko $x$ substituira z vrednostjo $V$, dodatno pa spremenljivko $f$ substituira z lambda funkcijo $F$, ki sprejme enoto in vrne svežo kopijo prestreznika. Znotraj $M$ lahko uporabimo $F$ in tako ponovno namestimo prestreznik.   


\begin{figure}[h]
	\centering
	\small
	\begin{align*}
	\tmlet{x}{(\tmwithrec{op}{f}{y}{M}{p}{N_1})}{N_2} &\reduces \tmwithrec{op}{f}{y}{M}{p}{(\tmlet{x}{N_1}{N_2})}
	\\
	\tmwithrec{op}{f}{x}{M}{p}{\tmopout{op'}{V}{N}} &\reduces \tmopout{op'}{V}{\tmwithrec{op}{f}{x}{M}{p}{N}}
	\\
	\tmopin{op}{V}{\tmwithrec{op}{f}{x}{M}{p}{N}} &\reduces \tmlet{p}{M[V/x, F/f]}{\tmopin{op}{V}{N}} \\
	F = \;& \tmfunano{y}{(\tmwithrec{op}{f}{x}{M}{p}{\tmreturn{p}})} \\
	\tmopin{op'}{V}{\tmwithrec{op}{f}{x}{M}{p}{N}} &\reduces \tmwithrec{op}{f}{x}{M}{p}{\tmopin{op'}{V}{N}} \\
	&\qquad {\color{rulenameColor}(\op \neq \op')}
	\end{align*}
	
	\caption{Operacijska semantika rekurzivnega prestreznika}
	\label{fig:semantika-prestreznik}
\end{figure}

Program~\ref{prog:rekurzivna-obljuba} ima isti semantičen pomen kot program~\ref{prog:obljuba-v-rekurzivni-funkciji}, le da je tokrat napisan z rekurzivno obljubo.
\begin{lstlisting}[caption={Rekurzivna obljuba.},label={prog:rekurzivna-obljuba}]
	promise (op x k -> send op x; k())
	as _ in ()
\end{lstlisting}

Kot sintaktični sladkor dodamo tudi varovan prestreznik, ki se sproži le kadar je poleg prave operacije tudi pravi pripadajoči tovor.

\begin{figure}[h]
	\centering
	\small
	\begin{align*}
	\intertext{\textbf{Sintaksa}}
	\tmwithrecgu{op}{x}{f}{P(x)}{M}{p}{N}  & &\text{varovan rekurzivni prestreznik}
	\\%[1ex]
	\intertext{\textbf{Semantika}}
	\tmopin{op}{V}{\tmwithrecgu{op}{f}{x}{P}{M}{p}{N}} \\ \reduces \tmlet{p}{(\ite{P[V/x]}{M[V/x, F/f]}{F\,()})}{\tmopin{op}{V}{N}} \\
	{\color{rulenameColor}F = \tmfunano{y}{(\tmwithrec{op}{f}{x}{M}{p}{\tmreturn{p}})}}
	\end{align*}
	
	\caption{Varovan rekurzivni prestreznik}
	\label{fig:izrazi-prestreznik}
\end{figure}


\subsection{Prenosljivi tipi}

Ko imamo nek signal je ključno, da tovor ne vsebuje vrednosti z tipom obljuba, kar smo do sedaj rešili tako, da smo omejili signale na osnovne tipe. Glavni problem tega pristopa je, da funkcijski tip ni med osnovnimi tipi in posledično ne moremo poslati lambda funkcij. Ta problem rešimo z uvedbo zavite vrednosti in zavitega tipa. Zavite vrednosti bodo lahko vrednosti tudi lambda funkcije, ki ne bodo vsebovale zunanje obljube. Posledično jih bo varno poslati. Osnovne tipe skupaj z zavitim tipom imenujemo prenosljivi tipi. Da bomo lahko zavito vrednost tudi uporabili dodamo izračun odvijanje. 

Tipi za izračune in procese ostanejo enaki.

\begin{figure}[h]
	\centering
	\small
	\begin{align*}
	\intertext{\textbf{Vrednosti}}
	V
	\bnfis& ...                            & &\text{obstoječe vrednosti} \\
	\bnfor& \tmboxed{V}  & &\text{zavita vrednost}
	\intertext{\textbf{Izračuni}}
	M, N
	\bnfis& ...                            & &\text{obstoječi izračuni} \\
	\bnfor& \tmunbox{V}{x}{M}  & &\text{odvijanje}
	\end{align*}
	
	\caption{Prenosljivi izrazi.}
	\label{fig:izrazi-prenosljivi}
\end{figure}

\begin{figure}[h]
	\centering
	\small
	\begin{align*}
	\text{Prenosljivi tipi vrednosti $\tymobile{A}$, $\tymobile{B}$}
	\bnfis & \tysym{int} \,\bnfor\! \tysym{bool} \,\bnfor\! \tyunit \,\bnfor\! \tyempty \,\bnfor\! \typrod{\tymobile{A}}{\tymobile{B}} \,\bnfor\! \tysum{\tymobile{A}}{\tymobile{B}} \,\bnfor\! \tyboxed{A}
	\\%[1ex]
	\text{Tipi vrednosti $A$, $B$}
	\bnfis & \tymobile{A} \, \bnfor\! \typrod{A}{B} \,\bnfor\! \tysum{A}{B} \,\bnfor\! \tyfun{A}{\tycomp{B}{\o,\i}} \,\bnfor\! \typromise{A}
	\end{align*}
	\vspace{-5ex}
	\begin{align*}
	(op_1, \tymobile{A}_{op_1}),\, (op_2, \tymobile{A}_{op_2}),\, ... ,\, (op_n, \tymobile{A}_{op_k})
	\end{align*}
	 
	\caption{Prenosljivi tipi.}
	\label{fig:tipi-prenosljivi}
\end{figure}

Da zagotovimo, da nebi kakšni vrednosti, ki vsebuje zunanjo obljubo, dodeliti zavit tip, nekoliko spremenimo pravilo za izračun tipa spremenljivke. Hkrati tudi dodamo pravili za zavito vrednost in izračun odvijanje.

\begin{figure}[h]
	\centering
	\small
	\begin{mathpar}
		\coopinfer{}{\text{A prenosljiv ali } \blacksquare \notin \Gamma'
		}{
			\Gamma, x \of A, \Gamma' \types x \of A
		}
		\quad
		\coopinfer{}{\Gamma, \blacksquare \types V \of A
		}{
			\Gamma \types \tmboxed{V} \of \tyboxed{A}
		}
	    \quad
		\coopinfer{}{\Gamma \types V \of \tyboxed{A} \\ \Gamma, x \of A \types M \of B
		}{
			\Gamma \types \tmunbox{V}{x}{M} \of B
		}
	\end{mathpar}

	\caption{Pravila za izračun tipov prenosljivih izrazov.}
	\label{fig:tipi-pravila-prenosljivi}
\end{figure} 

\subsection{Spawn}


%\section{Implementacija}\label{sec:impl}

\subsection{Sistem tipov}\label{sec:tipi}

\subsubsection{Hindley–Milner}

\subsubsection{Bidirectional}


\subsection{Tolmač}\label{sec:interpreter}

\subsubsection{prej}

\subsubsection{potem}

Pokazat da, če naredimo korak v nek izračun, obstaja zaporedje korakov v prejšnji, ki vodijo v isti izračun.

Pokazati da potrebujemo manj korakov

\section{Zaključek}

Pokazali smo kako združiti asinhrono izvajanje programa v navezi z učinki in prestrezniki. Namesto da čakamo na odgovor v obliki prekinitve, izvajamo nadaljevanje. Šele ko pride ustrezna prekinitev pa nanjo reagiramo. Glavne ideje so formalizirane v \lae-računu. Rešili smo tri primanjkljaje osnovne verzije. Dodali smo rekurzivne prestreznike, da nimamo potrebe po prekomerni uporabi pomožnih rekurzivnih funkcij. Kadar je smiselno dovolimo uporabo tovorov višjega rega. Pri tem bistveno dovolimo uporabo funkcij v tovoru. Dodali smo tudi ustvarjanje novih procesov, s čimer lahko program naredimo še bolj asinhron in vzporeden. Pokazali smo izreka o napredku in ohranitvi za osnovno in razširjeno verzijo \lae-računa.

\subsection{Implementacija}\label{sec:implementacija}

Medtem ko je cilj \lae-računa, da pokaže pravilnost asinhronih konceptov prikazanih v prejšnjih poglavjih, je cilj jezika \aeff{}, da te koncepte lahko tudi preizkusimo in uporabimo. Zato ima jezik \aeff{} določene stvari, ki jih \lae-račun nima.

Jezik \aeff{} je implementiran v jeziku Ocaml. Implementacija je sestavljena iz treh glavnih delov.
Parserja, ki prevede tekst v abstraktno sintaktično drevo (AST) in preveri ustreznost sintakse.  
Pregledovalnika tipov, ki preveri, da ima dani program tip.
In tolmača, ki izvaja korake, dokler ne pride do rezultata.


Da \aeff{} postane bolj uporaben so mu dodane konstantni vrednosti $\true$ in $\false$, funkcije kot so $+, -, *, / ...$, reference in globalne spremenljivke.
Dodamo tudi tip seznamov. Za prazen seznam uporabimo konstruktor []. Elementa mu lahko dodajamo z $::$, recimo 1::2::3::[].
Sedaj imamo problem, ker uporabljamo [] tako za sezname kot za zavite vrednosti.
V poglavju~\ref{sec:razsirjen-lae}, kjer v dveh primerih uporabimo seznam, nadomestimo konstruktor za prazen seznam z $\mathsf{nil}$ in ohranimo [] za zavite vrednosti, ki so bistvene v tem delu.
Nasprotno v dejanski implementaciji \aeff{} raje ohranimo [] za prazen seznam in spremenimo sintakso zavite vrednosti v $[|V|]$. 
Dodana je kratka standardna knjižnica v kateri lahko najdemo osnovne funkcije kot so $\mathsf{nth}$, $\mathsf{map}$, $\mathsf{filter}$, $\mathsf{min}$, $\mathsf{max}$, $\mathsf{length}$, $\mathsf{fold\_left}$, $\mathsf{fold\_right}$...

%Izračun $M;N$ je sintaktičen sladkor za $\tmlet{\_}{M}{N}$, ki ga lahko uporabimo kadar rezultat izračuna $M$ ni uporabljen v $N$.

Varovan prestreznik $$\tmwithrecgu{op}{x}{r}{P(x)}{M}{p}{N},$$ je sintaktičen sladkor za $$\tmwithrec{op}{x}{r}{\tmif{P(x)}{M}{r()}}{p}{N}.$$
Nanj lahko gledamo kot na prestreznik, ki se sproži, le kadar je poleg pravega imena operacije tudi pravi tovor.

Prvotna implementacija jezika \aeff{} izračuna tip programa z Hindley–Milnerjevim algoritmom. 
Kasneje je bil sistem za preverjanje tipov spremenjen, da je uporabljal dvosmerni sistem tipov (bidirectional typing), ki temelji na algoritmu iz članka~\cite{bidirectional}.
Glavna ideja dvosmernega sistema tipov je, da lahko dinamično izmenjuje med preverjanjem in računanjem tipa. Kadar so tipi pripisani izrazom, tip le preverimo, kar je enostavneje, kot ga izračunati in v primeru napake lahko izpiše boljše poročilo o napaki. Vendar pa pripisovanje tipov ni obvezno in ga v tem primeru lahko izračunamo. Posledično lahko uporabnik, poda tip le nekaterim delom programa, preostalim pa ne.
Tako prva kot druga verzija ne preverjata tipe učinkov, ampak le standardni tip.

\subsection{Bodoče delo}

Še zmeraj so določene stvari nenarejene ali pa bi jih lahko izboljšali.

Čeprav ima \aeff{} možnost vzporednega izvajanja in smo to do sedaj v primerih programov predpostavljali, sama implementacije zaenkrat le simulira vzporedno izvajanje.
V bodoče bi se lahko tolmač napisalo v razširitvi Multicore OCaml, ki omogoča vzporedno izvajanje, in posledično bi lahko tudi jezik \aeff{} podpiral vzporedno izvajanje. 

V implementaciji manjka preverjanje tipov učinkov. Posledično nimamo koristi, ki jih tipi učinkov prinesejo. Ena izmed koristi je sledeča optimizacija. Če imamo $\tmopin{op}{V}{M}$, kjer $\Gamma \types M \of \tycomp{A}{\o, \i}$ in $\i(\op) = \bot$, potem lahko naredimo korak direktno v $M$, saj iz $\i(\op) = \bot$ vemo, da $M$ ne vsebuje prestreznika za to prekinitev. 

Naslednja optimizacija bi bila da, namesto da se signali počasi širijo navzven do nivoja procesov korak za korakom, bi se lahko razširili v enem samem velikem koraku.

Sedaj ko imamo dinamično ustvarjanje procesov, lahko to potencialno ustvari zelo veliko novih procesov. Tej novo ustvarjeni procesi se običajno izvedejo relativno hitro v primerjavi z ročno ustvarjenimi. Prav tako se običajno bistvene informacije prenesejo v druge procese z učinki in nas njihov rezultat ne zanima. Posledično imamo potencialno veliko procesov, ki bodo prejemali prekinitve, kar lahko drastično upočasni izvajanje. Potencialna rešitev bi bila možnost odstranitve posameznega procesa.

\section*{Slovar strokovnih izrazov}



\geslo{await}{blokada}
\geslo{bidirectional type system}{dvosmerni sistem tipov}
\geslo{boxed type}{zavit tip}
\geslo{ground type}{osnovni tip}
\geslo{handler}{prestreznik}
\geslo{interrupt}{prekinitev}
\geslo{join-semilattice}{$\sqcup$-polmreža}
\geslo{mobile type}{prenosljiv tip}
\geslo{promise}{obljuba}
\geslo{signal}{signal}
\geslo{spawn}{dinamični proces}
\geslo{unbox}{odvijanje}


%guarded promise = varovana obljuba
%abstraction = abstrakcija
%spawn =  ustvaritev novega procesa
%interpreter = tolmač
%type sistem = sistem tipov
%computation = izračun
%calculus = račun
%unit = enota
%Syntactic sugar = sintaktičen sladkor
%black box in types = kocka 
%recursion = rekurzija
%bidirectional = dvosmerni?
%small step semantics = op. sem. malih korakov
%pattern matching = ujemanje vzorca
%substitution = substitucija
%substitute = substituirati
%source code = izvorna koda
%machine code = strojna koda
%shared memory = skupni spomin
%context = kontekst
%true/false = resnica/neresnica
%runtime error = napaka ob izvajanju
%typing rules = pravila za tipe
%interupt propagation = širitev prekinitve
%signal hoisting = dvig signala
%broadcasting = oddajanje signala
%evaluation context rule = Vrednotenje v kontekstu
%turing complete = turingovo poln


\cleardoublepage                           % na desni strani
\phantomsection                            % da prav delujejo hiperlinki
\addcontentsline{toc}{section}{\bibname}   % dodajmo v kazalo
\bibliographystyle{fmf-sl}                 % uporabljen stil je v datoteki fmf-sl.bst, na voljo tudi angleška verzija
\bibliography{\literatura}                 % literatura je v datoteki, definirani na začetku
% TeXStudio zmede \ zgoraj, tako da lahko notri napišeš dejansko ime .bib datoteke, če ti
% ne delajo predlogi citatov.

% Za stvarno kazalo
\cleardoublepage                           % na desni strani
\phantomsection                            % da prav delujejo hiperlinki
\addcontentsline{toc}{section}{\indexname} % dodajmo v kazalo
\printindex

\end{document}
