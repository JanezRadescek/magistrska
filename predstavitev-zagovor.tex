\documentclass{beamer}
\usepackage[slovene]{babel}
\usepackage[utf8]{inputenc}
\usepackage[T1]{fontenc}

\usetheme{Montpellier} %beamer
\usecolortheme{beaver}

\usepackage{listings} %koda
\lstset{
	escapeinside={(*@}{@*)},          % if you want to add LaTeX within your code
}
\usepackage{mathpartir} % inference rules
\usepackage{mathtools} % mathllap
\usepackage{rotating} % down facing reduction

% bold matematika znotraj \textbf{ }, tudi v naslovih, kot \omega spodaj
\makeatletter \g@addto@macro\bfseries{\boldmath} \makeatother



% ukazi za matematična okolja
\theoremstyle{definition} % tekst napisan pokončno
\newtheorem{definicija}{Definicija}[section]
\newtheorem{primer}[definicija]{Primer}
\newtheorem{opomba}[definicija]{Opomba}
\newtheorem{aksiom}{Aksiom}

\theoremstyle{plain} % tekst napisan poševno
\newtheorem{lema}[definicija]{Lema}
\newtheorem{izrek}[definicija]{Izrek}
\newtheorem{trditev}[definicija]{Trditev}
\newtheorem{posledica}[definicija]{Posledica}

% !TEX root = paper.tex

% Any macro that is actually used should have a comment explaining what it is for.
% Please fight macro pollution and remove the macros that are not used.

\newcommand{\defeq}{\mathrel{\overset{\text{\tiny def}}{=}}} % Definitional equality

\newcommand{\defiff}{\mathrel{\overset{\text{\tiny def}}{\Longleftrightarrow}}} % Definitional iff

\newcommand{\pl}[1]{\textsc{#1}} % the name of a programming language

\newcommand{\lambdaAEff}{$\lambda_{\text{\ae}}$} % the name of the calculus

\newcommand{\lae}[0]{$\lambda_{\text{\ae}}$} %shorter name
\newcommand{\aeff}[0]{Æff} % the name of the language

\newcommand{\rulename}[1]{\textsc{#1}} % For typing/results rules 

% BNF grammars
\newcommand{\bnfis}{\mathrel{\;{:}{:}{=}\ }}
\newcommand{\bnfor}{\mathrel{\;\big|\ \ }}

%%%%% Semantic concepts

%%% Sets

\newcommand{\One}{\mathbb{1}} % singleton set as denotation of unit type
\newcommand{\one}{\star} % canonical element of the singleton set
\newcommand{\Zero}{\mathbb{0}} % empty set as denotation of empty type

\newcommand{\Bool}{\mathbb{B}} % two-element set of booleans
\newcommand{\true}{\mathbf{true}} % constant true
\newcommand{\false}{\mathbf{false}} % constant false

\newcommand{\expto}{\Rightarrow} % set exponentiation
\newcommand{\lam}[1]{\lambda #1 \,.\,} % lambda abstraction
\newcommand{\pair}[2]{\langle #1 , #2 \rangle} % pairing

\newcommand{\lifted}[1]{#1_\bot} % lifting monad
\newcommand{\idte}[4]{\mathbf{ifdef}~#1~\mathbf{then}~#2 \mapsto #3~\mathbf{else}~#4} % test if element of a lifted set is defined (non-bottom) or not, and then use it in the then branch

\newcommand{\ite}[3]{\mathbf{if}~#1~\mathbf{then}~#2~\mathbf{else}~#3} % if-then-else used in semantic definitions


%%% Signatures

\newcommand{\Tree}[2]{\mathrm{Tree}_{#1}\left(#2\right)} % The tree algebra for an operation signature
\newcommand{\retTree}[1]{\mathsf{return}\,#1} % the inclusion of generators into trees

\newcommand{\opsym}[1]{\mathsf{#1}} % a custom operation symbol
\newcommand{\op}{\opsym{op}} % a generic operation symbol

\newcommand{\sig}{\Sigma} % the global signature of signal and interrupt names

\renewcommand{\o}{o} % effect annotation describing possible outgoing operations
\renewcommand{\i}{\iota} % effect annotation describing possible incoming operations

\newcommand{\opincomp}[2]{{\mathsf{#1}}\,{\tmkw{\downarrow}}\,#2} % action of incoming interrupt on computation types
\newcommand{\opincompp}[2]{{\mathsf{#1}}\,{\tmkw{\downarrow\downarrow}}\,#2} % action of a list of incoming interrupts on computation types

\newcommand{\bb}[0]{\blacksquare}   %black box needed for mobile types


%%% Theories
\newcommand{\eq}{\mathrm{Eq}} % a set of equations

\newcommand{\FreeAlg}[2]{\mathrm{Free}_{#1}\left(#2\right)} % Free algebra for a signature generated by a set
\newcommand{\lift}[1]{#1^\dagger} % the Kleisli lifting of a map
\newcommand{\freelift}[1]{#1^\ddagger} % the lifting of a map induced by the free model property

\newcommand{\M}{\mathcal{M}} % a generic model for a theory
\newcommand{\Mcarrier}{\vert \mathcal{M} \vert} % the carrier of a generic model

\newcommand{\T}{T} % A generic monad


%%% Example effect theories

\newcommand{\sigget}{\mathsf{get}}
\newcommand{\sigset}{\mathsf{set}}


%%%%% Types

\newcommand{\at}{\mathbin{!}} % the ! sign, with proper spacing
\newcommand{\att}{\mathbin{!!}} % the !! sign, with proper spacing

%% Value types

\newcommand{\tysym}[1]{\mathsf{#1}}
\newcommand{\tybase}{\tysym{b}} % a base type
\newcommand{\tyunit}{\tysym{1}} % the unit ground type
\newcommand{\tyint}{\tysym{int}} % the integer ground type
\newcommand{\tystring}{\tysym{string}} % the integer ground type
\newcommand{\tylist}[1]{\tysym{list}~\tysym{#1}} % the list ground type
\newcommand{\tyempty}{\tysym{0}} % the empty ground type
\newcommand{\typrod}[2]{#1 \times #2} % product type
\newcommand{\tysum}[2]{#1 + #2} % sum type
\newcommand{\tyfun}[2]{#1 \to #2} % user function type
\newcommand{\typromise}[1]{\langle #1 \rangle} % type of promises
\newcommand{\tyboxed}[1]{[#1]} % type of boxed value
\newcommand{\tymobile}[1]{\widetilde{#1}} % type of boxed value

%% Computation types

\newcommand{\tycomp}[2]{#1 \at (#2)} % computation type

%% Process types

\newcommand{\tyrun}[2]{#1 \att (#2)} % type of the run M process
\newcommand{\typar}[2]{(#1 \mathbin{\tmkw{\vert\vert}}  #2)} % type of parallel processes
\newcommand{\tyC}{C} % meta variable ranging over process types
\newcommand{\tyD}{D} % meta variable ranging over process types

%%%%% Display of source code in math mode

\newcommand{\tm}[1]{\mathsf{#1}} % the source code font

%
\definecolor{keywordColor}{cmyk}{0.9, 0.4, 0.1, 0.2} % i dont have that color otherwise
\newcommand{\tmkw}[1]{\tm{\color{keywordColor}#1}} % source code keyword, colored

\newcommand{\tmpromise}[1]{\langle #1 \rangle} % completed promise

%% values
\newcommand{\tmconst}[1]{\tm{#1}}
\newcommand{\tmunit}{()} % the element of the unit type
\newcommand{\tmpair}[2]{( #1 , #2 )} % ordered pair
\newcommand{\tminl}[2][]{\tmkw{inl}_{#1}\,#2} % left injection
\newcommand{\tminr}[2][]{\tmkw{inr}_{#1}\,#2} % right injection
\newcommand{\tmfun}[2]{{\mathop{\tmkw{fun}}}\; (#1) \mapsto #2} % function abstraction
\newcommand{\tmfunano}[2]{{\mathop{\tmkw{fun}}}\; #1 \mapsto #2} % function abstraction (no type annotation expected)
\newcommand{\tmfunrec}[3]{\tmkw{fun}\; \tmkw{rec}\; #1\; (#2) \mapsto #3} % recursive abstraction
\newcommand{\tmfunrecano}[3]{\tmkw{fun}\; \tmkw{rec}\; #1\; #2 \mapsto #3} % recursive abstraction (no type annotation expected)
\newcommand{\tmapp}[2]{#1\,#2} % application
\newcommand{\tmboxed}[1]{[#1]} % boxed value

%% computations
\newcommand{\tmunbox}[3]{\tmkw{unbox}\; #1\; \tmkw{as} \; \tmboxed{#2} \; \tmkw{in}\;#3} % unbox comand
\newcommand{\tmreturn}[2][]{\tmkw{return}_{#1}\, #2} % pure computation
\newcommand{\tmlet}[3]{\tmkw{let}\; #1 = #2 \;\tmkw{in}\; #3} % let-binding
\newcommand{\tmletrec}[5][]{\tmkw{let}\;\tmkw{rec}\; #2\; #3 #1 = #4 \;\tmkw{in}\; #5} % recursive definitions

\newcommand{\tmop}[4]{\tm{#1}\;(#2, #3. #4)} % operation call
\newcommand{\tmopin}[3]{\tmkw{\downarrow}\, \tm{#1}\,(#2, #3)} % incoming interrupt
\newcommand{\tmopout}[3]{\tmkw{\uparrow}\,\tm{#1}\, (#2, #3)} % outgoing signal
\newcommand{\tmopoutbig}[3]{\tmkw{\uparrow}\,\tm{#1}\, \big(#2, #3\big)} % outgoing signal with big brackets
\newcommand{\tmopoutgen}[2]{\tmkw{\uparrow}\,\tm{#1}\, #2} % generic variant of outgoing signal

\newcommand{\tmmatch}[3][]{\tmkw{match}\;#2\;\tmkw{with}\;\{#3\}_{#1}} % match statement

\newcommand{\tmawait}[3]{\tmkw{await}\;#1\;\tmkw{until}\;\tmpromise{#2}\;\tmkw{in}\;#3} % awaiting for a promise to be completed

\newcommand{\tmwith}[5]{\tmkw{promise}\; (\tm{#1}\; #2 \mapsto #3)\; \tmkw{as}\; #4\; \tmkw{in}\; #5} % interrupt hook
\newcommand{\tmwithrec}[6]{\tmkw{promise}\; (\tm{#1}\; #2\; #3 \mapsto #4)\; \tmkw{as}\; #5\; \tmkw{in}\; #6} % rec interrupt hook
\newcommand{\tmwithrecgu}[7]{\tmkw{promise}\; (\tm{#1}\; #2\; #3\; \tmkw{when}\; #4 \mapsto #5)\; \tmkw{as}\; #6\; \tmkw{in}\; #7} % rec interrupt hook with guard
%\newcommand{\tmwith}[6]{\tmkw{promise}\; (\tm{#1}\; #2 \mapsto #3)\; \tmkw{as}\; #4 \of \typromise{#5}\; \tmkw{in}\; #6} % interrupt hook
\newcommand{\tmspawn}[2]{\tmkw{spawn} (#1, #2)} % Spawn new process

%% process

\newcommand{\tmrun}[1]{\tmkw{run}\; #1} % running a computation as a process
\newcommand{\tmpar}[2]{#1 \mathbin{\tmkw{\vert\vert}} #2} % parallel composition of processes

%%% Operational semantics

\newcommand{\reduces}{\leadsto} % small-step reduction
\newcommand{\tyreduces}{\rightsquigarrow} % reduction of process types

\newcommand{\E}{\mathcal{E}} % evaluation context for computations
\renewcommand{\H}{\mathcal{H}} % signal hoisting context
\newcommand{\F}{\mathcal{F}} % evaluation context for processes

%%% Typing rules

\newcommand{\types}{\vdash} % typing judgement
\newcommand{\of}{\mathinner{:}} % the colon in a typing judgement

\newcommand{\sub}{\sqsubseteq} % subtyping relation

\definecolor{rulenameColor}{cmyk}{0.1, 0.1, 0.1, 0.4} % i dont have that color otherwise
\newcommand{\coopinfer}[3]{\inferrule*[Lab={\color{rulenameColor}#1}]{#2}{#3}}

%%% Meta-theory

\makeatletter
\newcommand{\hourglass}{}                  % hourglass symbol for classifying temporarity blocked computations
\DeclareRobustCommand{\hourglass}{\mathrel{\mathpalette\hour@glass\relax}}

\newcommand\hour@glass[2]{%
  \vcenter{\hbox{%
    \rotatebox[origin=c]{90}{\scalebox{0.8}{$\m@th#1\bowtie$}}%
  }}%
}
\makeatother

\newcommand{\awaiting}[2]{#1 \hourglass #2} % computations blocked on awaiting a particular promise variable to be fulfilled

\newcommand{\CompResult}[2]{\mathsf{CompRes}\langle#1 \,\vert\, #2\rangle} % top-level result forms of individual computations
\newcommand{\RunResult}[2]{\mathsf{RunRes}\langle#1 \,\vert\, #2\rangle} % local (under-signal) result forms of individual computations

\newcommand{\Result}[2]{\mathsf{Res}\langle#1 \,\vert\, #2\rangle} % top-level result forms of computations

\newcommand{\ProcResult}[1]{\mathsf{ProcRes}\langle #1 \rangle} % top-level result forms of parallel processes
\newcommand{\ParResult}[1]{\mathsf{ParRes}\langle #1 \rangle} % intermediate result forms of parallel processes

%%% Maths

\newcommand{\cond}[3]{\mathsf{if}\;#1\;\mathsf{then}\;#2\;\mathsf{else}\;#3} % single line conditional

\newcommand{\carrier}[1]{\vert #1 \vert} % carrier of a cpo
\newcommand{\order}[1]{\sqsubseteq_{#1}} % partial order of a cpo
\newcommand{\lub}[1]{\bigsqcup_n \langle #1 \rangle} % least upper bound of an omega-chain

\newcommand{\Pow}[1]{\mathcal{P}(#1)} % powerset
\newcommand{\sem}[1]{[\![#1]\!]} % semantic bracket

\makeatletter
\providecommand*{\cupdot}{%     % disjoint union of sets
  \mathbin{%
    \mathpalette\@cupdot{}%
  }%
}
\newcommand*{\@cupdot}[2]{%
  \ooalign{%
    $\m@th#1\cup$\cr
    \hidewidth$\m@th#1\cdot$\hidewidth
  }%
}
\makeatother


%%% Redex highlighting

\definecolor{redexColor}{rgb}{0.83, 0.83, 0.83} % the color of highlighted redexes
\newcommand{\highlightgray}[1]{{\setlength{\fboxsep}{1.5pt}\colorbox{redexColor}{$#1$}}} % highlight redexes with gray(ish) background
\newcommand{\highlightwhite}[1]{{\setlength{\fboxsep}{1.5pt}\colorbox{white}{$#1$}}} % highlight redexes with white background


%%%%%% Highlighting important parts of proofs
\definecolor{kost}{RGB}{240, 220, 180}
\newcommand{\sitem}{\colorbox{kost}{$\odot$}}
\newcommand{\sitemm}{\textcolor{red}{$\odot$}}

% ukaz za slovarsko geslo / angleško-slovenski slovar
\newcommand{\geslo}[2]{\noindent\textbf{#1}\hspace*{3mm}\hangindent=\parindent\hangafter=1 #2\par}



% todo notes to comment the code when working in a group
\usepackage{todonotes}
\definecolor{jcyan}{cmyk}{1, 0, 0.15, 0.05}
\newcommand\mP[1]{\todo[inline,color=red]{#1 -MP}}	% comments by matija
\newcommand\jR[1]{\todo[inline,color=jcyan]{#1 -JR}} % comments by janez


\begin{document}
	
	\title{Asinhroni algebrajski učinki}
	\author[Janez Radešček]{Janez Radešček \\[3mm] Mentor: doc.~dr.~Pretnar~Matija\\[3mm] Somentor: asist.~raz.~dr.~Ahman~Danel}
	\institute{FMF, Univerza v Ljubljani}
	\date{\today}
	
	\frame{\titlepage}

\begin{frame}[fragile]{Uvod}
	\begin{columns}[T]
		\tiny
		\begin{column}{0.35\textwidth}
			\begin{lstlisting}
x' = priblizek 42
zaslon x'
x  = funkcija 42
zaslon x

nadaljevanje1 ()
nadaljevanje2 x
			\end{lstlisting}
		\end{column}
		\begin{column}{0.65\textwidth}
			\begin{lstlisting}
operation opravilo : int
operation rezultat : int

run 
	send opravilo 42;
	let x' = priblizek 42 in
	zaslon x';
	promise (rezultat x ->
		zaslon x;
		<x>
	) as p in
	nadaljevanje1 ();
	await p until <y> in
	nadaljevanje2 y

run 
	promise(opravilo y ->
		let x = funkcija y in
		send rezultat x;
		<()>
	) as _ in
	()
			\end{lstlisting}
		\end{column}
	\end{columns}
\end{frame}

\begin{frame}[fragile]{Uvod}
	\jR{Dodaj levo zgoraj asinhron primer.}
\tiny
\begin{center}
	\begin{tabular}{ c c c }
		Proces 1: & & Proces 2:
		\\
		\\
		$\tmkw{\uparrow}\, \mathsf{opravilo}\,42;M_1$ & $\optright{opravilo}{42}$ & $\tmwith{opravilo}{y}{N_1}{\_}{()}$
		\\
		\dreduces && \dreduces
		\\
		$\tmlet{x'}{\mathsf{priblizek}\ 42}{M_1'}$ & & $N_1[42/y]$
		\\
		\dreduces && \dreduces
		\\
		$\mathsf{zaslon}\ x';M_1''$ & & $N_1'$
		\\
		\dreduces && \dreduces
		\\
		$\tmwith{rezultat}{x}{N_2}{p}{M_2}$ &  & $N_1''$
		\\
		\dreduces && \dreduces
		\\
		$\tmwith{rezultat}{x}{N_2}{p}{\colorbox{kost}{$M_2'$}}$ & $\optleft{rezultat}{X}$ & $\tmkw{\uparrow}\, \mathsf{rezultat}\,X;\tmreturn{()}$
		\\
		\dreduces && \dreduces
		\\
		$\tmlet{p}{N_2[X/x]}{M_2'}$ & & $\tmreturn{()}$
		\\
		\dreduces && \dreduces
		\\
		... && ...
	\end{tabular}
\end{center}
\end{frame}


	\begin{frame}{Račun \lae{}}
		\begin{itemize}
			\item Programski jezik \aeff{} temelji na računu \lae{}.
			\item Račun \lae{} je razširitev Levy $\lambda$ računa.
			\item Izrazi, operacijska semantika in tipi.
		\end{itemize}
	\end{frame}


	\begin{frame}{Izrazi}


		\begin{columns}[T]
			\begin{column}{0.22\textwidth}
				\begin{figure}[hp]
					\parbox{\textwidth}{
						\centering
						\tiny
						\begin{align*}
						\intertext{\colorbox{kost}{\textbf{Procesi}}}
						P, Q
						\bnfis & \tmrun M &  \\
						\bnfor & \tmpar P Q &  \\
						\bnfor & \tmopout{op}{V}{P} &  \\
						\bnfor & \tmopin{op}{V}{P}  & 
						\end{align*}
					} 
				\end{figure}
			\end{column}
		
			\begin{column}{0.39\textwidth}
				\begin{figure}[hp]
					\parbox{\textwidth}{
						\centering
						\tiny
						\begin{align*}
						\intertext{\textbf{Izračuni}}
						M, N
						\bnfis& \tmreturn{V}                             \\
						\bnfor& \tmlet{x}{M}{N}                          \\
						\bnfor& V\,W                                   \\
						\bnfor& \tmmatch{V}{\tmpair{x}{y} \mapsto M}    \\
						\bnfor& \tmmatch[]{V}{}                         \\
						\bnfor& \tmmatch{V}{\tminl{x} \mapsto M, \tminr{y} \mapsto N}	\\
						\bnfor& \colorbox{kost}{$\tmopout{op}{V}{M}$}       \\
						\bnfor& \colorbox{kost}{$\tmopin{op}{V}{M}$}          \\
						\bnfor& \colorbox{kost}{$\tmwith{op}{x}{M}{p}{N}$}      \\
						\bnfor& \colorbox{kost}{$\tmawait{V}{x}{M}$}           
						\end{align*}
					} 
				\end{figure}
			\end{column}
		
			\begin{column}{0.35\textwidth}
				\begin{figure}[hp]
					\parbox{\textwidth}{
						\centering
						\tiny
						\begin{align*}
						\intertext{\textbf{Vrednosti}}
						V, W
						\bnfis& n \bnfor\! \true \bnfor\! \false        \\
						\bnfor& x                                       \\
						\bnfor& \tmunit \bnfor\! \tmpair{V}{W}          \\
						\bnfor& \tminl[Y]{V} \bnfor\! \tminr[X]{V}      \\
						\bnfor& \tmfun{x}{M}                        \\
						\bnfor& \tmfunrec{f}{x}{M}                   \\
						\bnfor& \colorbox{kost}{$\tmpromise{V}$}                          
						\end{align*}
					} 
				\end{figure}
			\end{column}
			
		\end{columns}
		
	\end{frame}



	\begin{frame}{Standardna operacijska semantika izračunov}
		\begin{figure}[tp]
			\tiny
			\begin{align*}
			\tmapp{(\tmfun{x \of X}{M})}{V} &\reduces M[V/x]
			\\
			\tmapp{(\tmfunrec{f}{x \of X}{M})}{V} &\reduces M[V/x, (\tmfunrec{f}{x \of X}{M})/f]
			\\
			\tmlet{x}{(\tmreturn V)}{N} &\reduces N[V/x]
			\\
			\tmmatch{\tmpair{V}{W}}{\tmpair{x}{y} \mapsto M} &\reduces M[V/x, W/y]
			\\
			\mathllap{\tmmatch{(\tminl[Y]{V})}{\tminl{x} \mapsto M, \tminr{y} \mapsto N}} &\reduces	M[V/x]
			\\
			\mathllap{\tmmatch{(\tminr[X]{W})}{\tminl{x} \mapsto M, \tminr{y} \mapsto N}} &\reduces	N[W/y]
			\end{align*}
		\end{figure}
	\end{frame}


	\begin{frame}{\lae{} operacijska semantika izračunov}
		\begin{figure}[tp]
			\tiny
			\begin{align*}
			\intertext{\textbf{Algebraičnost signala, prestreznika in blokade}}
			\tmlet{x}{(\tmopout{op}{V}{M})}{N} &\reduces \tmopout{op}{V}{\tmlet{x}{M}{N}}
			\\
			\tmlet{x}{(\tmwith{op}{y}{M}{p}{N_1})}{N_2} &\reduces \tmwith{op}{y}{M}{p}{(\tmlet{x}{N_1}{N_2})}
			\\
			\tmlet{x}{(\tmawait{V}{y}{M})}{N} & \reduces \tmawait{V}{y}{(\tmlet{x}{M}{N})}
			\\[1ex]
			\intertext{\textbf{Komutativnost signala in prestreznika}}
			\tmwith{op}{x}{M}{p}{\tmopout{op'}{V}{N}} &\reduces \tmopout{op'}{V}{\tmwith{op}{x}{M}{p}{N}}
			\\[1ex]
			\intertext{\textbf{Komutativnost blokade in prekinitve}}
			\tmopin{op}{V}{\tmawait{W}{x}{M}} &\reduces \tmawait{W}{x}{\tmopin{op}{V}{M}}
			\\[1ex]
			\intertext{\textbf{Širitev prekinitve}}
			\tmopin{op}{V}{\tmreturn W} &\reduces \tmreturn W
			\\
			\tmopin{op}{V}{\tmopout{op'}{W}{M}} &\reduces \tmopout{op'}{W}{\tmopin{op}{V}{M}}
			\\
			\tmopin{op}{V}{\tmwith{op}{x}{M}{p}{N}} &\reduces \tmlet{p}{M[V/x]}{\tmopin{op}{V}{N}}
			\\
			\tmopin{op'}{V}{\tmwith{op}{x}{M}{p}{N}} &\reduces \tmwith{op}{x}{M}{p}{\tmopin{op'}{V}{N}}
			\quad {\color{rulenameColor}(\op \neq \op')}
			\\[1ex]
			\intertext{\quad\,\textbf{Čakanje na izpolnitev obljube}}
			\tmawait{\tmpromise V}{x}{M} &\reduces M[V/x]
			\end{align*}
		\end{figure}
	\end{frame}

	\begin{frame}{Evalvacija v okolju}
		\centering
		\tiny
		\textbf{Evalvacija v okolju}
		\begin{align*}
			\coopinfer{}{
				M \reduces M'
			}{
				\tmlet{x}{M}{N} \reduces \tmlet{x}{M'}{N}
			}
		\end{align*}
		\vspace{-4ex}
		\begin{align*}
			\coopinfer{}{
				M \reduces M'
			}{
				\tmopout{op}{V}{M} \reduces \tmopout{op}{V}{M'}
			}
			\qquad
			\coopinfer{}{
				M \reduces M'
			}{
				\tmopin{op}{V}{M} \reduces \tmopin{op}{V}{M'}
			}
		\end{align*}
		\vspace{-4ex}
		\begin{align*}
			\coopinfer{}{
				N \reduces N'
			}{
				\tmwith{op}{x}{M}{p}{N} \reduces \tmwith{op}{x}{M'}{p}{N}
			}
		\end{align*}
	\end{frame}


	\begin{frame}{\lae{} operacijska semantika procesov}
		\begin{figure}[tp]
			\parbox{\textwidth}{
				\centering
				\tiny
				\begin{minipage}[t]{0.4\textwidth}
					\centering
					\begin{align*}
					\shortintertext{\textbf{Posamezen proces}}
					\coopinfer{}{
						M \reduces N
					}{
						\tmrun M \reduces \tmrun N
					}
					\end{align*}
				\end{minipage}
				\qquad
				\begin{minipage}[t]{0.4\textwidth}
					\centering
					\begin{align*}
					\shortintertext{\textbf{Prehod}}
					\tmrun {(\tmopout{op}{V}{M})}  &\reduces \tmopout{op}{V}{\tmrun M}
					\\
					\tmopin{op}{V}{\tmrun M} &\reduces \tmrun {(\tmopin{op}{V}{M})}
					\end{align*}
				\end{minipage}
				
				
				%%%%
				\begin{minipage}[t]{0.4\textwidth}
					\centering
					\begin{align*}
					\shortintertext{\textbf{Oddajanje signala}}
					\tmpar{\tmopout{op}{V}{P}}{Q} &\reduces \tmopout{op}{V}{\tmpar{P}{\tmopin{op}{V}{Q}}}
					\\
					\tmpar{P}{\tmopout{op}{V}{Q}} &\reduces \tmopout{op}{V}{\tmpar{\tmopin{op}{V}{P}}{Q}}
					\end{align*}
				\end{minipage}
				\qquad
				\begin{minipage}[t]{0.4\textwidth}
					\centering
					\begin{align*}
					\shortintertext{\textbf{Širitev prekinitve}}
					\tmopin{op}{V}{\tmpar P Q} &\reduces \tmpar {\tmopin{op}{V}{P}} {\tmopin{op}{V}{Q}}
					\\
					\tmopin{op}{V}{\tmopout{op'}{W}{P}} &\reduces \tmopout{op'}{W}{\tmopin{op}{V}{P}}
					\end{align*}
				\end{minipage}
				
				%%%%
				\vspace{4ex}
				\textbf{Evalvacija v okolju}
				\vspace{-3ex}
				\begin{align*}
					\coopinfer{}{
						P \reduces P'
					}{
						\tmpar{P}{Q}  \reduces \tmpar{P'}{Q}
					}
					\qquad
					\coopinfer{}{
						Q \reduces Q'
					}{
						\tmpar{P}{Q}  \reduces \tmpar{P}{Q'}
					}
				\end{align*}
				\vspace{-9ex}
				\begin{align*}
					\coopinfer{}{
						P \reduces P'
					}{
						\tmopout{op}{V}{P}  \reduces \tmopout{op}{V}{P'}
					}
					\qquad
					\coopinfer{}{
						P \reduces P'
					}{
						\tmopin{op}{V}{P}  \reduces \tmopin{op}{V}{P'}
					}
				\end{align*}
			} 
		\end{figure}
	\end{frame}



	\begin{frame}{Tipi in operacije}
		\begin{figure}[tb]
			\parbox{\textwidth}{
				\centering
				\tiny
				\begin{align*}
				\text{Osnovni tipi vrednosti $\bar{A}$, $\bar{B}$}
				\bnfis & \tysym{int} \,\bnfor\! \tysym{bool} \,\bnfor\! \tyunit \,\bnfor\! \tyempty \, 
				          \bnfor\! \typrod{\bar{A}}{\bar{B}} \,\bnfor\! \tysum{\bar{A}}{\bar{B}}
				\\
				\text{Tipi vrednosti $A$, $B$}
				\bnfis & \bar{A} \, \bnfor\! \typrod{A}{B} \,\bnfor\! \tysum{A}{B} \,\bnfor\! \tyfun{A}{B} \,\bnfor\! \colorbox{kost}{$\typromise{A}$}
				\\
				\text{Tip izračuna} \bnfis& \tycomp{A}{\o,\i}
				\\
				\text{Tip procesa \tyC, \tyD}  \bnfis & \tyrun{A}{\o,\i} \,\bnfor\! \typar{\tyC}{\tyD}
				\end{align*}
			} 
		\end{figure}
	
		\begin{figure}
			\centering
			\tiny
			\begin{align*}
			\intertext{Operacije}
			(op_1, \bar{A}_{op_1}),\, (op_2, \bar{A}_{op_2}),\, ... ,\, (op_n, \bar{A}_{op_n})
			\end{align*}
			\vspace{-15ex}
		\end{figure}
	\end{frame}

	\begin{frame}{Pravila za tipe vrednosti}
		\begin{figure}[tp]
			\centering
			\tiny
			\begin{mathpar}
				\coopinfer{}{
				}{
					\Gamma \types n : \tysym{int}
				}
				\qquad
				\coopinfer{}{
				}{
					\Gamma \types \true : \tysym{bool}
				}
				\qquad
				\coopinfer{}{
				}{
					\Gamma \types \false : \tysym{bool}
				}
				\quad
				\coopinfer{}{
				}{
					\Gamma, x \of A, \Gamma' \types x : A
				}
				\quad
				\coopinfer{}{
				}{
					\Gamma \types \tmunit : \tyunit
				}
				\\
				\coopinfer{}{
					\Gamma \types V : A \\
					\Gamma \types W : B
				}{
					\Gamma \types \tmpair{V}{W} : \typrod{A}{B}
				}
				\quad
				\colorbox{kost}{$\coopinfer{}{
					\Gamma \types V : A
				}{
					\Gamma \types \colorbox{kost}{$\tmpromise{V}$} : \typromise A
				}$}
				\quad
				\coopinfer{}{
					\Gamma \types V : A
				}{
					\Gamma \types \tminl[B]{V} : A + B
				}
				\quad
				\coopinfer{}{
					\Gamma \types W : B
				}{
					\Gamma \types \tminr[A]{W} : A + B
				}
				\\
				\coopinfer{}{
					\Gamma, x \of A \types M : B
				}{
					\Gamma \types \tmfun{x : A}{M} : \tyfun{A}{B}
				}
				\quad
				\coopinfer{}{
					\Gamma, x \of A \types M : B
				}{
					\Gamma \types \tmfunrec{f}{x : A}{M} : \tyfun{A}{B}
				}
			\end{mathpar}
		\end{figure}
	\end{frame}

	\begin{frame}{Pravila za tipe izračunov}
		\begin{figure}[tp]
			\centering
			\tiny
			\begin{mathpar}
				\coopinfer{}{
					\Gamma \types V : A
				}{
					\Gamma \types \tmreturn{V} : A
				}
				\qquad
				\coopinfer{}{
					\Gamma \types M : A
					\\
					\Gamma, x \of A \types N : B
				}{
					\Gamma \types
					\tmlet{x}{M}{N} : B
				}
				\qquad
				\coopinfer{}{
					\Gamma \types V : \tyfun{A}{B} \\
					\Gamma \types W : A
				}{
					\Gamma \types \tmapp{V}{W} : B
				}
				\\
				\coopinfer{}{
					\Gamma \types V : \typrod{A}{B} \\
					\Gamma, x \of A, y \of B \types M : C
				}{
					\Gamma \types \tmmatch{V}{\tmpair{x}{y} \mapsto M} : C
				}
				\qquad
				\coopinfer{}{
					\Gamma \types V : \tyempty
				}{
					\Gamma \types \tmmatch[Z]{V}{} : Z
				}
				\\
				\coopinfer{}{
					\Gamma \types V : A + B \\\\
					\Gamma, x \of A \types M : C \\
					\Gamma, y \of B \types N : C \\
				}{
					\Gamma \types \tmmatch{V}{\tminl{x} \mapsto M, \tminr{y} \mapsto N} : C
				}
				\\
				\colorbox{kost}{$
					\coopinfer{}{
						\Gamma \types V : A_\op \\
						\Gamma \types M : B 
					}{
						\Gamma \types \tmopout{op}{V}{M} : B
					}	
				$}
				\qquad
				\colorbox{kost}{$
					\coopinfer{}{
						\Gamma \types V : A_\op \\
						\Gamma \types M : B 
					}{
						\Gamma \types \tmopin{op}{V}{M} : B
					}	
					$}

				\\
				\colorbox{kost}{$
					\coopinfer{}{
						\Gamma, x \of A_\op \types M : \typromise B \\
						\Gamma, p \of \typromise B \types N : C 
					}{
						\Gamma \types \tmwith{op}{x}{M}{p}{N} : C
					}	
				$}
				\\
				\colorbox{kost}{$
					\coopinfer{}{
						\Gamma \types V : \typromise A \\
						\Gamma, x \of A \types M : B 
					}{
						\Gamma \types \tmawait{V}{x}{M} : B
					}	
				$}
				
			\end{mathpar}
		\end{figure}
	\end{frame}


	\begin{frame}{Pravila za tipe procesov}
		\begin{figure}[tp]
			\centering
			\small
			\begin{mathpar}
				\coopinfer{}{
					\Gamma \types M : \tycomp{A}{\o,\i}
				}{
					\Gamma \types \tmrun{M} : \tyrun{A}{\o,\i}
				}
				\quad
				\coopinfer{}{
					\Gamma \types P : \tyC \\
					\Gamma \types Q : \tyD
				}{
					\Gamma \types \tmpar{P}{Q} : \typar{\tyC}{\tyD}
				}
				\\
				\coopinfer{}{
					\Gamma \types V : A_\op \\
					\Gamma \types P : \tyC
				}{
					\Gamma \types \tmopout{op}{V}{P} : \tyC
				}
				\quad
				\coopinfer{}{
					\Gamma \types V : A_\op \\
					\Gamma \types P : \tyC 
				}{
					\Gamma \types \tmopin{op}{V}{P} : \opincomp{op}{\tyC}
				}  
			\end{mathpar}
		\end{figure}
	\end{frame}


	\begin{frame}{Rezultati}
			
		%Običajno bi bil izračun return V rezultat. 
		%Izračun await blokira dokler ustrezen prestreznik ne izpolni pripadajoče obljube. Ni nujno da bo vsak prestreznik ipolnil obljubo. Zato rezultatom dodamo izračune, ki čakajo izpolnitev obljube.
		%Izračun je lahko v postopku izpolnitve obljube na katero čakamo. Takih izračunov ne damo med rezultatov.
		
		\begin{figure}
			\centering
			\tiny
			\textbf{Čakajoči izrazi}
			\begin{mathpar}
				\coopinfer{}{
				}{
					\awaiting p {\tmawait p x M}
				}
				\quad
				\coopinfer{}{
					\awaiting p M
				}{
					\awaiting p {\tmlet x M N}
				}
				\quad
				\coopinfer{}{
					\awaiting p M
				}{
					\awaiting p {\tmopin{op}{V}{M}}
				}
			\end{mathpar}
		
			\textbf{Delni rezultati}
			\begin{mathpar}
%				\coopinfer{}{
%					\CompResult {\Psi} {M}
%				}{
%					\CompResult {\Psi} {\tmopout {op} V M}
%				}
%				\quad
%				\coopinfer{}{
%					\RunResult {\Psi} {M}
%				}{
%					\CompResult {\Psi} {M}
%				}
%				\vspace{-1ex}
%				\\
				\coopinfer{}{
				}{
					\RunResult {\Psi} {\tmreturn V}
				}
				\quad
				\coopinfer{}{
					\RunResult {\Psi \cup \{p\}} {N}
				}{
					\RunResult {\Psi} {\tmwith {op} x M p N}
				}
				\quad
				\coopinfer{}{
					p \in \Psi \\
					\awaiting p M
				}{
					\RunResult {\Psi} {M}
				}
			\end{mathpar}
		
			%Proces je rezultat če so vsi signali prišli do korenin drevesa procesov in so vsi procesi
		
			\textbf{Rezultati}
			\begin{mathpar}
				\coopinfer{}{
					\ProcResult {P}
				}{
					\ProcResult {\tmopout {op} V P}
				}
				\qquad
				\coopinfer{}{
					\ParResult {P}
				}{
					\ProcResult {P}
				}
				\qquad
				\coopinfer{}{
					\RunResult {\emptyset} {M}
				}{
					\ParResult {\tmrun M}
				}
				\qquad
				\coopinfer{}{
					\ParResult P \\
					\ParResult Q
				}{
					\ParResult {\tmpar P Q}
				}
			\end{mathpar}
		
		\end{figure}
		
	\end{frame}




	

	
	\begin{frame}{Napredek}
		\begin{lema}[1]
			Če ima izračun $M$ v kontekstu $\Gamma$ tip $A$, je $M$ delni rezultat ali pa lahko naredi korak v izraz $M'$.
		\end{lema}
		
		%glede na čas dokaz tudi leme, vsaj deloma.
		
					
		\begin{izrek}[o napredku]
			Če ima proces $M$ v kontekstu $\Gamma$ tip $A$, je $M$ rezultat ali pa lahko naredi korak v izraz $M'$.
		\end{izrek}
		
		\begin{proof}
			... 
			%Predpostavimo da M ni rezultat. Hočemo dokazati da obstaja M* v katerega lahko naredimo korak. Sedaj dokazujemo z indukcijo na globino izpeljave tipa $A$. Recimo da smo na globini 1. Edina možnost, da smo dokazali A, je da smo uporabili pravilo tip run. Se pravi M'' ima tip A. Po lemmi 1 je M'' rezultat ali pa lahko naredi korak. Če bi bil M'' rezultat bi bil po pravilu rezultat run tudi M rezultat. Se pravi da M'' lahko naredi korak. Po pravilu za posamezen proces pa to pomeni da tudi M lahko naredi korak. Naj bo globina sedaj n>1. Tedaj smo A dokazali z pravilom tip vzporedna, tip proces signal ali tip proces prekinitev. Recimo da smo dokazali z pravilom tip vzporedna. A = (C || D). Če sta P in Q rezultata bi bil po pravilu rezultat vzporedna tudi M rezultat. Se pravi da eden izmed P ali Q ni rezultat. Po indukcijski predpostavki lahko tisti, ki ni rezultat naredi korak. Po pravilu za vrednotenje v kontekstu to pomeni da M lahko naredi korak. Podobno za  tip proces signal in tip proces prekinitev.
		\end{proof}
		%Kako lepo definirati globino iz dokaza??
		
	\end{frame}

	
		
	\begin{frame}{Ohranitev}
			
		\begin{lema}[2]
			Če ima izračun $M$ v kontekstu $\Gamma$ tip $A$ in naredi korak v izračun $M'$, ima $M'$ v kontekstu $\Gamma$ tip $A$
		\end{lema}	
				
		\begin{izrek}[o ohranitvi]
			Če ima proces $M$ v kontekstu $\Gamma$ tip $A$ in naredi korak v proces $M'$, ima $M'$ v kontekstu $\Gamma$ tip $A$.
		\end{izrek}

	\end{frame}



	\begin{frame}{Rekurzivni prestrezniki}
		\begin{figure}[hp]
			\parbox{\textwidth}{
				\centering
				\tiny
				\begin{align*}
				\shortintertext{\textbf{Izračuni}}
				M, N
				\bnfis& ... \,\bnfor \tmwithrec{op}{f}{x \of A}{M \of \typromise{B}}{p}{N}                   
				\end{align*}
			} 
		\end{figure}
	
		\begin{figure}[tp]
			\centering
			\tiny
			\begin{align*}
			\shortintertext{\textbf{Operacijska semantika}}
			\tmopin{op}{V}{\tmwithrec{op}{x}{r}{M}{p}{N}} &\reduces \tmlet{p}{M[V/x, R/r]}{\tmopin{op}{V}{N}} \\
			F &= \tmfun{x}{\tmwithrec{op}{x}{r}{M}{p}{\tmreturn{p}}} \\
			\tmopin{op'}{V}{\tmwithrec{op}{x}{r}{M}{p}{N}} &\reduces \tmwithrec{op}{x}{r}{M}{p}{\tmopin{op'}{V}{N}}
			\quad {\color{rulenameColor}(\op \neq \op')}
			\end{align*}
		\end{figure}
		
		\begin{figure}[tp]
			\centering
			\tiny
			\textbf{Pravila za tipe izračunov}
			\begin{mathpar}
		\coopinfer{}{
	({\o'} , {\i'}) \order{O \times I} \i\, (\op)\\
	\Gamma, x \of \tymobile{A}_\op, r \of \tyfun{\tyunit}{\tycomp{\typromise{B}}{\emptyset, \{\op \mapsto (\o', \i')\}}} \types M : \tycomp{\typromise B}{\o',\i'} \\
	\Gamma, p \of \typromise B \types N : \tycomp{C}{\o,\i}
}{
	\Gamma \types \tmwithrec{op}{x}{r}{M}{p}{N} \of \tycomp{C}{\o, \i}
}
			\end{mathpar}
		\end{figure}
		
	\end{frame}



	\begin{frame}{Prenosljivi tipi}

		\begin{figure}[hp]
			\parbox{\textwidth}{
				\centering
				\tiny
				\begin{align*}
				\shortintertext{\textbf{Vrednosti}}
				V, W
				\bnfis & ... \,\bnfor\! \tmboxed{V}     
				\\[5ex]
				\shortintertext{\textbf{Izračuni}}
				M, N
				\bnfis & ... \,\bnfor\! \tmunbox{V}{x}{M}
				\end{align*}
			} 
		\end{figure}
		
		\begin{figure}[tp]
			\centering
			\tiny
			\begin{align*}
			\shortintertext{\textbf{Operacijska semantika}}
			\tmunbox{V}{u}{M} & \reduces M[V/u]
			\end{align*}
		\end{figure}
	
		\begin{figure}[tb]
			\parbox{\textwidth}{
				\centering
				\tiny
				\begin{align*}
				\text{Prenosljivi tipi vrednosti $\bar{A}$, $\bar{B}$}
				\bnfis & ... \,\bnfor\! \tyboxed{A}
				\end{align*}
			} 
		\end{figure}
	
		
		\begin{figure}[tp]
			\centering
			\tiny
			\textbf{Pravila za tipe vrednosti}
			\begin{mathpar}
				\coopinfer{}{\text{A prenosljiv ali } \blacksquare \notin \Gamma'
				}{
					\Gamma, x \of A, \Gamma' \types x : A
				}
				\coopinfer{}{\Gamma, \blacksquare \types V : A
				}{
					\Gamma \types \tmboxed{V} : \tyboxed{A}
				}
			\end{mathpar}
		\end{figure}
		
	\end{frame}


	\begin{frame}{Dinamični procesi}
		
		\begin{figure}[hp]
			\parbox{\textwidth}{
				\centering
				\tiny
				\begin{align*}
				\shortintertext{\textbf{Izračuni}}
				M, N
				\bnfis & ... \,\bnfor\! \tmspawn{M}{N}
				\end{align*}
			} 
		\end{figure}
		
		\begin{figure}[tp]
			\centering
			\tiny
			\begin{align*}
			\shortintertext{\textbf{Operacijska semantika}}
			\tmrun{(\tmspawn{M}{N})} & \reduces \tmpar{\tmrun{M}}{\tmrun{N}}
			\end{align*}
		\end{figure}
	
	
		\begin{figure}[tp]
			\centering
			\tiny
			\textbf{Pravila za tipe računov}
			\begin{mathpar}
				\coopinfer{}{\Gamma, \blacksquare \types M : B \\ \Gamma \types N : A
				}{
					\Gamma \types \tmspawn{M}{N} : A
				}
			\end{mathpar}
		\end{figure}
		
	\end{frame}
	
	
	
	
	\begin{frame}{Implementacija in prihodnje delo}
		\begin{enumerate}
			\item \aeff{} ima v primerjavi z \lae{} nekaj sintaktičnih sladkorjev in dodatnih izrazov.

			\item Uporaba tipov učinkov v operacijski semantiki. Če $\Gamma \types M \of \tycomp{A}{\o,\i},\ \i(\op) = \bot$, potem $\opincomp{\op}{M} \reduces M$.
			
			\item Odstranjevanje procesov. Izračun $\tmkw{spawn}$ lahko ustvari veliko kratkotrajnih procesov.
		\end{enumerate}
	\end{frame}
	
	
	
	
	\begin{frame}{Osnovna literatura}
		\phantom{\cite{aeff}}
				
		\bibliographystyle{fmf-sl}
		\bibliography{literatura.bib}
	\end{frame}
	
	
\end{document}